% Sandia National Laboratories is a multimission laboratory managed and
% operated by National Technology & Engineering Solutions of Sandia, LLC, a
% wholly owned subsidiary of Honeywell International Inc., for the U.S.
% Department of Energy’s National Nuclear Security Administration under
% contract DE-NA0003525.

% Copyright 2002-2021 National Technology & Engineering Solutions of Sandia,
% LLC (NTESS).

\label{FFT}

Performs Fast Fourier Transform analysis of transient analysis output.

\begin{Command}

\format
\begin{alltt}
.FFT <ov> [NP=<value>] [WINDOW=<value>] [ALFA=<value>]
+ [FORMAT=<value>] [START=<value>] [STOP=<value>]
+ [FREQ=value] [FMIN=value] [FMAX=value]
\end{alltt}

\examples
\begin{alltt}
.FFT v(1)
.FFT v(1,2) NP=512 WINDOW=HANN
.FFT \{v(3)-v(2)\} START=1 STOP=2
\end{alltt}

\arguments

\begin {Arguments}

\argument{ov}
The desired solution output to be analyzed. Only one
output variable can be specified on a {\tt .FFT} line. However,
multiple {\tt .FFT} lines with the same output variable but, for
example, different windowing functions may be used in a netlist.
The available outputs are:

\begin{itemize}
  \item \texttt{V(<circuit node>)} the voltage at \texttt{<circuit node>}
  \item \texttt{V(<circuit node>,<circuit node>)} to output the voltage
    difference between the first \texttt{<circuit node>} and second
    \texttt{<circuit node>}
  \item \texttt{I(<device>)} the current through a two terminal device
  \item \texttt{I<lead abbreviation>(<device>)} the current into a particular lead
    of a three or more terminal device (see the Comments, below, for details)
  \item \texttt{N(<device parameter>)} a specific device parameter (see the
    individual devices in Section~\ref{Analog_Devices} for syntax)
\end{itemize}

\argument{NP}
The number of points in the FFT.  This value must be a power of 2.  If
the user-entered value is not a power of two then it will be rounded to
the nearest power of 2.  The minimum allowed value of {\tt NP} is 4.
The default value is 1024.

\argument{WINDOW}
The windowing function that will be applied to the sampled waveform
values.  The allowed values are as follows, where table~\ref{FFT_Window_Funcs}
gives their exact definitions:

\begin{itemize}
  \item \texttt{RECT} = rectangular window (default)
  \item \texttt{BART} = Bartlett (triangular) window
  \item \texttt{HANN} = Hanning window
  \item \texttt{HAMM} = Hamming window
  \item \texttt{BLACK} = Blackman window
  \item \texttt{HARRIS} = Blackman-Harris window
\end{itemize}

\argument{ALFA}
This parameter is supported for HSPICE compatibility. It currently
has no effect though, since the {\tt GAUSS} and {\tt KAISER} windows
are not supported.

\argument{FORMAT}
The allowed values are {\tt NORM} and {\tt UNORM}.  The default
value is {\tt NORM}. If {\tt NORM} is selected then the magnitude
values will be normalized to the largest magnitude.  If {\tt UNORM}
is selected then the actual magnitude values will be output instead.

\argument{START}
The start time for the FFT analysis.  The default value is the
start time for the transient analysis. {\tt FROM} is an allowed
synonym for {\tt START}.

\argument{STOP}
The end time for the FFT analysis.  The default value is the
end time for the transient analysis. {\tt TO} is an allowed
synonym for {\tt STOP}.

\argument{FREQ}
The ``first harmonic'' of the frequencies provided in the output
file.  The default value for {\tt FREQ} is {\tt 1/(STOP - START)}.
If {\tt FREQ} is given then it is rounded to the nearest integer
multiple of the default value.  The \Xyce{} Users' Guide\UsersGuide
provides an example.

\argument{FMIN}
This parameter is supported for HSPICE compatibility. It currently
has no effect though.

\argument{FMAX}
This parameter is supported for HSPICE compatibility. It currently
has no effect though.

\end{Arguments}

\comments
Multiple \texttt{.FFT} lines may be used in a netlist.  All results from FFT analyses
will be returned to the user in a file with the same name as the netlist file suffixed
with \texttt{.fftX} where {\tt X} is the step number.

\texttt{<lead abbreviation>} is a single character designator for individual
leads on a device with three or more leads.  For bipolar transistors these are:
c (collector), b (base), e (emitter), and s (substrate).  For mosfets, lead
abbreviations are: d (drain), g (gate), s (source), and b (bulk).  SOI
transistors have: d, g, s, e (bulk), and b (body).  For PDE devices, the nodes
are numbered according to the order they appear, so lead currents are
referenced like I1(\texttt{<device>}), I2(\texttt{<device>}), etc.

For this analysis, the phase data is always output in degrees.

\end{Command}

Table~\ref{FFT_Window_Funcs} gives the definitions of the window functions
implemented in \Xyce{}.  For HSPICE compatibility, the {\tt BLACK} window type
is actually the ``-67 dB Three-Term Blackman-Harris window''~\cite{Doerry2017}
rather than the ``Conventional Blackman Window'' used by Matlab and scipy.
The definition of the {\tt BART} window type ~\cite{oppenheimSchafer} was chosen
to match Matlab and scipy.  The \Xyce{} definition may differ from HSPICE.

\begin{longtable}[h] {>{\raggedright\small}m{0.75in}|>{\raggedright\let\\\tabularnewline\small}m{1.75in}
  |>{\raggedright\let\\\tabularnewline\small}m{3.5in}}
  \caption{.FFT Window Function Definitions (N = Number of Points).} \\ \hline
  \rowcolor{XyceDarkBlue}
  \color{white}\bf Value &
  \color{white}\bf Description &
  \color{white}\bf Definition \\ \hline \endfirsthead
  \rowcolor{XyceDarkBlue}
  \color{white}\bf Value &
  \color{white}\bf Description &
  \color{white}\bf Definition \\ \hline \endhead
  \label{FFT_Window_Funcs}

  RECT & Rectangular &
    $w(i)=1, \hspace*{0.1in}\textrm{for}~0 \leq i < N$ \\ \hline
  BART & Barlett~\cite{oppenheimSchafer} &
    $w(i) = \left\{\begin{array}{ll}\frac{2 i}{N-1},\hspace*{0.1in}\textrm{if}~i < 0.5 \cdot (N-1) \\ 2-\frac{2 i}{N-1},\hspace*{0.1in}\textrm{otherwise}\end{array}\right.$\\ \hline
  HANN & Hanning~\cite{oppenheimSchafer} &
    $w(i) = sin^2(\frac{\pi i}{N-1}), \hspace*{0.1in}\textrm{for}~0 \leq i < N$ \\ \hline
  HAMM & Hamming~\cite{oppenheimSchafer} &
    $w(i) = 0.54 - 0.46 \cdot cos(\frac{2 \pi i}{N-1}), \hspace*{0.1in}\textrm{for}~0 \leq i < N $ \\ \hline
  BLACK & -67 dB Three-Term Blackman-Harris window~\cite{Doerry2017} &
    $w(i) = 0.42323-0.49755 \cdot cos(\frac{2 \pi i}{N-1})+0.07922 \cdot cos(\frac{4 \pi i}{N-1}), \hspace*{0.1in}\textrm{for}~0 \leq i < N$  \\ \hline
  HARRIS & -92 dB Four-Term Blackman-Harris window~\cite{Doerry2017} &
    $w(i) = 0.35875 - 0.48829 \cdot cos(\frac{2 \pi i}{N-1}) + 0.14128 \cdot cos(\frac{4 \pi i}{N-1}) - 0.01168 \cdot cos(\frac{6 \pi i}{N-1}), \hspace*{0.1in}\textrm{for}~0 \leq i < N$  \\ \hline
\end{longtable}

\subsubsection{Additional FFT Metrics}
\label{FFT_metrics}
The following additional metrics will be sent to stdout if
\texttt{.OPTIONS FFT FFTOUT=1} is used in the netlist.

\subsubsection{Re-Measure}
\label{Measure_FFT_ReMeasure}
\index{measure fft!re-measure}
\Xyce{} can re-calculate (or re-measure) the values for {\tt .FFT} statements
using existing \Xyce{} output files.  Section~\ref{Measure_ReMeasure} discusses
this topic in more detail for both {\tt .MEASURE} and {\tt .FFT} statements. One
additional caveat is that \texttt{FFT\_ACCURATE} is set to 0 during the re-measure
operation.  This should have no impact of the accuracy of the re-measured results
if the output file was previously generated with \texttt{FFT\_ACCURATE} set to 1.

\subsubsection{Compatibility Between OUTPUT and FFT Options}
Some \texttt{.OPTIONS OUTPUT} settings are incompatible with using the default
setting of \texttt{.OPTIONS FFT FFT\_ACCURATE=1}.  The use of
\texttt{.OPTIONS OUTPUT OUTPUTTIMEPOINTS} is compatible.  However, the use of
\texttt{.OPTIONS OUTPUT INITIAL\_INTERVAL} is not.  In that latter case,
\texttt{.OPTIONS FFT FFT\_ACCURATE} will be automatically set to 0.
