% This table was generated by Xyce:
%   Xyce -doc O 1
%
\index{lossy transmission line!device model parameters}
\begin{DeviceParamTableGenerated}{Lossy Transmission Line Device Model Parameters}{O_1_Device_Model_Params}
ABS & Abs. rate of change of deriv. for bkpt & -- & 1 \\ \hline
C & Capacitance per unit length & F/m & 0 \\ \hline
COMPACTABS & special abstol for straight line checking & -- & 1e-12 \\ \hline
COMPACTREL & special reltol for straight line checking & -- & 0.001 \\ \hline
COMPLEXSTEPCONTROL & do complex time step control using local truncation error estimation & logical (T/F) & false \\ \hline
G & Conductance per unit length & $\mathsf{\Omega}^{-1}$ m$^{-1}$ & 0 \\ \hline
L & Inductance per unit length & Hm$^{-1}$ & 0 \\ \hline
LEN & length of line & m & 0 \\ \hline
LININTERP & use linear interpolation & logical (T/F) & false \\ \hline
MIXEDINTERP & use linear interpolation if quadratic results look unacceptable & logical (T/F) & false \\ \hline
NOSTEPLIMIT & don't limit timestep size based on the time constant of the line & logical (T/F) & false \\ \hline
QUADINTERP & use quadratic interpolation & logical (T/F) & true \\ \hline
R & Resistance per unit length & $\mathsf{\Omega}/$m & 0 \\ \hline
REL & Rel. rate of change of deriv. for bkpt & -- & 1 \\ \hline
STEPLIMIT & limit timestep size based on the time constant of the line & logical (T/F) & true \\ \hline
TRUNCDONTCUT & don't limit timestep to keep impulse response calculation errors low & logical (T/F) & false \\ \hline
TRUNCNR & use N-R iterations for step calculation in LTRAtrunc & logical (T/F) & false \\ \hline
\end{DeviceParamTableGenerated}
