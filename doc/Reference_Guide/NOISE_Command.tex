% Sandia National Laboratories is a multimission laboratory managed and
% operated by National Technology & Engineering Solutions of Sandia, LLC, a
% wholly owned subsidiary of Honeywell International Inc., for the U.S.
% Department of Energy’s National Nuclear Security Administration under
% contract DE-NA0003525.

% Copyright 2002-2025 National Technology & Engineering Solutions of Sandia,
% LLC (NTESS).


%%%
%%% Transient Analysis Table
%%%

Calculates the the small signal noise response of a circuit over a range of frequencies.
The .NOISE command can specify a linear sweep, decade logarithmic sweep, octave
logarithmic sweep, or a data table of multivariate values.

\begin{Command}

\format
\begin{alltt}
  .NOISE V(OUTPUT <, REF>) SRC <sweep type> <points value>
+ <start frequency value> <end frequency value>
\end{alltt}

\examples
\begin{alltt}
  .NOISE V(5) VIN LIN 101 100Hz 200Hz
  .NOISE V(5,3) V1 OCT 10 1kHz 16kHz
  .NOISE V(4) V2 DEC 20 1MEG 100MEG
  .NOISE V(4) V2 DATA=<table name>
\end{alltt}

\arguments

\begin{Arguments}

\argument{V(OUTPUT <,REF>)}
   The node at which the total output noise is desired. If REF is 
   specified, then the noise voltage V(OUTPUT) - V(REF) is calculated. By default, 
   REF is assumed to be ground. 

\argument{SRC}
   The name of an independent source to which input noise is referred. 

\argument{sweep type}

Must be \texttt{LIN}, \texttt{OCT}, or \texttt{DEC}, as described below.
\begin{description}

\item[\tt LIN] Linear sweep\\
The sweep variable is swept linearly from the starting to the ending value.

\item[\tt OCT] Sweep by octaves\\
The sweep variable is swept logarithmically by octaves.

\item[\tt DEC] Sweep by decades\\
The sweep variable is swept logarithmically by decades.

\item[\tt DATA] Sweep values from a table\\
Sweep variables are applied based on the rows of a data table.  This format
allows magnitude and phase to be swept in addition to frequency.  If using
this format, then the \texttt{V(OUTPUT <,REF>)} and \texttt{SRC} arguments
are still needed on the \texttt{.NOISE} line.

\end{description}

\argument{points value}
Specifies the number of points in the sweep, using an integer greater than or equal to 1.

\argument{\vbox{\hbox{start frequency value\hfil}\hbox{end frequency value}}}

The end frequency value must not be less than the start frequency value,
and both must be greater than zero. The whole sweep must include at
least one point.

\end{Arguments}

\comments

Noise analysis is a linear analysis. The simulator calculates the noise
response by linearizing the circuit around the bias point.

If specifying the sweep points using the \texttt{DATA} type, one can also
sweep the magnitude and phase of an AC source, as well as the values of
linear model parameters.  However, unlike the use of \texttt{DATA} for
\texttt{.STEP} and \texttt{.DC}, it is not possible to sweep nonlinear device
parameters.  This is because changing other nonlinear device parameters would
alter the correct DCOP solution, and the NOISE sweep happens after the DCOP
calculation in the analysis flow.  To sweep a nonlinear device parameter on
a NOISE problem, add a \texttt{.STEP} command to the netlist to provide an
outer parametric sweep around the analysis.

If \texttt{.DATA} is used with \texttt{.NOISE} then the integrals for the total
input noise and total output noise will only be calculated, and sent to stdout,
if the frequencies in the data table are monotonically increasing.

\index{\texttt{.PRINT}}\index{results!print}\index{\texttt{.PRINT}!\texttt{NOISE}}
A \texttt{.PRINT NOISE} must be used to get the results of the NOISE sweep
analysis.  See Section \ref{.PRINT}.

Noise analysis is a relatively new feature to \Xyce{}, so not all noise models
have been supported.

Power calculations (\texttt{P(<device>} and \texttt{W(<device>}) are not supported for any
devices for noise analysis.  Current variables (e.g., \texttt{I(<device>)} are only supported
for devices that have ``branch currents'' that are part of the solution vector. This includes
the V, E, H and L devices.  It also includes the voltage-form of the B device.

\end{Command}
