% Sandia National Laboratories is a multimission laboratory managed and
% operated by National Technology & Engineering Solutions of Sandia, LLC, a
% wholly owned subsidiary of Honeywell International Inc., for the U.S.
% Department of Energy’s National Nuclear Security Administration under
% contract DE-NA0003525.

% Copyright 2002-2025 National Technology & Engineering Solutions of Sandia,
% LLC (NTESS).


\label{DCOP_section}
\index{\texttt{.DCOP}}
\index{initial condition!DCOP}
\index{initial condition}

The \texttt{.DCOP} command allows the user to set initial conditions for operating 
point calculations from a file rather than via \texttt{.IC} or \texttt{.DCVOLT}
commands in the netlist.  These operating point conditions will be enforced the entire 
way through the nonlinear solve.  Initial conditions can be given for some or all of 
the circuit nodes.

The \texttt{.DCOP} command also allows the user to save the operating point conditions,
for all nodes in the netlist, to a file.  That file can be re-used in subsequent
simulations.
 
Consult the \Xyce{} User's Guide for more guidance on \texttt{.DCOP} usage.  

\begin{Command}
\format
\begin{alltt}
.DCOP <io type>=<file name>
\end{alltt}

\examples
\begin{alltt}
.DCOP input=initialConditions.op
.DCOP output=initialConditions.op
\end{alltt}

\arguments
\begin{Arguments}
\argument{io type}
The io type can be input or output.  Input reads in the initial conditions from the
specified file.  Output writes the initial conditions to the specified file.

\argument{file name}
Specifies the name of the file used for input or output.
\end{Arguments}

\comments
The \texttt{.DCOP} capability can only set voltage values, not current values.  See 
section \ref{IC_section} for more details on other limitations on initial conditions.

The input and output files are two-column text files. One column has the variable name, 
the other column has the value of that variable.

\end{Command}

