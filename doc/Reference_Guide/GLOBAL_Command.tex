% Sandia National Laboratories is a multimission laboratory managed and
% operated by National Technology & Engineering Solutions of Sandia, LLC, a
% wholly owned subsidiary of Honeywell International Inc., for the U.S.
% Department of Energy’s National Nuclear Security Administration under
% contract DE-NA0003525.

% Copyright 2002-2025 National Technology & Engineering Solutions of Sandia,
% LLC (NTESS).


\label{GLOBAL_section}

The \texttt{.GLOBAL} command provides another way to designate certain 
nodes as global nodes, besides starting their node name with the two 
characters ``\$G'' as discussed in section \ref{Voltage_Nodes}.  A typical 
usage of such global nodes is to define a VDD or VSS signal that all
subcircuits need to be able to access, but without having to provide 
VSS and VDD input nodes to every subcircuit. 

\begin{Command}

\format
\begin{alltt}
.GLOBAL <node1> [ node2 node3 ... ]
\end{alltt}

\examples
\begin{alltt}
.GLOBAL g1
.subckt rsub  a  b
Rab  a  b  2
* since node G1 is global, it may be used here without
* being listed on the .subckt line
Rbg  G1  b  3  
.ends
\end{alltt}

\comments
The name of the global node can be any legal node name, per 
section \ref{legalCharacters}.

\end{Command}

