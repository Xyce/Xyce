% Sandia National Laboratories is a multimission laboratory managed and
% operated by National Technology & Engineering Solutions of Sandia, LLC, a
% wholly owned subsidiary of Honeywell International Inc., for the U.S.
% Department of Energy’s National Nuclear Security Administration under
% contract DE-NA0003525.

% Copyright 2002-2025 National Technology & Engineering Solutions of Sandia,
% LLC (NTESS).


\index{\texttt{.HB}}
\index{analysis!HB} \index{harmonic balance analysis}
Calculates steady states of nonlinear circuits in the frequency domain.

\begin{Command}

\format
.HB <fundamental frequencies>

\examples
.HB 1e4

.hb 1e4 2e2
\arguments

\begin{Arguments}
\argument{fundamental frequencies}
Sets the fundamental frequencies for the analysis.

\end{Arguments}

\comments

Harmonic balance analysis calculates the magnitude and phase of voltages
and currents in a nonlinear circuit. Use a \texttt{.OPTIONS HBINT}
statement to set additional harmonic balance analysis options.

The \texttt{.PRINT HB}\index{\texttt{.PRINT}}\index{results!print}\index{\texttt{.PRINT}!\texttt{HB}}
statement must be used to get the results of the harmonic balance analysis. 
See section \ref{.PRINT}.

Some devices that may be expected to work in HB analysis do not at this time.  
  This includes some use cases of B sources (but not all).  A time-dependent B 
  source will not work with HB.  However, a B source that is purely dependent 
  (such as a nonlinear resistor) will work.    This same guidance applies to the 
  E,F,G, and H dependent sources.

\end{Command}
