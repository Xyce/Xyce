% Sandia National Laboratories is a multimission laboratory managed and
% operated by National Technology & Engineering Solutions of Sandia, LLC, a
% wholly owned subsidiary of Honeywell International Inc., for the U.S.
% Department of Energy’s National Nuclear Security Administration under
% contract DE-NA0003525.

% Copyright 2002-2024 National Technology & Engineering Solutions of Sandia,
% LLC (NTESS).


Extracts linear transfer parameters (S-, Y- and Z-parameters) for a general
multiport network.  Those parameters can be output in either
Touchstone format \cite{touchstone2_std_2009}.

\begin{Command}

\format
\begin{alltt}
.LIN [SPARCALC=<1|0>] [FORMAT=<TOUCHSTONE2|TOUCHSTONE>]
+ [LINTYPE=<S|Y|Z>] [DATAFORMAT=<RI|MA|DB>]
+ [FILE=<output filename>] [WIDTH=<print field width>]
+ [PRECISION=<floating point output precision>]
\end{alltt}

\examples
\begin{alltt}
.LIN
.LIN FORMAT=TOUCHSTONE DATAFORMAT=MA FILE=foo
\end{alltt}

\arguments

\begin{Arguments}

\argument{SPARCALC=<1|0>}
If this is set to 1 then the \texttt{LIN} analysis is done
at the frequency values specified on the \texttt{.AC} line.
The default value is 1.

\argument{FORMAT=<TOUCHSTONE2|TOUCHSTONE>} Output file format
\begin{description}
\item[\tt TOUCHSTONE] Output file is in Touchstone 1 format

\item[\tt TOUCHSTONE2] Output file is in Touchstone 2 format. The default
is \texttt{TOUCHSTONE2}.
\end{description}

\argument{LINTYPE=<S|Y|Z>} The type of parameter data (S, Y or Z) in the
output file.  The default is S.

\argument{DATAFORMAT=<RI|MA|DB>} Format for the S-, Y- or Z-parameter data

\begin{description}
\item[\tt RI] Real-imaginary format \\
The data is output as the real and imaginary parts for each
extracted S-, Y- or Z-parameter.  This is the default.

\item[\tt MA] Magnitude-angle format \\
The data is output as the magnitude and the phase angle of each
extracted S-, Y- or Z-parameter.  For compatibility with Touchstone formats,
the angle values are in degrees.

\item[\tt DB] Magnitude(dB)-angle format \\
The data is output as the magnitude (in dB) and the phase angle of
each extracted S-, Y- or Z-parameter.  For compatibility with Touchstone
formats, the angle values are in degrees.
\end{description}

\argument{FILE=<output filename>}
Specifies the name of the file to which the output will be written.
For HSPICE compatibility \texttt{FILENAME=} is an allowed synonym
for \texttt{FILE=} on \texttt{.LIN} lines.

\argument{WIDTH=<print field width>}

Controls the output width used in formatting the output.

\argument{PRECISION=<floating point precision>}

Number of floating point digits past the decimal for output data.

\end{Arguments}


\comments

The \texttt{.LIN} command line functions like a \texttt{.PRINT} line for
the extracted S-, Y- or Z-parameter data.  So, a netlist can have multiple
\texttt{.LIN} lines with different values for the \texttt{LINTYPE},
\texttt{DATAFORMAT} and \texttt{FILE} arguments on each line.  If there are
multiple \texttt{.LIN} lines in the netlist, then a linear analysis will
be performed if \texttt{SPARCALC=1} on any of those \texttt{.LIN} lines.

The default filename for both Touchstone formats is \texttt{<netlistName>.sNp}
where N is the number of ``ports'' (\texttt{P} devices) specified in the netlist.

The \Xyce{} Touchstone output is based on the Touchstone standard
\cite{touchstone2_std_2009}. So, it differs slightly from the
corresponding HSPICE output.  In particular, the full matrix of
S-, Y- or Z-parameters is always output.

The HSPICE \texttt{SPARDIGIT} and \texttt{FREQDIGIT} arguments are
not supported.  Instead, the \texttt{PRECISION} argument is used for
all of the output values.

The output of individual S-parameters via the \texttt{.PRINT AC} line is
supported.

If the \texttt{-r <raw-file-name>} and \texttt{-a} command line options
are used with \texttt{.LIN} with \texttt{SPARCALC=1} then \Xyce{} will
exit with a parsing error.

The \texttt{-o} command line option can be used with \texttt{.LIN}.
In that case, the output defaults to Touchstone 2 format and any
\texttt{FILE=<filename>} argument on the \texttt{.LIN} line is
ignored.

\end{Command}
