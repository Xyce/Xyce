% Sandia National Laboratories is a multimission laboratory managed and
% operated by National Technology & Engineering Solutions of Sandia, LLC, a
% wholly owned subsidiary of Honeywell International Inc., for the U.S.
% Department of Energy’s National Nuclear Security Administration under
% contract DE-NA0003525.

% Copyright 2002-2021 National Technology & Engineering Solutions of Sandia,
% LLC (NTESS).


User-defined global parameter, which can be time dependent, or can be used in \texttt{.STEP} loops.

\begin{Command}

\format
.GLOBAL\_PARAM [<name>=<value>]*

\examples
.GLOBAL\_PARAM T=\{27+100*time\}

\begin{Arguments}
\argument{name}
Name of the global parameter.

\argument{value}
Global parameter value.  An expression is used for the value when specified within curly braces (\{\}).

\end{Arguments}

\comments

You may use parameters defined by \texttt{.PARAM} in expressions used to
define global parameters, but you may {\em not} use global parameters in
\texttt{.PARAM} definitions.

Unlike \texttt{.PARAM} parameters, global parameters are evaluated at
the time they are needed. They may, therefore, be time dependent, and may
depend on other time dependent quantities in the circuit.  They may also
be frequency dependent.

For example to load an external data file with time voltage pairs of data on each 
line into a global parameter, use this syntax:

\texttt{.GLOBAL\_PARAM extdata = \{tablefile("filename")\}}

where \texttt{filename} would be the name of the file to load.  See \ref{ExpressionDocumentation} for 
further information.

Global parameters are accessible, and have the same value, throughout all
levels of the netlist hierarchy.  It is not legal to redefine global parameters
in different levels of the netlist hierarchy.

\end{Command}
