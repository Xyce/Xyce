% Sandia National Laboratories is a multimission laboratory managed and
% operated by National Technology & Engineering Solutions of Sandia, LLC, a
% wholly owned subsidiary of Honeywell International Inc., for the U.S.
% Department of Energy’s National Nuclear Security Administration under
% contract DE-NA0003525.

% Copyright 2002-2024 National Technology & Engineering Solutions of Sandia,
% LLC (NTESS).


Calculates the operating point\index{Operating Point} for the circuit for a
range of values.  Primarily, this capability is applied to independent
voltage sources, but it can also be applied to most device parameters.
Note that this may be repeated for multiple sources in the
same \texttt{.DC} line.

The .DC command can specify a linear sweep, decade logarithmic sweep,
octave logarithmic sweep, a list of values, or a data table of multivariate values.

\subsubsection{Linear Sweeps}
\index{analysis!DC!Linear sweeps} \index{DC analysis!Linear sweeps}

\begin{Command}
\format
\begin{alltt}
.DC [LIN] <sweep variable name> <start> <stop> <step>
+ [<sweep variable name> <start> <stop> <step>]*
\end{alltt}

\examples
\begin{alltt}
.DC LIN V1 5 25 5
.DC VIN -10 15 1
.DC R1 0 3.5 0.05 C1 0 3.5 0.5

.param start=5, stop=25, points=5
.DC \{start\} \{stop\} \{points\} 
\end{alltt}

\comments
A \texttt{.PRINT DC} must be used to get the results of the DC sweep
analysis.  See Section \ref{.PRINT}.
\index{\texttt{.PRINT}}\index{results!print}\index{\texttt{.PRINT}!\texttt{DC}}

A \texttt{.OP} comand will result in a linear DC analysis if there is no .DC specified.

If the stop value is smaller than the start value, the step value
should be negative.  If a positive step value is given in this case,
only a single point (at the start value) will be performed, and a
warning will be emitted.

\end{Command}

\subsubsection{Decade Sweeps}
\index{analysis!DC!Decade sweeps} \index{DC analysis!Decade sweeps}

\begin{Command}
\format
\begin{alltt}
.DC DEC <sweep variable name> <start> <stop> <points>
+ [DEC <sweep variable name> <start> <stop> <points>]*
\end{alltt}

\examples
\begin{alltt}
.DC DEC VIN 1 100 2 
.DC DEC R1 100 10000 3 DEC VGS 0.001 1.0 2

.param start=1, stop=100, points=2
.DC DEC VIN \{start\} \{stop\} \{points\}
\end{alltt}

\comments
The stop value should be larger than the start value.  If a stop value
smaller than the start value is given, only a single point at the
start value will be performed, and a warning will be emitted.  The
points value must be an integer.

\end{Command}

\subsubsection{Octave Sweeps}
\index{analysis!DC!Octave sweeps} \index{DC analysis!Octave sweeps}

\begin{Command}
\format
\begin{alltt}
.DC OCT <sweep variable name> <start> <stop> <points>
+ [OCT <sweep variable name><start> <stop> <points>]\ldots 
\end{alltt}

\examples
\begin{alltt}
.DC OCT VIN 0.125 64 2 
.DC OCT R1 0.015625 512 3 OCT C1 512 4096 1

.param start=0.125, stop=64, points=2
.DC OCT VIN \{start\} \{stop\} \{points\}
\end{alltt}

\comments
The stop value should be larger than the start value.  If a stop value
smaller than the start value is given, only a single point at the
start value will be performed, and a warning will be emitted.  The
points value must be an integer.

\end{Command}

\subsubsection{List Sweeps}
\index{analysis!DC!List Sweeps} \index{DC analysis!List sweeps}

\begin{Command}
\format
\begin{alltt}
.DC <sweep variable name> LIST <val> <val> <val>*
+ [ <sweep variable name> LIST <val> <val>* ]*
\end{alltt}

\examples
\begin{alltt}
.DC VIN LIST 1.0 2.0 5.0 6.0 10.0 
.DC VDS LIST 0 3.5 0.05 VGS LIST 0 3.5 0.5
.DC TEMP LIST 10.0 15.0 18.0 27.0 33.0

.param val1=0, val2=3.5, val3=0.5
.DC VDS LIST \{val1\} \{val2\} \{val3\} 
\end{alltt}

\end{Command}


\subsubsection{Data Sweeps}
\index{analysis!DC!Data Sweeps} \index{DC analysis!Data sweeps}

\begin{Command}
\format
\begin{alltt}
.DC DATA=<data table name> 
\end{alltt}

\examples
.DC data=resistorValues

.data resistorValues \\
+ r1   r2 \\
+ 8.0000e+00  4.0000e+00 \\
+ 9.0000e+00  4.0000e+00 \\
.enddata

\end{Command}
