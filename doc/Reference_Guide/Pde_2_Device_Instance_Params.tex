% This table was generated by Xyce:
%   Xyce -doc_cat Pde 2
%
\index{2d pde (level 2)!device instance parameters}
\begin{DeviceParamTableGenerated}{2D PDE (level 2) Device Instance Parameters}{Pde_2_Device_Instance_Params}
BULKMATERIAL & Material of bulk material. & -- & 'SI' \\ \hline
DISPLCUR & If true, displacement current is computed and output & logical (T/F) & false \\ \hline
DOPINGPROFILES &  & \multicolumn{2}{c}{See Table~\ref{DOPINGPROFILES_Composite_Params}}  \\ \hline
MAXVOLTDELTA & Maximum voltage change used by two-level Newton algorithm. & V & 0.025 \\ \hline
MESHFILE & This is a required field for a 2D simulation.  If the user specifies meshfile=internal.mesh, the model will create aCartesian mesh using the parameters L,W,NX and NY.  If the user specifies anything else (for example meshfile=diode.msh), the model will attempt to read in a mesh file of that name.  The format is assumed to be that of the SG Framework. & -- & 'internal.msh' \\ \hline
MOBMODEL & Mobility model. & -- & 'ARORA' \\ \hline
NODE &  & \multicolumn{2}{c}{See Table~\ref{NODE_2D_Composite_Params}}  \\ \hline
NX & Number of mesh points, x-direction. & -- & 11 \\ \hline
NY & Number of mesh points, y-direction. & -- & 11 \\ \hline
REGION &  & \multicolumn{2}{c}{See Table~\ref{REGION_Composite_Params}}  \\ \hline
TYPE & P-type or N-type - this is only relevant if using the default dopings & -- & 'PNP' \\ \hline
USEMATRIXGID &  & -- & false \\ \hline
USEOLDNI & Flag for using old (inaccurate) intrinsic carrier calculation. & logical (T/F) & false \\ \hline
USEVECTORGID &  & -- & false \\ \hline
VOLTLIM &  & logical (T/F) & false \\ \hline

\category{Doping Parameters}\\ \hline
GRADED & Flag for graded junction vs. abrupt junction. – (1/true=graded, 0/false=abrupt) & logical (T/F) & false \\ \hline
NA & Acceptor doping level & cm$^{-3}$ & 1e+15 \\ \hline
ND & Donor doping level & cm$^{-3}$ & 1e+15 \\ \hline
WJ & Junction width, if graded junction enabled. & cm & 0.0001 \\ \hline

\category{Geometry Parameters}\\ \hline
AREA & Cross sectional area of the device. & cm$^{-2}$ & 1 \\ \hline
CYL & Flag to enable cylindrical geometry & logical (T/F) & false \\ \hline
L & Device length & cm & 0.001 \\ \hline
W & Device width & cm & 0.001 \\ \hline

\category{Temperature Parameters}\\ \hline
TEMP & Device temperature & $^\circ$C & 27 \\ \hline

\category{Model Output Parameters}\\ \hline
GNUPLOTLEVEL & Flag for gnuplot output.
0 - no gnuplot files.
1 - gnuplot files.
gnuplot is an open source plotting program that is usually installed on Linux systems. gnuplot files will have the *Gnu.dat suffix, and the prefix will be thename of the device instance. & -- & 0 \\ \hline
INTERPGRIDSIZE &  & -- & 20 \\ \hline
OUTPUTINTERVAL & Time interval for tecplot output (if tecplot is enabled). & s & 0 \\ \hline
OUTPUTNLPOISSON & Flag to determine if the results of the nonlinear Poisson calculation is included in the output files.  Normally, this calculation is used to initialize a drift-diffusion calculation and isn't of interest. & logical (T/F) & false \\ \hline
%SGPLOTLEVEL & Flag for sgplot output.
%0 - no sgplot files.
%1 - sgplot files.
%sgplot is a plotting program that comes as part of the SG Framework. sgplot files will have the *.res suffix, and the prefix will be the name of the device instance & -- & 0 \\ \hline
TECPLOTLEVEL & Setting for Tecplot output:
0 - no Tecplot files
1 - Tecplot files, each output in a separate file. 2 - Tecplot file, each outputappended to a single file.
Tecplot files will have the .dat suffix, and the prefix will be the name of the device instance & -- & 1 \\ \hline
TXTDATALEVEL & Flag for volume-averaged text output.
0 - no text files.
1 - text files.
txtdataplot files will have the *.txt suffix, and the prefix will be the name of the device instance. & -- & 1 \\ \hline

\category{Scaling Parameters}\\ \hline
X0 & Length scalar; adjust to mitigate convergence problems. The model will do all of its scaling automatically, so it is generally not necessary to specify it manually. & cm & 0.0001 \\ \hline

\category{Boundary Condition Parameters}\\ \hline
CONSTBOUNDARY &  & -- & false \\ \hline
\end{DeviceParamTableGenerated}
