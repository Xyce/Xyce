% Sandia National Laboratories is a multimission laboratory managed and
% operated by National Technology & Engineering Solutions of Sandia, LLC, a
% wholly owned subsidiary of Honeywell International Inc., for the U.S.
% Department of Energy’s National Nuclear Security Administration under
% contract DE-NA0003525.

% Copyright 2002-2020 National Technology & Engineering Solutions of Sandia,
% LLC (NTESS).



\begin{Device}\label{G_DEVICE}

\symbol
{\includegraphics{vccsSymbol}}

\device
\begin{alltt}
G<name> <(+) node> <(-) node> <(+) controlling node>
+ <(-) controlling node> <transconductance>
G<name> <(+) <node> <(-) node> VALUE = \{ <expression> \}
G<name> <(+) <node> <(-) node> TABLE \{ <expression> \} =
+ < <input value>,<output value> >*
G<name> <(+) <node> <(-) node> POLY(<value>)
+ [<+ controlling node> <- controlling node>]*
+ [<polynomial coefficient>]*
\end{alltt}

\examples
\begin{alltt}
GBUFFER 1 2 10 11 5.0
GPSK 11 6 VALUE = \{5MA*SIN(6.28*10kHz*TIME+V(3))\}
GA2 2 0 TABLE \{V(5)\} = (0,0) (1,5) (10,5) (11,0)
\end{alltt}

\parameters

\begin{Parameters}

\param{\vbox{\hbox{(+) node\hfil}\hbox{(-) node}}}

Output nodes. Positive current flows from the \texttt{(+)} node through
the source to the \texttt{(-)} node.

\param{\vbox{\hbox{(+) controlling node\hfil}\hbox{(-) controlling node}}}

Node pairs that define a set of controlling voltages. A given node may
appear multiple times and the output and controlling nodes may be the
same.

\end{Parameters}

\comments

In the first form, the voltage drop between the controlling nodes is
multiplied by the transconductance to obtain the current-source
output of the \texttt{G} device. 

The second through fourth forms using the \texttt{VALUE}, \texttt{TABLE}, and
\texttt{POLY} keywords, respectively, are used in analog behavioral modeling.
They are provided primarily for netlist compatibility with other simulators.
These two forms are automatically converted within \Xyce{} to its principal ABM
device, the B nonlinear dependent source device. See the B-source section
(\ref{B_Source_Device}) and the \Xyce{} User's Guide for more guidance on
analog behavioral modeling.  For details concerning the use of the
\texttt{POLY} format, see section~\ref{PspicePoly}.

For HSPICE compatibility, \texttt{CUR} is an allowed synonym for
\texttt{VALUE} for the G-source.

The power supplied or dissipated by this source device is calculated 
with $I \cdot \Delta V$ where the voltage drop is calculated as $(V_+ - V_-)$ 
and positive current flows from $V_+$ to $V_-$.  Dissipated power has a
positive sign, while supplied power has a negative sign.

{\bf NOTE:} The expression given on the left hand side of the equals
sign in G source TABLE expressions may be enclosed in braces, but is
not required to be.  Further, if braces are present there must be
exactly one pair of braces and it must enclose the entire expression.
It is not legal to use additional pairs of braces as parentheses
inside these expressions.  So
\begin{alltt}
GA2 2 0 TABLE \{V(5)+5\} = (0,0) (1,5) (10,5) (11,0)
GA3 2 0 TABLE V(5) = (0,0) (1,5) (10,5) (11,0)
\end{alltt}
are legal, but
\begin{alltt}
GA2 2 0 TABLE \{V(5)+\{5\}\} = (0,0) (1,5) (10,5) (11,0)
\end{alltt}
is not.  This last will result in a parsing error.
\end{Device}
