% Sandia National Laboratories is a multimission laboratory managed and
% operated by National Technology & Engineering Solutions of Sandia, LLC, a
% wholly owned subsidiary of Honeywell International Inc., for the U.S.
% Department of Energy’s National Nuclear Security Administration under
% contract DE-NA0003525.

% Copyright 2002-2024 National Technology & Engineering Solutions of Sandia,
% LLC (NTESS).


%%
%% Device Selections Table
%%

\begin{OptionTable}{Options for Device Package}
\label{DevicePKG}%
DEFAD & MOS Drain Diffusion Area & 0.0 \\ \hline
DEFAS & MOS Source Diffusion Area & 0.0 \\ \hline
DEFL & MOS Default Channel Length & 1.0E-4 \\ \hline
DEFW & MOS Default Channel Width & 1.0E-4 \\ \hline
DIGINITSTATE & This option controls the behavior of the Digital Latch (DLTCH), D Flip-Flop (DFF),
JK Flip-Flop (JKFF) and T Flip-Flop (TFF) behavioral digital devices during the DC Operating Point (DCOP)
calculations. See ~\ref{U_DEVICE} for more details.  & 3 \\ \hline 
GMIN & Minimum Conductance & 1.0E-12 \\ \hline
MINRES & This is a minimum resistance to be used in place of the default zero value of semiconductor device internal resistances.  It is only used when model specifications (\texttt{.MODEL} cards) leave the parameter at its default value of zero, and is not used if the model explicitly sets the resistance to zero.   & 0.0 \\ \hline
MINCAP & This is a minimum capacitance to be used in place of the default zero value of semiconductor device internal capacitances.  It is only used when model specifications (\texttt{.MODEL} cards) leave the parameter at its default value of zero, and is not used if the model explicitly sets the capacitance to zero.   & 0.0 \\ \hline
TEMP & Temperature & 27.0 $^\circ$C (300.15K) \\ \hline
TNOM & Nominal Temperature & 27.0 $^\circ$C (300.15K) \\ \hline
% TVR: I can't find any use of this parameter in Xyce!
%\debug{\texttt{SCALESRC}} & \debug{Scaling factor for source scaling} & \debug{0.0} \\ \hline
\debug{NUMJAC} & \debug{Numerical Jacobian flag (only use for small problems)}
& \debug{0 (FALSE)} \\ \hline
VOLTLIM & Voltage limiting & 1 (TRUE) \\ \hline
B3SOIVOLTLIM & BSIMSOI3 Voltage limiting.  This flag is similar to VOLTLIM, except that it only 
  applies to the BSIMSOI version 3  (the newer versions of the BSIM SOI do not have voltage limiting).  
  Turning this off will often improve numerical robustness.  Unlike VOLTLIM, turning this off 
  does not disable the initial condition code in the BSIMSOI model.  & 1 (TRUE) \\ \hline
B3SOIGMINSCALING & This flag enables scaling of gmin in the BSIMSOI version 3 device model. When
it is enabled, a scaling factor of 1e-6 is applied to the gmin value. & 1 (TRUE) \\ \hline

\debug{icFac} 
& \debug{This is a multiplicative factor which is applied to right-hand side vector loads of 
.IC initial conditions during the DCOP phase. }
& \debug{10000.0} \\ \hline

MAXTIMESTEP & Maximum time step size & 1.0E+99 \\ \hline

SMOOTHBSRC & This flag enables smooth transitions by adding a RC network to the output of ABM devices    &    0  \\ \hline


RCCONST & This option controls the smoothness of the transitions if the
SMOOTHBSRC flag is enabled. This is done by specifying the RC constant of the
RC network & 1e-9 \\ \hline

\\ \hline
\multicolumn{3}{|c|}{\color{XyceDarkBlue}\em\bfseries MOSFET Homtopy parameters} \\ \hline

VDSSCALEMIN & Scaling factor for Vds    & 0.3      \\ \hline
VGSTCONST   & Initial value for Vgst    & 4.5 Volt \\ \hline
LENGTH0     & Initial value for length  & 5.0e-6   \\ \hline
WIDTH0      & Initial value for width   & 200.0e-6 \\ \hline
TOX0        & Initial value for oxide thickness & 6.0e-8  \\ \hline

\\ \hline
\multicolumn{3}{|c|}{\color{XyceDarkBlue}\em\bfseries Debug output parameters} \\ \hline

\debug{DEBUGLEVEL} & \debug{The higher this number, the more info is output} & \debug{1} \\ \hline
\debug{DEBUGMINTIMESTEP} & \debug{First time-step debug information is output} & \debug{0} \\ \hline
\debug{DEBUGMAXTIMESTEP} & \debug{Last time-step of debug output} & \debug{65536} \\ \hline
\debug{DEBUGMINTIME} & \debug{Same as \texttt{DEBUGMINTIMESTEP} except controlled by time (sec.) instead of step number} & \debug{0.0} \\ \hline
\debug{DEBUGMAXTIME} & \debug{Same as \texttt{DEBUGMAXTIMESTEP} except controlled by time (sec.) instead of step number} & \debug{100.0} \\ \hline
\end{OptionTable}
