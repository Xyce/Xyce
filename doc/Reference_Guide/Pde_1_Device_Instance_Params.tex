% This table was generated by Xyce:
%   Xyce -doc_cat Pde 1
%
\index{1d pde (level 1)!device instance parameters}
\begin{DeviceParamTableGenerated}{1D PDE (level 1) Device Instance Parameters}{Pde_1_Device_Instance_Params}
AUGER & Flag to turn on/off Auger recombination & logical (T/F) & true \\ \hline
BULKMATERIAL & Bulk semiconductor material & -- & 'SI' \\ \hline
DOPINGPROFILES &  & \multicolumn{2}{c}{See Table~\ref{DOPINGPROFILES_Composite_Params}}  \\ \hline
FERMIDIRAC & Use Fermi-Dirac statistics. & logical (T/F) & false \\ \hline
FIELDDEP & If true, use field dependent mobility. & logical (T/F) & false \\ \hline
LAYER &  & \multicolumn{2}{c}{See Table~\ref{LAYER_Composite_Params}}  \\ \hline
MASKVARSTIA & If set to true, then some variables are excluded from the time integration error control calculation. & logical (T/F) & false \\ \hline
MAXVOLTDELTA & Maximum voltage change used by two-level Newton algorithm. & V & 0.025 \\ \hline
MESHFILE &  & -- & 'internal.msh' \\ \hline
MOBMODEL & Mobility model. & -- & 'ARORA' \\ \hline
NODE &  & \multicolumn{2}{c}{See Table~\ref{NODE_Composite_Params}}  \\ \hline
NX & Number of mesh points & -- & 11 \\ \hline
REGION &  & \multicolumn{2}{c}{See Table~\ref{REGION_Composite_Params}}  \\ \hline
SRH & Flag to turn on/off Shockley-Read-Hall (SRH) recombination. & logical (T/F) & true \\ \hline
THERMIONICEMISSION &  & logical (T/F) & false \\ \hline
TUNNELING &  & -- & 'none' \\ \hline
USEOLDNI & Flag for using old(inaccurate) intrinsic carrier calculation. & logical (T/F) & false \\ \hline
VOLTLIM & Flag to apply voltage limiting.  This is only relevant for an experimental two-level Newton solver. & logical (T/F) & false \\ \hline

\category{Doping Parameters}\\ \hline
DOPING\_FILE & File containing doping profile. & -- & 'NOFILE' \\ \hline
GRADED & Flag for graded junction vs. abrupt junction. – (1/true=graded, 0/false=abrupt) & logical (T/F) & false \\ \hline
NA & Acceptor doping level & cm$^{-3}$ & 1e+15 \\ \hline
ND & Donor doping level & cm$^{-3}$ & 1e+15 \\ \hline
NDOPE\_FILE & File containing doping profile for N-type dopants. & -- & 'NOFILE' \\ \hline
PDOPE\_FILE & File containing doping profile for P-type dopants. & -- & 'NOFILE' \\ \hline
WJ & Junction width, if graded junction enabled. & cm & 0.0001 \\ \hline

\category{Geometry Parameters}\\ \hline
ANODE.AREA & Anode area (used for two-terminal devices) & cm$^{-2}$ & 0 \\ \hline
AREA & Cross sectional area of the device. & cm$^{-2}$ & 1 \\ \hline
BASE.AREA & Base area (used for three-terminal (BJT) devices) & cm$^{-2}$ & 0 \\ \hline
BASE.LOC & Location of base contact (necessary if running with three terminals). & cm & 0.0005 \\ \hline
CATHODE.AREA & Cathode area (used for two-terminal devices) & cm$^{-2}$ & 0 \\ \hline
COLLECTOR.AREA & Collector area (used for three-terminal (BJT) devices) & cm$^{-2}$ & 0 \\ \hline
EMITTER.AREA & Emitter area (used for three-terminal (BJT) devices) & cm$^{-2}$ & 0 \\ \hline
L & Device width. (Synonym with W parameter) & cm & 0.001 \\ \hline
W & Device width. (Synonym with L parameter) & cm & 0.001 \\ \hline

\category{Temperature Parameters}\\ \hline
TEMP & Device temperature & $^\circ$C & 27 \\ \hline

\category{Model Output Parameters}\\ \hline
FIRSTELECTRODEOFFSET & This is an output parameter.  It is only used if OFFSETOUTPUTVOLTAGE=true. (see description of that paramaeter & logical (T/F) & false \\ \hline
GNUPLOTLEVEL & Flag for gnuplot output.
0 - no gnuplot files.
1 - gnuplot files.
gnuplot is an open source plotting program that is usually installed on Linux systems. gnuplot files will have the *Gnu.dat suffix, and the prefix will be thename of the device instance. & -- & 1 \\ \hline
OFFSETOUTPUTVOLTAGE & This is an output parameter that determines the ``zero'' of the potential at output.  If OFFSETOUTPUTVOLTAGE=true (default) it will adjust the voltages at output so that the minimum voltage is zero. If true and also FIRSTELECTRODEOFFSET=true, then the voltage of the first electrode is the zero point.  If OFFSETOUTPUTVOLTAGE=false, the output voltage sets the intrisic Fermi level to zero.  Depending on circumstances each of these may be more or less convenient for plotting. & logical (T/F) & true \\ \hline
OUTPUTINTERVAL & Time interval for tecplot output (if tecplot is enabled). & s & 0 \\ \hline
OUTPUTNLPOISSON & Flag to determine if the results of the nonlinear Poisson calculation is included in the output files.  Normally, this calculation is used to initialize a drift-diffusion calculation and isn't of interest. & logical (T/F) & false \\ \hline
%SGPLOTLEVEL & Flag for sgplot output.
%0 - no sgplot files.
%1 - sgplot files.
%sgplot is a plotting program that comes as part of the SG Framework. sgplot files will have the *.res suffix, and the prefix will be the name of the device instance & -- & 0 \\ \hline
TECPLOTLEVEL & Setting for Tecplot output:
0 - no Tecplot files
1 - Tecplot files, each output in a separate file. 2 - Tecplot file, each outputappended to a single file.
Tecplot files will have the .dat suffix, and the prefix will be the name of the device instance & -- & 1 \\ \hline

\category{Scaling Parameters}\\ \hline
C0 & Density scalar; adjust to mitigate convergence problems.The model will do all of its scaling automatically, so it is generally not necessary to specify it manually. & cm$^{-3}$ & 1e+15 \\ \hline
DENSITYSCALARFRACTION & Fraction of the maximum doping by which density will be scaled.The model will do all of its scaling automatically, so it is generally not necessary to specify it manually. & logical (T/F) & 0.1 \\ \hline
SCALEDENSITYTOMAXDOPING & If set the density will be scaled by a fraction of the maximum doping.The model will do all of its scaling automatically, so it is generally not necessary to specify it manually. & logical (T/F) & true \\ \hline
t0 & Time scalar; adjust to mitigate convergence problems.The model will do all of its scaling automatically, so it is generally not necessary to specify it manually. & s & 1e-06 \\ \hline
X0 & Length scalar; adjust to mitigate convergence problems.The model will do all of its scaling automatically, so it is generally not necessary to specify it manually. & cm & 1e-07 \\ \hline

\category{Boundary Condition Parameters}\\ \hline
ANODE.BC & Anode voltage boundary condition.  Only used if device is uncoupled from circuit, and running in diode mode.
 & V & 0.5 \\ \hline
BASE.BC & Base voltage boundary condition.  Only used if device is uncoupled from circuit, and running in BJT mode.
 & V & 0 \\ \hline
CATHODE.BC & Cathode voltage boundary condition.  Only used if device is uncoupled from circuit, and running in diode mode.
 & V & 0 \\ \hline
COLLECTOR.BC & Collector voltage boundary condition.  Only used if device is uncoupled from circuit, and running in BJT mode.
 & V & 0 \\ \hline
EMITTER.BC & Emitter voltage boundary condition.  Only used if device is uncoupled from circuit, and running in BJT mode.
 & V & 0.5 \\ \hline
\end{DeviceParamTableGenerated}
