\Xyce{} makes use of code developed by various third parties.  The following
text is provided to comply with the licenses of the codes that require it.

% Acknowledgement of original SPICE code use
The ksparse solver in \Xyce{} contains code derived from SPICE 3f5 source code:
\verbatiminput{SPICE_COPYRIGHT.txt}

% Acknowledgement of AMD code use
\Xyce{}'s linear solver makes use of the AMD library:
\begin{verbatim}
AMD, Copyright (c), 1996-2015, Timothy A. Davis,
Patrick R. Amestoy, and Iain S. Duff.  All Rights Reserved.
Used in Xyce under the LGPL v2.1 license.
\end{verbatim}

% Acknowledgement of Open MPI code use
Parallel builds of \Xyce{} use the Open MPI library:
\verbatiminput{OpenMPI_LICENSE.txt}

% Acknowledgement of Trilinos code use
\Xyce{} uses the Trilinos Solver Framework:
\verbatiminput{Trilinos_COPYRIGHT.txt}

Some versions of \Xyce{} use the Intel Math Kernel Library:
\verbatiminput{Intel_MKL_LICENSE.txt}

% Acknowledgement of original Diode-CMC code use
\Xyce{}'s implementation of the Diode-CMC model, version 2.0.0, is derived from
Verilog-A sources provided under the following license:
\verbatiminput{DIODE-CMC_COPYRIGHT.txt}

% Acknowledgement of original L-UTSOI code use
\Xyce{}'s implementations of the following models are derived from Verilog-A
sources provided under the ECL-2.0 license:
\begin{itemize}[noitemsep]
     \item The L-UTSOI is Copyright 2020 CEA-Leti.
\end{itemize}
The ECL-2.0 license text is as follows:
\begin{verbatim}
Licensed under Educational Community License, Version 2.0 (the
"License"); you may not use this file except in compliance with the
License. You may obtain a copy of the license at

http://opensource.org/licenses/ECL-2.0

Unless required by applicable law or agreed to in writing, software
distributed under the License is distributed on an "AS IS" BASIS,
WITHOUT WARRANTIES OR CONDITIONS OF ANY KIND, either express or
implied. See the License for the specific language governing
permissions and limitations under the License.
\end{verbatim}

% Acknowledgement of original MEXTRAM code use
\Xyce{}'s implementation of the MEXTRAM model, version 504.12.1, is derived
from Verilog-A sources provided under the following license:
\verbatiminput{MEXTRAM_COPYRIGHT.txt}

% Acknowledgement of original HICUM/L0 code use
\Xyce{}'s implementation of the HICUM/L0 model, version 1.32, is derived from
Verilog-A sources provided under the following license:
\verbatiminput{HICUM_L0_COPYRIGHT.txt}

% Acknowledgement of original HICUM/L2 code use
\Xyce{}'s implementation of the HICUM/L2 model, version 2.34, is derived from
Verilog-A sources provided under the following license:
\verbatiminput{HICUM_L2_COPYRIGHT.txt}

% Acknowledgement of the BSIM3, BSIM4 and BSIM-SOI v. 3.2.
% All three models are coded directly in C++.
% The specific models are the BSIM3 v.3.2.2, the BSIM4 v. 4.6.1,
% and the BSIM-SOI v. 3.2. 
\Xyce{}'s implementations of the BSIM3 v.3.2.2, the BSIM4 v. 4.6.1,
the BSIM4 v. 4.7.0, and the BSIM-SOI v. 3.2, are based on the original code of
those devices provided by University of California, Berkeley.  They all have
the following license:
\verbatiminput{BSIM_OPEN_COPYRIGHTS.txt}

Version 4.8.2 of the BSIM4 is licensed under the Educational Community License, Version 2.0:
\verbatiminput{BSIM_ECL_2.0.txt}

% Acknowledgement of original BSIM6 code use
\Xyce{}'s implementation of the BSIM6 model, version 6.1.1, is derived from
Verilog-A sources provided under the following license:
\verbatiminput{BSIM6_COPYRIGHT.txt}

% Acknowledgement of original BSIM-SOI code use
\Xyce{}'s implementations of the BSIM-SOI model, versions 4.5 and 4.6.1, are
derived from Verilog-A sources provided under the following license:
\verbatiminput{BSIM-SOI_COPYRIGHT.txt}

% Acknowledgement of original BSIM-CMG code use
% This is for the BSIM-CMG v. 
\Xyce{}'s implementations of the BSIM-CMG model, versions 107.0.0 and 110.0.0,
are derived from Verilog-A sources provided under the following license:
\verbatiminput{BSIM-CMG_COPYRIGHT.txt}

% Acknowledgement of original MVS code use
\Xyce{}'s implementation of the MVS model, version 2.0.0, is derived from
Verilog-A sources provided under the following license:
\verbatiminput{MVS_COPYRIGHT.txt}

% Acknowledgement of original PSP 102 code use
\Xyce{}'s implementation of the PSP model, version 102.5.0, is derived from
Verilog-A sources provided under the following license:
\verbatiminput{PSP_102_COPYRIGHT.txt}

% Acknowledgement of original PSP 103 and the JUNCAP diode code use
\Xyce{}'s implementations of the PSP model, version 103.4.0, and the JUNCAP
diode, are derived from Verilog-A sources provided under the following license:
\verbatiminput{PSP_103_COPYRIGHT.txt}

% Acknowledgement of original EKV3 301.02 use
\Xyce{}'s implementations of the EKV 2.6 and EKV 3.0 models are derived from
Verilog-A sources developed by the EKV Team of the Electronics Laboratory-TUC
(Technical University of Crete).  They are included in \Xyce{} under license
from Technical University of Crete.  The official web site of the EKV model is
\url{http://ekv.epfl.ch/}.

\textbf{Due to licensing restrictions, the EKV MOSFETs are not available in
     open-source versions of \Xyce{}.  The license for EKV3 authorizes Sandia
     National Laboratories to distribute EKV3 only in binary versions of the code.}

