% Sandia National Laboratories is a multimission laboratory managed and
% operated by National Technology & Engineering Solutions of Sandia, LLC, a
% wholly owned subsidiary of Honeywell International Inc., for the U.S.
% Department of Energy’s National Nuclear Security Administration under
% contract DE-NA0003525.

% Copyright 2002-2021 National Technology & Engineering Solutions of Sandia,
% LLC (NTESS).

% -------------------------------------------------------------------------
% Sampling Analysis Section ---------------------------------------------------
% -------------------------------------------------------------------------

\clearpage
\section{Polynomial Chaos Expansion methods}
\label{PCE_Analysis}
\label{pce_Overview}
\index{analysis!PCE} \index{PCE analysis} 
\index{\texttt{.SAMPLING}}

\Xyce{} supports several styles of Polynomial Chaos expansion (PCE) methods, which are 
stochastic expansion methods that approximate the functional dependence
of a simulation response on uncertain model parameters by expansion
in a polynomial basis~\cite{XiuKarn02,Xiu2010}.   
In \Xyce{} the Stokhos library~\cite{stokhos} has been used to implement 
generalized PCE methods using the Wiener-Askey scheme\cite{XiuKarn02}.   

To propagate input uncertainty through a model using PCE, \Xyce{}
performs the following steps: (1) input uncertainties are transformed
to a set of uncorrelated random variables, (2) a basis such
as Hermite polynomials is selected, and (3) the parameters of the
functional approximation are determined.  The general PCE 
for a response $O$ has the form:
\begin{equation}
  O(p) \approx \hat{O}(p) = \sum_{i=0}^{P} \alpha^i \varPsi^i(p),
  \label{eq:genPCEresponse}
\end{equation}
where each multivariate basis polynomial $\varPsi^i(p)$ involves products of univariate
polynomials that are tailored to the individual random variables. 
More details about these methods and their \Xyce{} implementation can be found in
reference~\cite{xyceAdvancedUQ} and the references therein.

\Xyce{} supports three versions of polynomial chaos.  They mainly differ in how the coefficients $\alpha$ in equation~\ref{eq:genPCEresponse} are computed.  These three methods are:
\begin{itemize}
  \item Non-intrusive regression polynomial chaos
  \item Non-intrusive spectral projection  (NISP)
  \item Fully intrusive spectral projection
\end{itemize}
The first two versions are non-intrusive, meaning that they don't directly modify the forward problem being solved.  
Instead, they are augmentations to the \texttt{.SAMPLING} and \texttt{.EMBEDDEDSAMPLING} analysis methods.
The fully intrusive method is a separate analysis type, and is invoked using the \texttt{.PCE} command.

\subsection{Regression-based Polynomial Chaos}
\label{regressionPCE}

Regression-based PCE is briefly described here.
Regression-based PCE approaches aim to solve the linear system:
\begin{equation}
\boldsymbol{\varPsi} \boldsymbol{\alpha} \approx \boldsymbol{R} \label{eq:regression}
\end{equation}
for a set of PCE coefficients $\boldsymbol{\alpha}$ that best
reproduce a set of response values $\boldsymbol{R}$.  
The regression approach finds a set of PCE coefficients $\alpha^i$ which best fit a set of response
values obtained from a sampling study on the density function of the uncertain 
parameters~\cite{pt_colloc1}.  One advantage of regression-based methods is that
they are very flexible with respect to the exact sample points used to evaluate the responses, 
and can readily be used with standard sampling methods such as Monte Carlo and Latin Hypercube Sampling.
For this project, the method for solving 
Eq. \eqref{eq:regression} that has been implemented in \Xyce{} is 
least squares regression.

Regression-based PCE can be applied as a post-process step 
to \texttt{.SAMPLING} and \texttt{.EMBEDDEDSAMPLING}.  It is invoked by adding 
\texttt{regression\_pce=true} to the \texttt{.options SAMPLES} 
or \texttt{.options EMBEDDEDSAMPLES} command lines.

\subsection{Non-Intrusive Spectral Projection (NISP)}
\label{nisp}

The spectral projection PCE approach projects the response
against each basis function $\Psi_j(p)$ using inner
products and employs the polynomial orthogonality properties to
extract each coefficient. Each inner product involves a
multidimensional integral over the support range of the weighting
function, which can be evaluated numerically using sampling,
tensor-product quadrature, Smolyak sparse grid \cite{Smolyak_63}, or
cubature \cite{stroud} approaches.   For the \Xyce{} implementation 
this means evaluating the following equation for each coefficient using 
quadrature rules:
\begin{equation}
\alpha^{i} = \frac{\langle O\varPsi ^{i}\rangle} {\langle (\varPsi ^{i})^{2}\rangle} = \frac{1} {\langle (\varPsi^{i})^{2}\rangle}\int _{\varGamma}O(p ;y)\varPsi ^{i}(y)\rho (y)dy = 0,\quad i = 0,\ldots,P.
  \label{nispProjection}
\end{equation}
In this equation $\rho$ is the joint density function, with respect to which the polynomials $\varPsi^i$ are orthogonal.
Stokhos supports several quadrature methods, and supplies the necessary quadrature points 
to \Xyce{} for the specified details.  Similar to the regression methods 
described in section~\ref{regressionPCE}, in which \Xyce{} must provide function 
evaluations a specified sample points, for this method \Xyce{} provides 
function evaluations at each specified quadrature point, and then Eq. \eqref{nispProjection}
is evaluated for each requested output response.

Regression-based PCE can be applied as a post-process step 
to \texttt{.SAMPLING} and \texttt{.EMBEDDEDSAMPLING}.  It is invoked by adding 
\texttt{projection\_pce=true} to the \texttt{.options SAMPLES} 
or \texttt{.options EMBEDDEDSAMPLES} command lines.  

Regression PCE will use the samples selected by the sampling algorithm (Monte Carlo or Latin Hypercube).  
The regression PCE algorithm will estimate the optimal number of samples to use, and so this must 
be set by the user.   For regression PCE, the number of samples needed for an accurate answer
should be smaller than the number that would have been required by conventional sampling.
For regression problems, to get an accurate answer it is considered good practice to oversample 
the basis size by at least a factor of 2. 

A simple example netlist, which uses regression PCE 
with \texttt{.SAMPLING} is given in figure~\ref{regressionPCE_Netlist1}.


\begin{figure}[htbp]
\begin{centering}
\shadowbox{
\begin{minipage}{0.8\textwidth}
\begin{vquote}
\color{blue}Regression PCE example \color{black}
.param testNorm=\{aunif(2k,1k)\}
.param R1value=\{testNorm*2.0\}
R2 1 0 6K
R1 1 2 \{R1value\}
v1 2 0 1000V

.dc v1 1000 1000 1
.print dc format=tecplot v(1)
*
.SAMPLING useExpr=true

.options SAMPLES numsamples=10
+ regression\_pce=true
+ order=5
+ outputs=\{v(1)\}
+ sample\_type=lhs
+ stdoutput=true
+ resample=true
.end
\end{vquote}
\end{minipage}
}
\caption{Voltage divider regression PCE netlist.
\label{regressionPCE_Netlist1}}
\end{centering}
\end{figure}

\begin{figure}[htbp]
\fontsize{9pt}{10pt}
\begin{centering}
\shadowbox{
\begin{minipage}{0.99\textwidth}
\begin{vquote}
***** Number of sample points = 10

Seeding random number generator with 3355129392

LHS sampling mean of {V(1)} = 594.262
LHS sampling stddev of {V(1)} = 81.8265
LHS sampling variance of {V(1)} = 6695.58
LHS sampling skew of {V(1)} = 0.548949
LHS sampling kurtosis of {V(1)} = 1.96246
LHS sampling max of {V(1)} = 749.366
LHS sampling min of {V(1)} = 509.557

(traditional sampling) regression PCE mean of {V(1)} = 608.198
(traditional sampling) regression PCE stddev of {V(1)} = 71.3839

Statistics from re-sampling regression PCE approximation of {V(1)}:
mean     = 607.045
stddev   = 72.5006
variance = 5256.34
skew     = 0.256431
kurtosis = 1.8671
max      = 749.815
min      = 499.999
\end{vquote}
\end{minipage}
}
\caption{Voltage divider regression PCE result.
\label{regressionPCE_Result1}}
\end{centering}
\end{figure}

\subsection{Fully Intrusive Spectral Projection}
\label{intrusivePCE}
The fully intrusive form of PCE involves solving a system of equations which is constructed by 
applying the orthogonal polynomial expansion defined by 
Eq. \eqref{eq:genPCEresponse} to all the terms of the differential-algebraic 
system of equations solved by \Xyce{}.  The details of this construction are given in 
reference~\cite{xyceAdvancedUQ}.


\clearpage
