\subsection{Notes about .SENS accuracy and formulation}
The sensitivity calculation in \Xyce{} is based on a chain rule
calculation.  Ultimately, the calculation will produce $dO/dp$, where
$O$ is a user-specified objective function, and $p$ is a
user-specified parameter.  $dO/dp$ is equal to:
\begin{equation}
  \frac{dO}{dp} = \frac{\partial O}{\partial x}\frac{\partial x}{\partial p} + \frac{\partial O}{\partial p}
  \label{objectiveDerivative}
\end{equation}
\noindent where $x$ is a solution vector and $\partial x/\partial p$
is the sensitivity of that solution vector with respect to the
parameter.  
\subsubsection{DC and Transient}
Evaluating~\ref{objectiveDerivative} requires that
$\partial x/\partial p$ be computed first.  The direct sensivity
calculation for $\partial x/\partial p$ can be derived by considering
the DAE form of the residual equation:
\begin{equation}
  F(x,t) = \dot{q}(x,t) + j(x,t) - b(t) = 0
  \label{residual}
\end{equation}
\noindent In this equation, $q$ represents quantities that must be
differentiated with respect to time (such as capacitor charge), and
$j$ represents algebraic terms that depend on the solution $x$ (such
as DC currents), and $b$ are independent sources that only depend on
time.  Equation~\ref{residual} is solved to obtain the solution, $x$.
To obtain sensitivities, one must differentiate this equation with
respect to a parameter, $p$.
\begin{equation}
  \frac{dF}{dp} = \frac{d}{dp}\left( \dot{q} + j - b \right) = 0
  \label{dfdp}
\end{equation}
\paragraph{DC Sensitivities} In steady-state, equation~\ref{dfdp} simplifies to:
\begin{equation}
  \frac{dF}{dp} = \frac{dj}{dp} - \frac{db}{dp} = 0
  \label{steady_dfdp}
\end{equation}
\noindent As $j$ is dependent upon $x$, the $dj/dp$ term  
be expanded via chain rule as $\frac{dj}{dp} = \frac{\partial j}{\partial x}\frac{\partial x}{\partial p} + \frac{\partial j}{\partial p}$, where $\frac{\partial j}{\partial x}$ is the DC Jacobian matrix,
and the resulting equation must be re-arranged to set up a matrix equation that can be solved to obtain
$\frac{\partial x}{\partial p}$:
\begin{equation}
  \frac{\partial j}{\partial x}\frac{\partial x}{\partial p} = - \frac{\partial j}{\partial p} + \frac{db}{dp}.
  \label{finalSens}
\end{equation}
\noindent In equation~\ref{finalSens}, the terms on the right hand
side are computed by the individual device models once the Newton loop
the circuit analysis has converged.  The Jacobian matrix on the left
hand side ($\partial j/\partial x$) is the same matrix used by the original analysis.
Once the linear system is solved, then $\partial x/\partial p$ is available for the
given parameter, and it can then be applied to
equation~\ref{objectiveDerivative}.  

\paragraph{Transient Sensitivities} For transient, a similar linear
system is solved, which depends on the specific time integration
method used.  For Backward-Euler the linear system is:
\begin{equation}
  \left[ \frac{1}{h} \frac{\partial q}{\partial x} 
  + \frac{\partial j}{\partial x} \right] \frac{\partial x}{\partial p}_{n+1} 
 =
  -\frac{1}{h} \left[ \frac{\partial q}{\partial p}_{n+1} - \frac{\partial q}{\partial p}_n \right] 
 - \frac{\partial j}{\partial p} 
 + \frac{db}{dp} 
 + \frac{1}{h} \left[ \frac{\partial q}{\partial x} \right] \frac{\partial x}{\partial p}_n 
 \label{finalTransientSens}
\end{equation}
\noindent Where $h$ is the time step size.  As with ~\ref{finalSens},
the left hand side of the equation contains the original Jacobian
matrix.

\paragraph{Analytical vs Numerical derivatives}
The accuracy of the above calculation is dependent on the
accuracy of the individual derivatives that comprise
equations~\ref{objectiveDerivative} and~\ref{finalSens}
or~\ref{finalTransientSens}.  In \Xyce{}, most of the derivatives are computed analytically.
For example, the Jacobian derivatives
($\partial j/\partial x$ and $\partial q/\partial x$) are always analytic.  Similarly, the objective
function derivatives ($\partial O/\partial x$) are also always analytic.  

However, the derivatives on the right hand side of~\ref{finalTransientSens}
($\partial j/\partial p$, $\partial b/\partial p$ and $\partial q/\partial p$) depend on particular device
implementations.  If the device was implemented with analytic
parameter sensitivies, then those sensitivities are used.  If analytic
derivatives were not available, then the $\partial j/\partial p$, $\partial q/\partial p$ and $\partial b/\partial p$
derivatives are computed using finite differences.  A list of \Xyce{}
devices, and which of them support analytic sensitivities, is given in
the \Xyce{} Reference Guide\ReferenceGuide{}.

For some problems, finite difference derivatives will work fine, but
some devices and/or circuit problems have wide ranges of solution
and/or parameter scalings, and this can render inaccurate finite
difference derivatives.  As this capability develops, most devices
should eventually provide analytic derivatives.

The above derivation and arguments were given for direct sensitivities, but
the same ideas apply for adjoint sensitivites.

For transient, note that the transient direct calculation uses the
same time steps that are used for the original circuit analysis.  It
does not impose any additional error control that is specific to the
accuracy of $\partial x/\partial p$.

\subsubsection{AC}

The math for AC sensitivities is a bit different than DC and transient.


