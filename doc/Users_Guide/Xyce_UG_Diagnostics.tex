% Sandia National Laboratories is a multimission laboratory managed and
% operated by National Technology & Engineering Solutions of Sandia, LLC, a
% wholly owned subsidiary of Honeywell International Inc., for the U.S.
% Department of Energy’s National Nuclear Security Administration under
% contract DE-NA0003525.

% Copyright 2002-2023 National Technology & Engineering Solutions of Sandia,
% LLC (NTESS).

\chapter{Circuit Diagnostics and Troubleshooting}
\index{\Xyce{}!troubleshooting}
\index{\Xyce{}!diagnostics}

\chapteroverview{Chapter Overview}
{
This chapter provides guidance on using \Xyce{}'s diagnostic output 
to understand simulation performance and aid in correcting
circuit problems.
\begin{XyceItemize}
\item Section~\ref{DiagnosticOutput}, {\em Diagnostic Output}
%\item Section~\ref{SimulationIssues}, {\em Simulation Issues}
\end{XyceItemize}
}

\section{Diagnostic Output}
\label{DiagnosticOutput}

\subsection{Introduction}
To aid in understanding circuit simulation performance, \Xyce{} has a diagnostic output mode
where additional data can be sent to an external file for the user to study.  The data 
can include times and values of currents and voltages that are either extreme or over some
user defined limit.  Additionally, there is a capability to output solution nodes that 
appear to be discontinuous within a user specified limit.  The data is sent to a separate
file to avoid it being lost in the normal simulation output.  The following sections will
present examples of using diagnostic output in a circuit.


\subsection{Enabling Diagnostic Output}
To enable diagnostic output in a simulation add the line
\begin{NetlistFigure}{Basic diagnostic option line}{diagOptionExample}
.options diagnostic <diagnostic mode> 
+ diagfilename=<output file name>
\end{NetlistFigure}
where \texttt{<diagnostic mode>} can be any of the following modes
\begin{itemize}
  \item \texttt{EXTREMALIMIT=<value>}  Output the single most extreme solution value 
    whose absolute value is greater than \texttt{value}.
  \item \texttt{CURRENTLIMIT=<value>}  Output all of the branch currents and lead currents  
    whose absolute value is greater than \texttt{value}.
  \item \texttt{VOLTAGELIMIT=<value>}  Output all of the solution voltages  
    whose absolute value is greater than \texttt{value}
  \item \texttt{DISCLIMIT=<value>}  Output all solution variables where the absolute value
    of the difference between the predictor and corrector is greater than \texttt{value}
\end{itemize}
Multiple modes can be specified and the results will be grouped by diagnostic mode for 
each step within the diagnostic output file.

Finally, the \texttt{.options diagnostic diagfilename} is optional.  If it is not given then 
the diagnostic output will be saved in a file called \texttt{XyceDiag.out}.  No matter whether 
it is set, or left at its default value, Xyce will overwrite any existing file with new 
output each time Xyce is run.  

The next sections will give examples for each of the diagnostic modes listed above.

\subsection{Extrema Output}
To obtain the solution node name and value that has the largest absolute value over some 
limit, use the \texttt{EXTREMALIMIT} option. 

In this simple \Xyce{} input netlist, \texttt{EXTREMALIMIT} is used to output values over 6.0 Volts

\begin{NetlistFigure}{Sample use of the diagnostic EXTREMALIMIT}{extremaExample}
* 4049 Circuit

X2 node3 node8 node2 0 CD4049UB
V1 node2 0 5
C1 node8 node5 1.088N
R1 node3 node5 46.5K
R2 node1 node5 98.6K
X1 node1 node3 node2 0 CD4049UB

.tran .05U 500U NOOP

.options timeint method=gear abstol=1.0e-7 
+ reltol=1.0e-6

.options diagnostic EXTREMALIMIT=6.0 
+ diagfilename=ExtremaCheck1.dia

.print tran V(node8) v(node5) V(node1) V(node3)

* Subcircuit and Model Definitions
.SUBCKT CD4049UB 1  2   3   4
*HEX BUFFER    IN OUT VCC VSS
M1 2 1 3 3 CD4049P
M2 2 1 4 4 CD4049N
.MODEL CD4049P PMOS (LEVEL=1 VTO=-2.9 KP=2M 
+ GAMMA=3.97U PHI=.75 LAMBDA=1.87M RD=28.2 
+ RS=45.2 IS=31.2F PB=.8 MJ=.46 CBD=148P 
+ CBS=177P CGSO=218N CGDO=182N CGBO=299N)
.MODEL CD4049N NMOS (LEVEL=1 VTO=2.1 KP=5M 
+ GAMMA=3.97U PHI=.75 LAMBDA=1.87M RD=4.2 
+ RS=4.2 IS=31.2F PB=.8 MJ=.46 CBD=105P 
+ CBS=127P CGSO=156N CGDO=130N CGBO=214N)
.ENDS

.END
\end{NetlistFigure}

Running this netlist in \Xyce{} will produce a diagnostic output file with 
\begin{verbatim}
Xyce Diagnostic Output
 Solution Extrema output requested with extremalimit = 6
 Extreme value found in TRAN analysis at STEP_STARTED time=0.000174793
     V(NODE5)=6.0347
 Extreme value found in TRAN analysis at STEP_SUCCESSFUL time=0.000174793
     V(NODE5)=6.0347
\end{verbatim}

The output shows what type of diagnostic was requested, the diagnostic limit imposed
and then points at when the limit was reached in the solution.  In this case, the 
voltage node \texttt{NODE5} exceeded the limit of 6 with a value of 6.0347 at the 
simulation time of 0.000174793 seconds.

The extrema output diagnostic can be useful to determine the largest values encountered
during a simulation.  Once the extrema are known it can be useful to focus on when 
voltages or currents exceed a set value.  That will be examined in the next section.

\subsection{Voltage and Current Output}
Similar to \texttt{EXTREMALIMIT}, the diagnostic output options, \texttt{VOLTAGELIMIT} and
\texttt{CURRENTLIMIT} allow reporting on voltages and currents at different limits.  Additionally,
unlike \texttt{EXTREMALIMIT}, \texttt{VOLTAGELIMIT} and \texttt{CURRENTLIMIT} report {\bf all} 
voltages and currents that are over a given limit at each step.

Thus, the line
\begin{verbatim}
.options diagnostic VOLTAGELIMIT=5.0 CURRENTLIMIT=3e-2 diagfilename=VICheck1.dia
\end{verbatim}
will report voltage variables that have absolute values greater than 5.0 and currents
with absolute values greater than 3.0e-2.  If used in the oscillator circuit from 
Figure \ref{extremaExample}, the diagnostic output will look like:

\begin{verbatim}
Xyce Diagnostic Output
 Node voltage output requested with voltagelimit = 5
 Lead current output requested with currentlimit = 0.03
...
Voltage over limit value found in TRAN analysis at STEP_STARTED time=7.18202e-05
     V(M:X2:1_drainprime)=5.00228
     V(M:X2:1_sourceprime)=5.00023
     V(NODE8)=5.00232
     V(M:X2:2_drainprime)=5.00233
...
Current over limit value found in TRAN analysis at STEP_STARTED time=0.000102911
     solution I(L3_branch)=0.0300498
     solution I(V2_branch)=-0.0300498
     lead I(L3:BRANCH_D)=0.0300498
     lead I(R5:BRANCH_D)=0.0300469
     lead I(V2:BRANCH_D)=-0.0300498
\end{verbatim}

\noindent Note that unlike the extrema diagnostic, the voltage and current diagnostic report all 
solution values and branch currents whose absolute values exceed the limits.  This can
be useful in understanding the operational ranges of currents and voltages within a circuit.

\subsection{Discontinuity Output}
Discontinuous behavior in a solution variable can cause problems for step progression 
in transient analysis as the time integrator will try to minimize dramatic changes 
in the solution.  The time integrator does this by comparing the predicted solution 
that is based on the trajectory of the prior solution points with the actual solution, i.e. corrector, 
found by the nonlinear solver.  If the difference between the predicted solution and the corrector 
is greater than some tolerance, then the time integrator
will reject that step and try a smaller time step.  When the circuit has a true 
discontinuity in it, smaller time steps will not make the discontinuity smaller and 
time integration may ultimately fail.

The following example circuit has discontinuous behavior in the output from 
the device \texttt{bsrc} which changes from 0 to 1 when the voltage on node 1 goes 
above 1.  Normally this would cause the transient simulation to fail, but this netlist
also instructs the time integrator to ignore predictor-corrector error and just 
accept steps where the nonlinear solver converges.  This option is {\bf not recommended}, 
but used just for illustrative purposes.  

\begin{NetlistFigure}{Example circuit with an intentional discontinuity}{DiscExample}
*
* circuit with an intentional discontinuity 
*

Vin 1 0 pulse 0 5 1e-7 1e-6 1e-6 1e-3 2e-3

rload 1 0 10


bsrc 3 0 v = \{if(v(1)>1, 1,0)\}
rload3 3 0 100


.options timeint erroption=1

.options diagnostic DISCLIMIT=1e-2 
+ diagfilename=DisconCheck2.dia

.print tran v(1) v(3)
.tran 0 1e-2

.end

\end{NetlistFigure}

\noindent When this circuit is run in \Xyce{}, the diagnostic output will look like:
\begin{verbatim}
Xyce Diagnostic Output
 Discontinuity output requested with disclimit = 0.01
 ...
 Discontinuity over limit value found in TRAN analysis at STEP_STARTED time=4.27e-07
     V(3)=1 xPredictor  = 0 qPredictor  = 0 |diff| = 1
\end{verbatim}

\noindent The output shows what type of diagnostic was requested, the diagnostic limit imposed
and then points at when node 3 had a discontinuity of 1 at the simulation time where
node 1 triggered the step change in output.  If we used \Xyce{}'s standard time integration 
options and removed the \texttt{.options timeint erroption=1} line from the above netlist 
then \Xyce{} would fail with a time-step-too-small error.  The diagnostic output would show:

\begin{verbatim}
Xyce Diagnostic Output
 Discontinuity output requested with disclimit = 0.01
 ...
 Discontinuity over limit value found in TRAN analysis at STEP_STARTED time=3.01602e-07
     V(1)=0.990008 xPredictor  = 1.06201 qPredictor  = 0 |diff| = 0.072
 Discontinuity over limit value found in TRAN analysis at STEP_FAILED time=3.01602e-07
     V(1)=0.990008 xPredictor  = 1.00801 qPredictor  = 0 |diff| = 0.018
\end{verbatim}

At the start of a transient step \texttt{V(1)=0.990008} and the predicted 
value for \texttt{V(1)} the end of that step would be 1.06201.  That would trigger \texttt{V(3)} 
to jump from 0 to 1 and thus the step fails.  When running the time integration with the default 
step-error control, the actual discontinuity at node 3 is harder to see because its effect
is limiting the change on node 1.  However, given that the simulation is having 
problems when node 1 is trying to cross the voltage threshold 1.0, it provides useful insight into
where the problem is.  Thus, for best use of discontinuity output, it may be necessary to try 
turning step error control on and off in a set of simulations.  

Finally, it is recommended to not set \texttt{DISCLIMIT} 
too small, i.e. below the time integrator \texttt{ABSTOL}, as this will cause most of the 
solution variables to be reported as discontinuous. 

%\section{Simulation Issues}
\label{SimulationIssues}

