\documentclass[11pt]{article}
\usepackage{tikz}
\usepackage[empty]{fullpage}
\usepackage{fancybox,graphicx}

\RequirePackage{alltt}

\newenvironment{vquote}{\begin{quote}\begin{alltt}}
    {\end{alltt}\end{quote}}

\begin{document}

\begin{figure}[hbt]
\centering
\resizebox{.98\linewidth}{!}{ \input quick }
\caption{Gilbert Cell Mixer.  The netlist is given in figure~\ref{Netlist}, and has previously appeared in figure 7.7 of~\cite{Xyce_Users_Guide_5_2}.  There are many possible variations, and the original Gilbert cell reference is~\cite{1049925}.  In this version, the LO voltage is given by \texttt{v(6,2)} and \texttt{v(15,10)} is the input. The output is the difference between collector voltages of Q4/Q6 and Q3/Q5 (\texttt{V(5,3)}), and should be the approximate product of the LO and input voltages.}

\label{Symbolic_label}
\end{figure}

\begin{figure}[htbp]
  \begin{centering}
    \shadowbox{
      \begin{minipage}{0.8\textwidth}
        \begin{vquote}
\color{blue}* A gilbert cell mixer\color{black}
R5 1 2 100
Q3 3 2 4 DEFAULTS
Q4 5 6 4 DEFAULTS
Q5 3 6 8 DEFAULTS
Q6 5 2 8 DEFAULTS
R6 2 1 100
\color{blue}*the local oscillator\color{black}
VLO 6 2 DC 0 SIN(0 .05V 4e6 0 0)
Q1 4 10 11 DEFAULTS
R1 11 12 10
R2 12 13 10
\color{blue}* input bias current\color{black}
I1 12 0 DC 1.8mA
Q2 8 15 13 DEFAULTS
R4 15 16 1500
R3 16 10 1500
V1 16 0 DC 1.8V
\color{blue}*the input voltage to be mixed with the LO\color{black}
V5 15 10 DC 0 sin(0 .05V 3e6 0 0)
R7 5 17 1500
R8 3 17 1500
V3 17 0 DC 8V
V2 1 0 DC 6V
.MODEL DEFAULTS NPN

.TRAN 1ns 1us 
.PRINT TRAN v(6,2) v(15,10) v(5,3)

.SENS objfunc={v(5)} param=DEFAULTS:IS
.options sensitivity adjoint=0 direct=1
.print SENS format=tecplot v(5)
.end
\end{vquote}
\end{minipage}
}
\caption[Gilbert Cell Mixer netlist]
{Gilbert Cell Mixer netlist \label{Netlist} }
\end{centering}
\end{figure}

\bibliographystyle{unsrt}
\bibliography{gilbert}

\end{document}
