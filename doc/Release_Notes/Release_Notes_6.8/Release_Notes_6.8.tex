% Sandia National Laboratories is a multimission laboratory managed and
% operated by National Technology & Engineering Solutions of Sandia, LLC, a
% wholly owned subsidiary of Honeywell International Inc., for the U.S.
% Department of Energy’s National Nuclear Security Administration under
% contract DE-NA0003525.

% Copyright 2002-2019 National Technology & Engineering Solutions of Sandia,
% LLC (NTESS).


%%-------------------------------------------------------------------------
%% Purpose        : Release notes for Xyce version 6.8.
%% Special Notes  : Graphic files (pdf format) work with pdflatex.  
%% Creation Date  : {03/24/2006}
%%-------------------------------------------------------------------------

\documentclass{article}
\usepackage[hyperindex=true, colorlinks=false]{hyperref}
\usepackage{ltxtable, multirow}
\usepackage{Xyce}
\usepackage{geometry}

\pdfcatalog {/PageMode /UseNone}
\renewcommand{\arraystretch}{1.2}

% Sets the page margins to be the same as the Guides (SAND reports)
\geometry{pdftex, inner=1in, textwidth=6.5in, textheight=9in}

% Gets rid of Section numbers
\setcounter{secnumdepth}{0}

% Set this here once, and use \XyceVersionVar{} in the document
\XyceVersion{6.8}

% ---------------------------------------------------------------------------- %
%
% Set the title, author, and date
%
\title{\XyceTitle{} Parallel Electronic Simulator\\
Version \XyceVersionVar{} Release Notes} 

\author{ Sandia National Laboratories}

\date{\today}

% Submitted to R&A 05 October 2017, Tracking number 703121, approved 18 October 2017
% SAND Number SAND2017-11342 O
% ---------------------------------------------------------------------------- %
% Start the document

\begin{document}
\maketitle

The \XyceTM{} Parallel Electronic Simulator has been written to support the
simulation needs of Sandia National Laboratories' electrical designers.
\XyceTM{} is a SPICE-compatible simulator with the ability to solve extremely
large circuit problems on large-scale parallel computing platforms, but also
includes support for most popular parallel and serial computers.

For up-to-date information not available at the time these notes were produced,
please visit the \XyceTM{} web page at
{\color{XyceDeepRed}\url{http://xyce.sandia.gov}}.


\tableofcontents
\vspace*{\fill}
\parbox{\textwidth}
{
  \raisebox{0.13in}{\includegraphics[height=0.5in]{snllineblubrd}}
  \hfill
  \includegraphics[width=1.5in]{xyce_flat_white}
}


\newpage
\section{New Features and Enhancements}

\subsubsection*{New Devices and Device Model Improvements}
\begin{XyceItemize}
\item A code optimization of the BSIM-CMG model 110 (MOSFET level 110)
  has improved performance of this model when self-heating effects are
  disabled by setting SHMOD=0 or RTH=0 in the .model card.
  
\item The VBIC 1.2 (Level=10) model has been deprecated since \Xyce{}
  Release 6.6.  It is still available in \Xyce{} 6.8, but is
  unmaintained and will be removed in a future release.  To remind
  users that this model is slated for future removal, \Xyce{} will show
  the text ``*DEPRECATED*'' next to the model name in its device count
  summary.

  The VBIC 1.3 model (levels 11 and 12) will be the only supported
  VBIC models in the future.  Please update any netlists you have that
  use the level=10 model to use one of these VBIC 1.3 models instead.
  See the \Xyce{} Reference Guide for details on the differences between
  the 1.2 and 1.3 versions of VBIC.

  Users are advised to update their VBIC netlists as soon as possible.
  
\item The Piecewise Empirical Model (PEM) memristor model has been added to \Xyce{}.  See the 
  \Xyce{} Reference Guide for details.
  
\end{XyceItemize}

\subsubsection*{Enhanced Solver Stability, Performance and Features}
\begin{XyceItemize}
 \item The BDF time integration method (METHOD=BDF or METHOD=6) has
   been deprecated since version 6.6 of \Xyce{}, and has been removed in
   this release. The Trapezoid (METHOD=TRAP or METHOD=7) and Gear
   (METHOD=GEAR or METHOD=8) methods are the only supported time
   integration methods in version 6.8.  Please update any netlists you
   have that explicitly use the BDF time integration method to use the
   Gear method instead.

 \item Improved robustness of GMIN stepping and SPICE DC OP strategy.
\end{XyceItemize}

\subsubsection*{Interface Improvements}
\begin{XyceItemize}
\item The CSV print format is now supported for homotopy and noise output.
\item A new GNUPLOT print format that outputs data in standard columns, like a
.prn file, but with improved Gnuplot compatibility for .STEP data.
\end{XyceItemize}

\subsubsection*{Deployment Improvements}
\begin{XyceItemize}
\item The wrapper scripts \texttt{runxyce} and \texttt{xmpirun} are no
  longer provided by our packaged installers.  \Xyce{} is now run
  directly without the use of these scripts.  For details please see
  the \Xyce{} Users' Guide.
\item This version of \Xyce{} has been built against the latest
  released version 12.12.1 of Trilinos.
\item \Xyce{} no longer requires the UMFPACK library for serial builds 
  and neither UMFPACK nor ParMETIS are required for parallel builds.
\end{XyceItemize}

\newpage
\section{Defects Fixed in this Release}
% Sandia National Laboratories is a multimission laboratory managed and
% operated by National Technology & Engineering Solutions of Sandia, LLC, a
% wholly owned subsidiary of Honeywell International Inc., for the U.S.
% Department of Energy's National Nuclear Security Administration under
% contract DE-NA0003525.

% Copyright 2002-2025 National Technology & Engineering Solutions of Sandia,
% LLC (NTESS).

% Sandia National Laboratories is a multimission laboratory managed and
% operated by National Technology & Engineering Solutions of Sandia, LLC, a
% wholly owned subsidiary of Honeywell International Inc., for the U.S.
% Department of Energy's National Nuclear Security Administration under
% contract DE-NA0003525.

% Copyright 2002-2025 National Technology & Engineering Solutions of Sandia,
% LLC (NTESS).


%%
%% Fixed Defects.
%%
{
\small

\begin{longtable}[h] {>{\raggedright\small}m{2in}|>{\raggedright\let\\\tabularnewline\small}m{3.5in}}
     \caption{Fixed Defects.  The Xyce team has multiple issue
     trackers, and the table below indicates fixed issues by
     indentifying both the tracker and the issue number.  Further,
     some issues are reported by open source users on GitHub and these
     issues may be tracked using multiple issue numbers.} \\ \hline
     \rowcolor{XyceDarkBlue} \color{white}\textbf{Defect} & \color{white}\textbf{Description} \\ \hline
     \endfirsthead
     \caption[]{Fixed Defects.  Note that we have two multiple issue tracking systems for Sandia Users.
     SON and SRN refer to our legacy open- and restricted-network Bugzilla system, and Gitlab refers to issues in our gitlab repositories.  } \\ \hline
     \rowcolor{XyceDarkBlue} \color{white}\textbf{Defect} & \color{white}\textbf{Description} \\ \hline
     \endhead

  \textbf{Xyce Project Backlog/xxx}: Desciption
  &  Details
  \\\hline

\textbf{Xyce Backlog Bugs/64}: LTRA device does not properly initialize history vector when UIC/NOOP is used &
  The lossy transmission line device (LTRA) was not bein intialized correctly,
  for transient simulations that skipped the DCOP calculation.  As a result,
  transient simulations using this device would crash if the netlist contained
  a \texttt{UIC} or \texttt{NOOP} keyword on the \texttt{.TRAN} line.  This has
  been fixed.  \\ \hline

  \textbf{Xyce Project Backlog/571}: Modify \Xyce{} parser so that it can
  handle multiple \texttt{.options} statements for the same component (for
  example, support \texttt{.options timeint} statements) &
  Most SPICE-style simulators use the \texttt{.option} (non-plural) command,
  and allow the netlist to have multiple \texttt{.option} comamnds.   \Xyce{}
  historically was different in two respects.  One difference is that \Xyce{}
  uses a plural \texttt{.options} command, and another difference is a required
  keyword to identify the part of the code to which the block of options is
  applied.  Also, finally, \Xyce{} was designed to only have a single
  \texttt{.options} statement of each type.  So, for example, at most one
  \texttt{.options device} statement, at most one \texttt{.options timeint}
  statement, etc.  Unforutunately, recent versions of \Xyce{} silently accepted
  multiple \texttt{.options} statements, and behaved differently depending on
  the keyword.  For example, if the netlist contained multiple \texttt{.options
  timeint} statements, \Xyce{} would only use one of them and silently ignore
  the others.  In contrast, if the netlist contained multiple \texttt{.options
  device} statements, it would use them all, but would ignore duplicate
  parameters.  This was confusing for users of other simulators, and has been
  fixed.  \Xyce{} now treats multiple statements as a combined single
  statement, and issues warnings for duplicate parameters.
  \\\hline
 
 \textbf{Xyce Project Backlog/609}: 
  Global parameters that are applied to subcircuit instance parameters don't work in parallel & 
  If a subcircuit instance parameter (on the \texttt{X} line) depended on a
  \texttt{.param} or \texttt{.global\_param}, and that parameter was a
  variable, this dependency worked in serial, but not parallel.  In this case,
  ``variable'' means parameters that are allowed to change as part of a
  \texttt{.STEP} sweep, or as part of a sampling method.  This was related to
  the order of operations in the parser.  This has been fixed.  \\ \hline

 \textbf{Xyce Project Backlog/631}: 
  Expression-based local variation doesn't always work with subcircuit arguments & 
  When expression-based local variation was applied to a subcircuit argument, it 
  was handled incorrectly.  This meant that subcircuit arguments that should have 
  had a random distribution were fixed to the mean value during sampling.  
  This has been fixed.  \\ \hline

  \textbf{Xyce Project Backlog/637}: Support shared library build of \Xyce{} under Windows.
  &  Modify the cmake build system to support building a shared library version
  of xyce, \texttt{xyce-share.dll}, so that Xyce functionality can be accessed
  as a library from other applications.  \\\hline


\end{longtable}
}



\newpage
\section{Interface Changes in this Release}
% Sandia National Laboratories is a multimission laboratory managed and
% operated by National Technology & Engineering Solutions of Sandia, LLC, a
% wholly owned subsidiary of Honeywell International Inc., for the U.S.
% Department of Energy’s National Nuclear Security Administration under
% contract DE-NA0003525.

% Copyright 2002-2021 National Technology & Engineering Solutions of Sandia,
% LLC (NTESS).

% Sandia National Laboratories is a multimission laboratory managed and
% operated by National Technology & Engineering Solutions of Sandia, LLC, a
% wholly owned subsidiary of Honeywell International Inc., for the U.S.
% Department of Energy’s National Nuclear Security Administration under
% contract DE-NA0003525.

% Copyright 2002-2021 National Technology & Engineering Solutions of Sandia,
% LLC (NTESS).


%%
%% Changes to Xyce input since the last release.
%%
{
\small

\begin{longtable}[h] {>{\raggedright\small}m{2in}|>{\raggedright\let\\\tabularnewline\small}m{3.5in}}
  \caption{Changes to netlist specification since the last release.\label{newUsage}} \\ \hline
  \rowcolor{XyceDarkBlue}
  \color{white}\textbf{Change} &
  \color{white}\textbf{Detail} \\ \hline \endfirsthead
  \caption[]{Changes to netlist specification since the last release.\label{newUsage}} \\ \hline
  \rowcolor{XyceDarkBlue}
  \color{white}\textbf{Change} &
  \color{white}\textbf{Detail} \\ \hline \endhead

Continued support for FRAC\_MAX qualifier on TRIG-TARG measure lines &
For backwards compatibility with previous \Xyce{} versions for internal
users, \texttt{.OPTIONS MEASURE USE\_LTTM=<value>} has been added.  This
option defaults to 0, which uses the new version of the TRIG-TARG measure;
while setting it to 1 will use the old version of the TRIG-TARG measure
for all TRIG-TARG measures in the netlist.  If the FRAC\_MAX qualifier is
used on a TRIG-TARG line then \Xyce{} will automatically default to
\texttt{USE\_LTTM=1} for that particular measure line.  \\ \hline

\textbf{Gitlab-ex issue 338}: Added the function \texttt{Simulator:: getTimeVoltagePairsSz} & The function
\texttt{Simulator::getTimeVoltagePairsSz} supplies the maximun 
number of time, voltage or state values for an ADC in a subsequent call to 
\texttt{Simulator::getTimeVoltagePairs} or \texttt{Simulator::getTimeStatePairs}.
Also, calling \texttt{Simulator::getTimeStatePairs} no longer clears the voltage
history of an ADC, so a calling program that needs both state and voltage history
can first call \texttt{Simulator::getTimeStatePairs} and then call 
\texttt{Simulator::getTimeVoltagePairs} to get both.  Calling 
\texttt{Simulator::getTimeVoltagePairs} still clears the ADC history.
See the Application note, Mixed Signal Simulation with \Xyce{} 7.5 for further 
details. \\ \hline

\textbf{Gitlab-ex issue 348}:  Added the function \texttt{xyce\_getTimeVoltagePairs\-ADCLimitData()}  &
  A new function has been added to the \texttt{XyceCInterface} called 
  \texttt{xyce\_getTimeVoltagePairsADCLimitData()} which limits the data copied 
  to the caller allocated space to whatever maximum 
  allocation length is provided.  This avoids potential memory overwriting that
  could occur with the general access function \texttt{xyce\_getTimeVoltagePairsADC()}.
  See the Application note, Mixed Signal Simulation with \Xyce{} 7.5 for further details.  \\ \hline

\end{longtable}
}



\newpage
\section{Known Defects and Workarounds}
% Sandia National Laboratories is a multimission laboratory managed and
% operated by National Technology & Engineering Solutions of Sandia, LLC, a
% wholly owned subsidiary of Honeywell International Inc., for the U.S.
% Department of Energy’s National Nuclear Security Administration under
% contract DE-NA0003525.

% Copyright 2002-2019 National Technology & Engineering Solutions of Sandia,
% LLC (NTESS).


%%
%% Known defects and workarounds.
%%
%% This section should highlight significant defects that were not fixed in
%% the release

\small

\begin{longtable}[h] {>{\raggedright\small}m{2in}|>{\raggedright\let\\\tabularnewline\small}m{3.5in}}
  \caption{Known Defects and Workarounds.} \\ \hline
  \rowcolor{XyceDarkBlue} \color{white}\bf Defect & \color{white}\bf Description
  \\ \hline \endfirsthead
  \caption[]{Known Defects and Workarounds.} \\ \hline
  \rowcolor{XyceDarkBlue} \color{white}\bf Defect & \color{white}\bf Description
  \\ \hline \endhead
% EXAMPLE:
%\textbf{bug number-SRN}: bug title & Description of KNOWN BUG THAT HAS NOT BEEN
  %FIXED.

%\textbf{\textit{Workaround}}: Describe how to work around this bug.
%\\ \hline
%
%  
\textbf{27-SON}: Fix handling of .options parameters & When specifying .options
  for a particular package, what gets applied as the non-specified default
  options might change.  \\ \hline


\textbf{37-SON}: Connectivity checking is broken for devices with more than 10
leads & The diagnostic code used by the \Xyce{} setup that checks circuit
topology for basic errors such as a node having no DC path to ground or a node
being connected to only one device has a bug in it that causes the code to emit
a cryptic error message, after which the code will exit.  This error has so far
only been seen when a user has attempted to connect a large number of inductors
together using multiple mutual inductor lines.  The maximum number of
non-ground leads that can be used without confusing this piece of code is 10.
If your circuit has that type of large, highly-connected mutual inductor and
the code exits with an error message, this bug may be the source of the problem.

The error message now includes a recommendation to use the workaround below.

\textbf{\textit{Workaround}}: Disable connectivity checking by adding the line

\begin{alltt} .OPTIONS TOPOLOGY CHECK_CONNECTIVITY=0 \end{alltt}

to your netlist.  This will disable the check for the basic errors such as
floating nodes and improperly connected devices, but will allow the netlist to
run with a highly-connected mutual inductor.
\\ \hline


\textbf{49-SON} \Xyce{} BSIM models recognize the model TNOM, but not the
instance TNOM & 
\\ \hline



\textbf{247-SON}: Expressions don't work on .options lines & Expressions enclosed
in braces (\{ \}) are handled specially throughout \Xyce{}, and may only be used
in certain contexts such as in device model or instance parameters or on
\texttt{.PRINT} lines.
\\ \hline


\textbf{250-SON}: NODESET in \Xyce{} is not equivalent to NODESET in SPICE & As
currently implemented, .NODESET applies the initial conditions given throughout
a full nonlinear solve for the operating point. It then uses that result as an
initial guess for a second nonlinear solve with no constraints.  This is not
the same as SPICE, which merely applies the given initial conditions to a
single nonlinear solve for the first two iterations, then lets the problem
converge with no further constraints.  This can lead to Xyce's .NODESET failing
where the same netlist in SPICE might not, if the initial conditions are such
that a full nonlinear solve cannot converge with those constraints in place.
There is no workaround.
\\ \hline


\textbf{468-SON}: It should be legal to have two model cards with the same model
name, but different model types. & SPICE3F5 and ngspice only require that
model cards of the same type have unique model names. They accept model cards
of different types with the same name.  \Xyce{} requires that all model card names be unique.
\\ \hline


\textbf{469-SON}: Belos memory consumption on FreeBSD and excessive CPU on other
platforms &
Memory or thread bloat can result when using multithreaded dense linear algebra
libraries, which are employed by Belos.  If this situation is observed, either build
\Xyce{} with a serial dense linear algebra library or use environment variables to control
the number of spawned threads in a multithreaded library.
\\ \hline


\textbf{583-SON}: Switch with RON=0 leads to convergence failure. &  The switch device does not prevent a user from specifying \texttt{RON=0} in its model, but then takes the inverse of this value to get the ``on'' conductance.  The resulting invalid division will either lead to a division by zero error on platforms that throw such errors, or produce a conductance with ``Not A Number'' or ``Infinity'' as value.  This will lead to a convergence failure.

\textbf{\textit{Workaround}}: Do not specify an identically zero resistance for the switch's ``on'' value.  A small value of resistance such as 1e-15 or smaller will generally work well as a substitute. \\ \hline

\textbf{620-SON}: TO, FROM and TD Qualifiers Function Incorrectly in the FREQ Measure &
The use of these qualifiers in a {\tt FREQ} measure statement may give an incorrect
measurement window.  This is noted in the Reference Guide. \\ \hline

\textbf{621-SON}: FIND-WHEN Measure in \Xyce{} Only Correctly Supports the WHEN Syntax. 
FIND-WHEN is incorrect. & An example is given in the HSpice Compatibility subsection of
the {\tt .MEASURE} section of the Reference Guide. \\ \hline

\textbf{1595-SRN}: \Xyce{} won't allow access to inductors within subcircuits for
mutual inductors external to subcircuits & It is not possible to have a mutual
inductor outside of a subcircuit couple to inductors in a subcircuit.

\textbf{\textit{Workaround}}: Put all inductors and mutual inductance lines that couple to
them together at the same level of circuit hierarchy.
\\ \hline


\textbf{1903-SRN}: \Xyce{} fails to collect several inductors into a linear mutual inductor &
In some rare cases with complex include file usage, the mutual inductor syntax with multiple couplings can fail to work. \Xyce{} will return an error message that it can not find 
\texttt{L\_L1} in this example:
{\tt
\begin{verbatim}
L_L1         node1 node1 inductance1
L_L2         node3 node4  inductance2
L_L3         node5  node6  inductance3
L_L4         node7 node8  inductance4
K_K1         L_L1 L_L2 L_L3 L_L4   .999 
\end{verbatim}
}

\\ \hline


\textbf{1923-SRN}: LC lines run out of memory, even if equivalent (larger) RLC
lines do not. &  In some cases, circuits that run fine using an RLC approximation for a
transmission line, exit with an out-of-memory error if the (supposedly smaller) LC
approximation is used.
\\ \hline

\end{longtable}




\newpage
\section{Supported Platforms}
\subsection*{Certified Support}
The following platforms have been subject to certification testing for the
\Xyce{} version 6.8 release.
\begin{XyceItemize}
  \item Red Hat Enterprise Linux${}^{\mbox{\textregistered}}$ 7, x86-64 (serial and parallel)
  \item Red Hat Enterprise Linux${}^{\mbox{\textregistered}}$ 6, x86-64 (serial and parallel)
  \item Microsoft Windows 10${}^{\mbox{\textregistered}}$, x86-64 (serial)
  \item Apple${}^{\mbox{\textregistered}}$ OS X Sierra, x86-64 (serial and parallel)
\end{XyceItemize}

\subsection*{Build Support}
Though not certified platforms, \Xyce{} has been known to run on the following
systems.
\begin{XyceItemize}
  \item FreeBSD 10.x on Intel x86-64 architectures (serial and parallel)
  \item Distributions of Linux other than Red Hat Enterprise Linux 6
  \item Microsoft Windows under Cygwin and MinGW.
\end{XyceItemize}


\section{\Xyce{} Release \XyceVersionVar{} Documentation}
The following \Xyce{} documentation is available on the \Xyce{} website in pdf
form.
\begin{XyceItemize}
  \item \Xyce{} Version \XyceVersionVar{} Release Notes (this document)
  \item \Xyce{} Users' Guide, Version \XyceVersionVar{}
  \item \Xyce{} Reference Guide, Version \XyceVersionVar{}
  \item \Xyce{} Mathematical Formulation
  \item Power Grid Modeling with \Xyce{}
  \item Application Node: Using Open Source Schematic Capture Tools with \Xyce{}
\end{XyceItemize}
Also included at the \Xyce{} website as web pages are the following.
\begin{XyceItemize}
  \item Frequently Asked Questions
  \item Building Guide (instructions for building \Xyce{} from the source code)
  \item Running the \Xyce{} Regression Test Suite
  \item \Xyce{}/ADMS Users' Guide
  \item Tutorial:  Adding a new compact model to \Xyce{}
\end{XyceItemize}


\section{External User Resources}
\begin{itemize}
  \item Website: {\color{XyceDeepRed}\url{http://xyce.sandia.gov}}
  \item Google Groups discussion forum:
    {\color{XyceDeepRed}\url{https://groups.google.com/forum/#!forum/xyce-users}}
  \item Email support:
    {\color{XyceDeepRed}\href{mailto:xyce@sandia.gov}{xyce@sandia.gov}}
  \item Address:
    \begin{quote}
            Electrical Models and Simulation Department,\\
            Sandia National Laboratories\\
            P.O. Box 5800, M.S. 1177\\
            Albuquerque, NM 87185-1177 \\
    \end{quote}
\end{itemize}

\vspace*{\fill}
Sandia National Laboratories is a multimission laboratory managed and operated by 
National Technology and Engineering Solutions of Sandia, LLC, a wholly owned subsidiary
of Honeywell International, Inc., for the U.S. Department of Energy's National Nuclear 
Security Administration under contract DE-NA0003525.

\end{document}

