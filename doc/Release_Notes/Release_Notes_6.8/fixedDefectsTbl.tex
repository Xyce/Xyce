% Sandia National Laboratories is a multimission laboratory managed and
% operated by National Technology & Engineering Solutions of Sandia, LLC, a
% wholly owned subsidiary of Honeywell International Inc., for the U.S.
% Department of Energy’s National Nuclear Security Administration under
% contract DE-NA0003525.

% Copyright 2002-2019 National Technology & Engineering Solutions of Sandia,
% LLC (NTESS).



%%
%% Fixed Defects.
%%
{
\small

\begin{longtable}[h] {>{\raggedright\small}m{2in}|>{\raggedright\let\\\tabularnewline\small}m{3.5in}}
     \caption{Fixed Defects.  Note that we have two different Bugzilla systems for Sandia users.
     SON, which is on the open network, and SRN, which is on the restricted network. } \\ \hline
     \rowcolor{XyceDarkBlue} \color{white}\bf Defect & \color{white}\bf Description \\ \hline
     \endfirsthead
     \caption[]{Fixed Defects.  Note that we have two different Bugzilla systems for Sandia Users.
     SON, which is on the open network, and SRN, which is on the restricted network. } \\ \hline
     \rowcolor{XyceDarkBlue} \color{white}\bf Defect & \color{white}\bf Description \\ \hline
     \endhead

     \textbf{896-SON}:  Do away with ``runxyce'' and ``xmpirun'' scripts
     entirely & The \texttt{runxyce} and \texttt{xmpirun} scripts were
     historical artifacts dating back to early versions of \Xyce{}.  They were
     present only in released binary installs of the code, which required us to
     document two different methods of running the code depending on whether a
     user had installed via source or via binary installers.  Furthermore in
     recent releases have been doing nothing but invoking the \Xyce{} binary
     directly, and the scripts have actually been unnecessary since \Xyce{}
     Release 6.6.  These scripts have been removed entirely, and now users of
     \Xyce{} invoke the code in the same way whether they are installing from
     source or from binary distributions. \\ \hline 

     \textbf{28-SON}: Problems with parameter resolution when mutual inductors exist in subcircuits &
     When a subcircuit contained mutual inductors, a bug in the input parser's
     context resolution code led \Xyce{} to report incorrect errors regarding
     unrecognized symbols.  The error in context resolution has been fixed, and
     these error messages are no longer generated. \\ \hline

     \textbf{913-SON}: ``Unmatched parentheses'' error when certain words are
     used as parameter names & The implementation of ``vector composite
     parameters'' as used by a small number of special devices led to 15
     specific parameter names being unavailable for use as user defined
     parameters in \texttt{.PARAM} statements if the parameter value was an
     expression enclosed in curly braces.  These parameters were
     \texttt{DOPINGPROFILES}, \texttt{SOURCELIST}, \texttt{LAYER},
     \texttt{REGION}, \texttt{NODE}, \texttt{RXNREGION}, \texttt{COIL},
     \texttt{FIELDDATA}, \texttt{PARAM}, \texttt{MM\_CURRENT},
     \texttt{MM\_INDVARS}, \texttt{MM\_INDFEQUS}, \texttt{MM\_INDQEQUS},
     \texttt{MM\_FUNCTIONS}, and \texttt{MM\_PARAMETERS}.  Due to a bug in
     handling of these special names, if one used these names as the name of a
     parameter in a \texttt{.PARAM} statement with a value that is an
     expression enclosed in curly braces (``\{\}'') then \Xyce{} would emit an
     unhelpful ``Unmatched parentheses'' error.  The parser handling of vector
     composite was improved and these parameter names no longer behave as if
     they were reserved words.  \\ \hline

     \textbf{646-SON}: .csd format for PROBE output is incorrect for specific numbers of 
     output variables for .AC analyses  & This line (\texttt{.PRINT AC FORMAT=PROBE V(1) V(2) V(3) V(4)}) 
     would produce an incorrectly formatted .csd file that would not open in PSpice 16.6. In 
     general, the bug occurred for .AC analyses if the number of output variables (N) 
     on the \texttt{.PRINT AC FORMAT=PROBE} line satisfied this equation: (N+1)mod5 = 0.  
     This bug was fixed for .TRAN and .DC analyses for version 6.4 of \Xyce{}. 
     It is now also fixed for .AC analyses.  \\ \hline

     \textbf{785-SON}: \Xyce{} hangs in parallel when passed a directory rather than a file &
     This bug was related to SON Bug 730 (``\Xyce{} hangs when passed a directory instead of a 
     netlist ''), which was fixed in \Xyce{} 6.5.  Further testing showed that \Xyce{}
     would hang (during parallel execution) if a directory name was used, rather than a file name,
     in \texttt{.LIB} or \texttt{.INC} statements.  \Xyce{} might also hang in parallel, or
     silently not produce the requested file in both serial and parallel, if the ``output file''
     was either a directory or a file in a non-existent directory.  (Note: the \Xyce{}
     outputters will not create new subdirectories.)  That could happen with either the 
     \texttt{FILE=} keyword on a \texttt{.PRINT} line or with the \texttt{-o}, \texttt{-l} 
     or \texttt{-rsf} command line options.  These issues are fixed now.  In addition, the 
     error messages for all invalid input/output file names should be clearer now.  \\ \hline

     \textbf{843-SON}: Use of .STEP with .NOISE does not result in a complete output
     (.NOISE.prn) file & When .STEP was used with .NOISE, the output file would not 
     contain the simulation results for all of the requested steps. This is fixed now.
     \\ \hline

     \textbf{850-SON}: Segfault and improper error handling for improperly formated functions &
     When a user-defined function was called with the wrong number of parameters, \Xyce{} was 
     supposed to abort with an error message.  However, in some cases \Xyce{} segfaulted instead.  
     There were also cases where \Xyce{} would improperly run a simulation with an incomplete 
     argument list for a user-defined function.  This is fixed now. \\ \hline

     \textbf{903-SON}: .csd format for probe output is incorrect for large numbers 
     of output variables (e.g., for V(*)) &  This bug was found via a 
     \texttt{.PRINT TRAN FORMAT=PROBE V(*)} line that effectively output over 1000 
     variables.  The resultant \Xyce{}-generated .csd file would not open correctly 
     in the PSpice A/D waveform viewer.  The issue appeared to be a limit on the number of 
     characters that could be on a single line in the \#N (variable names) section of the 
     .csd file header.  \Xyce{} now limits the length of those lines.  So, this bug is fixed
     now, with the caveat that an extremely long variable name (e.g., $>$ 400 characters)
     might still cause problems. \\ \hline

     \textbf{909-SON}: Fix core dump for PWL source with FILE specified & This instance
     line (\texttt{VPWL1 1 0 PWL FILE}), where the file name for the Piecewise Linear
     (PWL) source specification is missing, would cause a core dump. This is fixed 
     now. \\ \hline

     \textbf{912-SON}: Fix core dump for .PRINT HOMOTOPY FORMAT=PROBE & \Xyce{}
     would core dump if a homotopy occurred and the print format was 
     \texttt{FORMAT=PROBE}.  That could occur during a transient simulation, using
     \texttt{.PRINT TRAN FORMAT=PROBE}, even if there was no \texttt{.PRINT
     HOMOTOPY} line in the netlist.  This is fixed now.  The \Xyce{}-generated
     .HOMOTOPY.csd files are now viewable within PSpice A/D with the caveats given
     in the Known Defects Table for SON Bug 939. \\ \hline
     
     \textbf{631-SON}: In support of memristor modeling needs, the TEAM, Yakopcic
     and PEM memristor models have been added to \Xyce{}  & See the reference 
     guide for details on the individual models. \\ \hline

     \textbf{933-SON}: \Xyce{} fails to run with Belos as the linear solver if singleton
     filtering reduces linear system to have zero rows & This bug is related to SON Bug
     141 (which was fixed in \Xyce{} 6.0), where this issue was resolved and tested for the
     AztecOO linear solver.  The issue is that \Xyce{} uses transformations to redistribute
     and permute the linear system.  For some circuits, these transformations result in
     a trivial linear system or generate a permutation that causes problems for the 
     iterative solvers.  This is fixed now. \\ \hline

     \textbf{805-SON}: Parser fails to collect linear mutual inductors in an include file &
     \Xyce{} would fail to collect component inductors together, into a mutual inductor, when
     they were in an include file rather than in the top-level netlist file.  This is fixed now.
     \\ \hline

     \textbf{1903-SRN}: \Xyce{} fails to collect several inductors into a linear mutual inductor &
     \Xyce{} was failing to collect linear mutual inductors defined in included files.  This was an
     issue with netlist parsing and has been resolved. \\ \hline

     \textbf{907-SON}: \Xyce{} parser fails to handle hierarchical .lib statements &
     \Xyce{} was failing to parse .lib statements that are found within other .lib files.
     This has been fixed and the parser can handle more complex hierarchical
     .lib parsing now.  \\ \hline

     \textbf{2073-SRN}: \Xyce{} could segfault, when .PRINT lines had no variables specified on them &
     \Xyce{} could segfault when there were multiple .PRINT lines for the same analysis 
     type in a netlist, and one of them had no variables on it. A simple example was:
     {\tt
     \begin{verbatim}
     .PRINT TRAN V(1) V(2)
     .PRINT TRAN
     \end{verbatim}
     }
     This is fixed now.\\ \hline

     \textbf{946-SON}: GMIN stepping does not work with .IC in some
     cases & The GMIN stepping homotopy method did not work with .IC
     if it was specified using continuation=1.  The .IC is
     ignored. This is fixed now. \\ \hline

     \textbf{947-SON}: SPICE strategy does not initialize properly &
     The SPICE strategy uses Newton first, and then GMIN stepping and
     source stepping continuation methods. It did not initialize
     properly for each method and so could fail with the automatic
     SPICE strategy, but converge with the same source stepping when
     specified explicitly. This is fixed now. \\ \hline

\end{longtable}
}
