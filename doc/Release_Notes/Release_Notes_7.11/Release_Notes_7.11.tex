% Sandia National Laboratories is a multimission laboratory managed and
% operated by National Technology & Engineering Solutions of Sandia, LLC, a
% wholly owned subsidiary of Honeywell International Inc., for the U.S.
% Department of Energy’s National Nuclear Security Administration under
% contract DE-NA0003525.

% Copyright 2002-2025 National Technology & Engineering Solutions of Sandia,
% LLC (NTESS).

\documentclass[letterpaper]{scrartcl}
\usepackage[hyperindex=true, colorlinks=false]{hyperref}
\usepackage{ltxtable, multirow}
\usepackage{Xyce}
\usepackage{geometry}
\usepackage{eso-pic}

\pdfcatalog {/PageMode /UseNone}
\renewcommand{\arraystretch}{1.2}

% Sets the page margins to be the same as the Guides (SAND reports)
\geometry{pdftex, inner=1in, textwidth=6.5in, textheight=9in}

% Gets rid of Section numbers
\setcounter{secnumdepth}{0}

% Set this here once, and use \XyceVersionVar{} in the document
\XyceVersion{7.11}

% ---------------------------------------------------------------------------- %
%
% Set the title, author, and date
%
\title{\XyceTitle{} Parallel Electronic Simulator\\
Version \XyceVersionVar{} Release Notes}

\author{ Sandia National Laboratories}

\date{\today}

% ---------------------------------------------------------------------------- %
% Start the document

\begin{document}
\maketitle

\AddToShipoutPictureBG*{%
  \AtPageUpperLeft{%
    \hspace{\dimexpr\paperwidth-1in\relax}%
    \raisebox{-0.5in}{%
         \makebox[0pt][r]{\textsf{SAND2024-XXXXXR}}%
}}}

The \XyceTM{} Parallel Electronic Simulator has been written to support the
simulation needs of Sandia National Laboratories' electrical designers.
\XyceTM{} is a SPICE-compatible simulator with the ability to solve extremely
large circuit problems on large-scale parallel computing platforms, but also
includes support for most popular parallel and serial computers.

For up-to-date information not available at the time these notes were produced,
please visit the \XyceTM{} web page at
{\color{XyceDeepRed}\url{http://xyce.sandia.gov}}.

\tableofcontents
\vspace*{\fill}
\parbox{\textwidth}
{
  \raisebox{0.13in}{\includegraphics[height=0.5in]{snllineblubrd}}
  \hfill
  \includegraphics[width=1.5in]{xyce_flat_white}
}


\newpage
\section{New Features and Enhancements}

\subsubsection*{XDM}
\begin{XyceItemize}
\item 
\end{XyceItemize}

\subsubsection*{New Devices and Device Model Improvements}
\begin{XyceItemize}
  \item The BSIM SOI version 4.7.0 has been added as the level 70470 MOSFET.  It has been observed that the \Xyce{} implementation of this model may be less robust than the previous version, 4.6.1.  For this reason, version 4.6.1 is still level 70.  Additionally, the older BSIM-SOI 4.5.0 continues to be the level 70450 MOSFET.
  \item The BSIM SOI version 4.6.1 had a capacitance bug that has been fixed.  (see issue 988 in the fixed defects table).  This was originally noticed a long time ago.  It was left unfixed until recently because the official version of this model from the BSIM group didn't include a fix.   But, after empirically observing that several commercial simulators had clearly fixed their own implementations of 4.6.1, the decision was made to fix the \Xyce{} implementation.
\end{XyceItemize}

\subsubsection*{Enhanced Solver Stability, Performance and Features}
\begin{XyceItemize}


\item Add new items here

\end{XyceItemize}

\subsubsection*{Interface Improvements}
\begin{XyceItemize}
\item Add new items here
\end{XyceItemize}

\subsubsection*{Important Announcements}
\begin{XyceItemize}
  \item Add new items here
\end{XyceItemize}

\newpage
\section{Defects Fixed in this Release}
% Sandia National Laboratories is a multimission laboratory managed and
% operated by National Technology & Engineering Solutions of Sandia, LLC, a
% wholly owned subsidiary of Honeywell International Inc., for the U.S.
% Department of Energy's National Nuclear Security Administration under
% contract DE-NA0003525.

% Copyright 2002-2025 National Technology & Engineering Solutions of Sandia,
% LLC (NTESS).

% Sandia National Laboratories is a multimission laboratory managed and
% operated by National Technology & Engineering Solutions of Sandia, LLC, a
% wholly owned subsidiary of Honeywell International Inc., for the U.S.
% Department of Energy's National Nuclear Security Administration under
% contract DE-NA0003525.

% Copyright 2002-2025 National Technology & Engineering Solutions of Sandia,
% LLC (NTESS).


%%
%% Fixed Defects.
%%
{
\small

\begin{longtable}[h] {>{\raggedright\small}m{2in}|>{\raggedright\let\\\tabularnewline\small}m{3.5in}}
     \caption{Fixed Defects.  The Xyce team has multiple issue
     trackers, and the table below indicates fixed issues by
     indentifying both the tracker and the issue number.  Further,
     some issues are reported by open source users on GitHub and these
     issues may be tracked using multiple issue numbers.} \\ \hline
     \rowcolor{XyceDarkBlue} \color{white}\textbf{Defect} & \color{white}\textbf{Description} \\ \hline
     \endfirsthead
     \caption[]{Fixed Defects.  Note that we have two multiple issue tracking systems for Sandia Users.
     SON and SRN refer to our legacy open- and restricted-network Bugzilla system, and Gitlab refers to issues in our gitlab repositories.  } \\ \hline
     \rowcolor{XyceDarkBlue} \color{white}\textbf{Defect} & \color{white}\textbf{Description} \\ \hline
     \endhead

  \textbf{Xyce Project Backlog/xxx}: Desciption
  &  Details
  \\\hline

\textbf{Xyce Backlog Bugs/64}: LTRA device does not properly initialize history vector when UIC/NOOP is used &
  The lossy transmission line device (LTRA) was not bein intialized correctly,
  for transient simulations that skipped the DCOP calculation.  As a result,
  transient simulations using this device would crash if the netlist contained
  a \texttt{UIC} or \texttt{NOOP} keyword on the \texttt{.TRAN} line.  This has
  been fixed.  \\ \hline

  \textbf{Xyce Project Backlog/571}: Modify \Xyce{} parser so that it can
  handle multiple \texttt{.options} statements for the same component (for
  example, support \texttt{.options timeint} statements) &
  Most SPICE-style simulators use the \texttt{.option} (non-plural) command,
  and allow the netlist to have multiple \texttt{.option} comamnds.   \Xyce{}
  historically was different in two respects.  One difference is that \Xyce{}
  uses a plural \texttt{.options} command, and another difference is a required
  keyword to identify the part of the code to which the block of options is
  applied.  Also, finally, \Xyce{} was designed to only have a single
  \texttt{.options} statement of each type.  So, for example, at most one
  \texttt{.options device} statement, at most one \texttt{.options timeint}
  statement, etc.  Unforutunately, recent versions of \Xyce{} silently accepted
  multiple \texttt{.options} statements, and behaved differently depending on
  the keyword.  For example, if the netlist contained multiple \texttt{.options
  timeint} statements, \Xyce{} would only use one of them and silently ignore
  the others.  In contrast, if the netlist contained multiple \texttt{.options
  device} statements, it would use them all, but would ignore duplicate
  parameters.  This was confusing for users of other simulators, and has been
  fixed.  \Xyce{} now treats multiple statements as a combined single
  statement, and issues warnings for duplicate parameters.
  \\\hline
 
 \textbf{Xyce Project Backlog/609}: 
  Global parameters that are applied to subcircuit instance parameters don't work in parallel & 
  If a subcircuit instance parameter (on the \texttt{X} line) depended on a
  \texttt{.param} or \texttt{.global\_param}, and that parameter was a
  variable, this dependency worked in serial, but not parallel.  In this case,
  ``variable'' means parameters that are allowed to change as part of a
  \texttt{.STEP} sweep, or as part of a sampling method.  This was related to
  the order of operations in the parser.  This has been fixed.  \\ \hline

 \textbf{Xyce Project Backlog/631}: 
  Expression-based local variation doesn't always work with subcircuit arguments & 
  When expression-based local variation was applied to a subcircuit argument, it 
  was handled incorrectly.  This meant that subcircuit arguments that should have 
  had a random distribution were fixed to the mean value during sampling.  
  This has been fixed.  \\ \hline

  \textbf{Xyce Project Backlog/637}: Support shared library build of \Xyce{} under Windows.
  &  Modify the cmake build system to support building a shared library version
  of xyce, \texttt{xyce-share.dll}, so that Xyce functionality can be accessed
  as a library from other applications.  \\\hline


\end{longtable}
}


\newpage
\section{Supported Platforms}
\subsection*{Certified Support}
The following platforms have been subject to certification testing for the
\Xyce{} version \XyceVersionVar{} release.
\begin{XyceItemize}
  \item Red Hat Enterprise Linux${}^{\mbox{\textregistered}}$ 8, x86-64 (serial and parallel)
  \item Microsoft Windows 11${}^{\mbox{\textregistered}}$, x86-64 (serial)
  \item Apple${}^{\mbox{\textregistered}}$ macOS, x86-64 (serial and parallel)
\end{XyceItemize}


\subsection*{Build Support}
Though not certified platforms, \Xyce{} has been known to run on the following
systems.
\begin{XyceItemize}
  \item FreeBSD 12.X on Intel x86-64 and AMD64 architectures (serial
    and parallel)
  \item Distributions of Linux other than Red Hat Enterprise Linux 6
  \item Microsoft Windows under Cygwin and MinGW
  \item Apple${}^{\mbox{\textregistered}}$ macOS, arm64, M-series CPUs (serial)
\end{XyceItemize}


\section{\Xyce{} Release \XyceVersionVar{} Documentation}
The following \Xyce{} documentation is available on the \Xyce{} website in pdf
form.
\begin{XyceItemize}
  \item \Xyce{} Version \XyceVersionVar{} Release Notes (this document)
  \item \Xyce{} Users' Guide, Version \XyceVersionVar{}
  \item \Xyce{} Reference Guide, Version \XyceVersionVar{}
  \item \Xyce{} Mathematical Formulation
  \item Power Grid Modeling with \Xyce{}
  \item Application Note: Coupled Simulation with the \Xyce{} General
    External Interface
  \item Application Note: Mixed Signal Simulation with \Xyce{} 7.2
\end{XyceItemize}
Also included at the \Xyce{} website as web pages are the following.
\begin{XyceItemize}
  \item Frequently Asked Questions
  \item Building Guide (instructions for building \Xyce{} from the source code)
  \item Running the \Xyce{} Regression Test Suite
  \item \Xyce{}/ADMS Users' Guide
  \item Tutorial:  Adding a new compact model to \Xyce{}
\end{XyceItemize}


\section{External User Resources}
\begin{itemize}
  \item Website: {\color{XyceDeepRed}\url{http://xyce.sandia.gov}}
  \item Google Groups discussion forum:
    {\color{XyceDeepRed}\url{https://groups.google.com/forum/#!forum/xyce-users}}
  \item Email support:
    {\color{XyceDeepRed}\href{mailto:xyce@sandia.gov}{xyce@sandia.gov}}
  \item Address:
    \begin{quote}
            Electrical Models and Simulation Dept.\\
            Sandia National Laboratories\\
            P.O. Box 5800, M.S. 1177\\
            Albuquerque, NM 87185-1177 \\
    \end{quote}
\end{itemize}

\vspace*{\fill}
\noindent
Sandia National Laboratories is a multimission laboratory managed and
operated by National Technology and Engineering Solutions of Sandia,
LLC, a wholly owned subsidiary of Honeywell International, Inc., for
the U.S. Department of Energy's National Nuclear Security
Administration under contract DE-NA0003525.

\end{document}

