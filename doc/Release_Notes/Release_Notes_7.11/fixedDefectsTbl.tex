% Sandia National Laboratories is a multimission laboratory managed and
% operated by National Technology & Engineering Solutions of Sandia, LLC, a
% wholly owned subsidiary of Honeywell International Inc., for the U.S.
% Department of Energy's National Nuclear Security Administration under
% contract DE-NA0003525.

% Copyright 2002-2025 National Technology & Engineering Solutions of Sandia,
% LLC (NTESS).

%%
%% Fixed Defects.
%%
{
\small

\begin{longtable}[h] {>{\raggedright\small}m{2in}|>{\raggedright\let\\\tabularnewline\small}m{3.5in}}
    \caption{Fixed Defects.  The Xyce team has multiple issue
     trackers, and the table below indicates fixed issues by
     indentifying both the tracker and the issue number.  Further,
     some issues are reported by open source users on GitHub and these
     issues may be tracked using multiple issue numbers.} \\ \hline
     \rowcolor{XyceDarkBlue} \color{white}\textbf{Defect} & \color{white}\textbf{Description} \\ \hline
     \endfirsthead
     \caption[]{Fixed Defects.  Note that we have two multiple issue tracking systems for Sandia Users.
     SON and SRN refer to our legacy open- and restricted-network Bugzilla system, and Gitlab refers to issues in our gitlab repositories.  } \\ \hline
     \rowcolor{XyceDarkBlue} \color{white}\textbf{Defect} & \color{white}\textbf{Description} \\ \hline
     \endhead

  \textbf{Xyce Project Backlog/888}: Xyce raw files (if specified via the command line "-r" argument) can incorrectly include adjoint sentivity text
  &  There was a logic error in the output manager of \Xyce{}.  This has been fixed.
  \\\hline

  \textbf{Xyce Project Backlog/849}: Sensitivities don't work with PWL source & Analytical sensitivity functions were not implemented for independent sources.  Also PWL sources had unique parsing issues that prevented any kind of derivative from working.  These issues have been addressed.
  \\\hline

  \textbf{Xyce Project Backlog/979}: Fix incorrect output of FD ASCII rawfile in sweeps & There was a logic error in the output manager of \Xyce{} that overwrites the number of points in a Raw ASCII file for AC analysis.  This has been fixed.
  \\\hline
  
  \textbf{Xyce Project Backlog/988}: Fix the capacitance issue in the BSIM SOI 4.6.1 &  There was an error in one of the capacitance models of the BSIM SOI version 4.6.1, resulting in a capacitance that was too large.  This was not fixed in the official version of the 4.6.1 from the BSIM group, but was fixed in the 4.6.1 implementation of commercial simulators.  As such, we chose to fix it in our own implementation. 
  \\\hline

\textbf{Xyce Project Backlog/xxx}: Description
  &  Details
  \\\hline


\end{longtable}
}
