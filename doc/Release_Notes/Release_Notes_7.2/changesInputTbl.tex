% Sandia National Laboratories is a multimission laboratory managed and
% operated by National Technology & Engineering Solutions of Sandia, LLC, a
% wholly owned subsidiary of Honeywell International Inc., for the U.S.
% Department of Energy’s National Nuclear Security Administration under
% contract DE-NA0003525.

% Copyright 2002-2025 National Technology & Engineering Solutions of Sandia,
% LLC (NTESS).

% Sandia National Laboratories is a multimission laboratory managed and
% operated by National Technology & Engineering Solutions of Sandia, LLC, a
% wholly owned subsidiary of Honeywell International Inc., for the U.S.
% Department of Energy’s National Nuclear Security Administration under
% contract DE-NA0003525.

% Copyright 2002-2025 National Technology & Engineering Solutions of Sandia,
% LLC (NTESS).


%%
%% Changes to Xyce input since the last release.
%%
{
\small

\begin{longtable}[h] {>{\raggedright\small}m{2in}|>{\raggedright\let\\\tabularnewline\small}m{3.5in}}
  \caption{Changes to netlist specification since the last release.\label{newUsage}} \\ \hline
  \rowcolor{XyceDarkBlue}
  \color{white}\textbf{Change} &
  \color{white}\textbf{Detail} \\ \hline \endfirsthead
  \caption[]{Changes to netlist specification since the last release.\label{newUsage}} \\ \hline
  \rowcolor{XyceDarkBlue}
  \color{white}\textbf{Change} &
  \color{white}\textbf{Detail} \\ \hline \endhead

The precedence of FROM and TO versus AT was made identical for the DERIV
and FIND measures. & The \texttt{FROM} and \texttt{TO} qualifiers now both
take precedence over the \texttt{AT} qualifier for both \texttt{DERIV}
and \texttt{FIND} measures.  This behavior matches other simulators.
\\ \hline

The RISE, FALL and CROSS qualifiers are no longer supported for TRAN-mode
AVG, INTEG and RMS measures. & This changes makes the qualifier support
for those three measure types consistent for all four measure modes
(AC, DC, NOISE and TRAN).  This is also consistent with HSPICE.
\\ \hline

The Xyce-specific RFC\_LEVEL qualifier is no longer supported for the
DERIV-WHEN, FIND-WHEN and WHEN measures & That qualifier was added to
improve the performance of the \texttt{RISE}, \texttt{FALL} and
\texttt{CROSS} qualifiers for \texttt{MAX}, \texttt{MIN} and \texttt{PP}
measures, and the performance of the \Xyce{}-specific \texttt{FRAC\_MAX}
qualifier for \texttt{TRIG} and \texttt{TARG} measures.  It was not
particularly useful for \texttt{WHEN} measures.
\\ \hline

The Points Value parameter on .AC and .NOISE lines must now be an integer
greater than or equal to 1 & Previously, the Point Value could be
non-integer on \texttt{.AC} and \texttt{.NOISE} lines, but had to be
non-negative.  This change affects \texttt{DEC} and \texttt{OCT} sweeps,
and prevents \Xyce{} from picking a non-intuitive set of sweep values.
\\ \hline

The Points parameter on .DC and .STEP lines must now be an integer
& Previously, the Point parameter had no constraints for these
two types of netlist commands.  This change affects \texttt{DEC} and
\texttt{OCT} sweeps, and prevents \Xyce{} from picking a non-intuitive
set of sweep values.
\\ \hline

\end{longtable}
}
