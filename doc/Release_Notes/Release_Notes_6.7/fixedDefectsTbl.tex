% Sandia National Laboratories is a multimission laboratory managed and
% operated by National Technology & Engineering Solutions of Sandia, LLC, a
% wholly owned subsidiary of Honeywell International Inc., for the U.S.
% Department of Energy’s National Nuclear Security Administration under
% contract DE-NA0003525.

% Copyright 2002-2019 National Technology & Engineering Solutions of Sandia,
% LLC (NTESS).



%%
%% Fixed Defects.
%%
{
\small

\begin{longtable}[h] {>{\raggedright\small}m{2in}|>{\raggedright\let\\\tabularnewline\small}m{3.5in}}
     \caption{Fixed Defects.  Note that we have two different Bugzilla systems for Sandia users.
     SON, which is on the open network, and SRN, which is on the restricted network. } \\ \hline
     \rowcolor{XyceDarkBlue} \color{white}\bf Defect & \color{white}\bf Description \\ \hline
     \endfirsthead
     \caption[]{Fixed Defects.  Note that we have two different Bugzilla systems for Sandia Users.
     SON, which is on the open network, and SRN, which is on the restricted network. } \\ \hline
     \rowcolor{XyceDarkBlue} \color{white}\bf Defect & \color{white}\bf Description \\ \hline
     \endhead

     %\textbf{bug xxxx }: Describe issue & Describe fix.  \\ \hline
     \textbf{877-SON} : Provide capability to force a seed on expression library
     random number generator &
     \Xyce{}'s expression library has three random number generation functions,
     \texttt{RAND}, \texttt{GAUSS}, and \texttt{AGAUSS}.  In previous releases,
     \Xyce{} would always select a seed for the random number generator that
     was based on the current time of day.  This remains the default behavior,
     but as of this release, the user may also specify a specific seed to be
     used by using the ``-randseed'' command line option.  ``-randseed'' takes
     a single argument, the seed to be used.  Additionally, \Xyce{} will output
     text indicating the seed being used whether ``-randseed'' is specified or
     not. \\ \hline

     \textbf{864-SON }: Xyce hangs when resolving parameter  &
     An error in the expression library led to \Xyce{} entering an infinite
     loop when a parameter (\texttt{.PARAM}) was defined with an expression
     involving comparison of an undefined parameter with some value using the
     not-equal ($!=$) operator.  After the fix for the bug, \Xyce{} will emit
     an error about being unable to resolve the parameter being defined in the
     \texttt{.PARAM} statement. \\ \hline

     \textbf{838-SON}: Shared library plugins do not work in parallel &
     While binaries of \Xyce{} are not distributed supporting this capability,
     is possible to build \Xyce{} from source using options to allow it to load
     device models as shared library plugins.  It was discovered for Xyce 6.6
     that this capability did not work correctly in parallel.  The loading
     error was identified and fixed, and parallel runs of \Xyce{} can now load
     shared library device plugins.  If both serial and parallel builds of
     \Xyce{} were created with the same base compiler, a shared library created
     for the serial build can be loaded by the parallel build. \\ \hline

    \textbf{833-SON}: Incorrect error message when requesting power for unsupported devices 
    (K, O, U and some Y devices) or non-existent device names &  Netlist parsing will fail 
    and produce an error message if power (\texttt{P()} or \texttt{W()}) is requested either
    for  a device (e.g., \texttt{K1}) that does not support power or for a device (e.g., 
    \texttt{RBOGO}) that does not exist in the netlist.  For a device like \texttt{K1}, those 
    error messages will now reference \texttt{P(K1)} rather than \texttt{I(K1)}. For a device 
    like \texttt{RBOGO}, the error message is now clearer that the issue is with the 
    \texttt{RBOGO} instance rather than with with power calculations for the \texttt{R} 
    device. \\ \hline     

     \textbf{839-SON}: Add warning/error message (about using default NORM) to ERROR 
     measure & Xyce 6.6 would silently default to using the L2NORM if an \texttt{ERROR}
     measure either did not specify a value for the \texttt{COMP\_FUNCTION} qualifier,
     or specified an invalid value. Xyce 6.7 explicitly says what \texttt{COMP\_FUNCTION} 
     value (\texttt{INFORM}, \texttt{L1NORM} or \texttt{L2NORM}) was used in the 
     descriptive output for each \texttt{ERROR} measure.  \\ \hline

     \textbf{845-SON}: Make the lossless transmission line device work correctly 
     with .STEP & \texttt{.STEP} now works correctly with the \texttt{TD}, \texttt{NL}, 
     \texttt{F}, and \texttt{Z0} instance parameters for the Lossless Transmission 
     Line device. \\ \hline 

     \textbf{848-SON}: Make the power grid devices work correctly with .STEP & 
     \texttt{.STEP} now works correctly with the appropriate instance parameters for 
     power grid devices. Consult the \Xyce{} Reference Guide for more details. \\ \hline 

     \textbf{859-SON}: Make the thermal resistor work correctly with .STEP & 
     \texttt{.STEP} now works correctly with a thermal resistor device (level
     2 resistor) that only specifies \texttt{L}, and not \texttt{A}. \\ \hline

     \textbf{874-SON}: Incorrect lead current and power values are output, when 
     the \texttt{IC} instance parameter is used for a capacitor & If the \texttt{IC}
     instance parameter, which set the initial voltage drop across the capacitor,
     was used for a capacitor \texttt{(C1)}, then the lead current \texttt{I(C1)}
     would only be correct for t=0.  It would always be reported as zero for t$>$0.  
     The power \texttt{P(C1)} would always be reported as 0 at t=0.  These errors
     have been fixed.  \\ \hline

     \textbf{856-SON}: Fix memory bloat in transient adjoint sensitivities & 
     Transient adjoint sensitivities previously used a dense storage scheme for
     information saved during the forward solve.  This has been fixed so that
     now it uses sparse storage.  As a result, the transient adjoint
     sensitivities  capability is much more practical.  \\ \hline

     \textbf{862-SON}: Update distribution scripts to bundle all three versions
     of libmkl\_avx* for OS X builds & 
     Mac machines with different processors require different versions of the
     blas functions. Now the OS X installs should work on all machines \\
     \hline

     \textbf{2044-SRN}: chargeSimpleMPDE.cir fails in parallel build serial run mode &
     This is due to a failure of the default iterative linear solver to converge in the MPDE solve.
     However, the MPDE analysis was not acknowledging the linear solver options line
     so that the solver could be changed.  This has been fixed. \\ \hline

\end{longtable}
}
