% Sandia National Laboratories is a multimission laboratory managed and
% operated by National Technology & Engineering Solutions of Sandia, LLC, a
% wholly owned subsidiary of Honeywell International Inc., for the U.S.
% Department of Energy’s National Nuclear Security Administration under
% contract DE-NA0003525.

% Copyright 2002-2019 National Technology & Engineering Solutions of Sandia,
% LLC (NTESS).



%%
%% Fixed Defects.
%%
{
\small

\begin{longtable}[h] {>{\raggedright\small}m{2in}|>{\raggedright\let\\\tabularnewline\small}m{3.5in}}
     \caption{Fixed Defects.  Note that we have two different Bugzilla systems for Sandia users.
     SON, which is on the open network, and SRN, which is on the restricted network. } \\ \hline
     \rowcolor{XyceDarkBlue} \color{white}\bf Defect & \color{white}\bf Description \\ \hline
     \endfirsthead
     \caption[]{Fixed Defects.  Note that we have two different Bugzilla systems for Sandia Users.
     SON, which is on the open network, and SRN, which is on the restricted network. } \\ \hline
     \rowcolor{XyceDarkBlue} \color{white}\bf Defect & \color{white}\bf Description \\ \hline
     \endhead

     \textbf{1040-SON}: Capacitor initial conditions incorrectly
     applied with NOOP & \Xyce{} was incorrectly applying initial
     conditions from the \texttt{IC} parameter of capacitors in the
     specific case of transient runs with \texttt{NOOP}
     or \texttt{UIC} keywords on the \texttt{.TRAN} line.  This has
     been fixed, and now these initial conditions are applied in a
     manner that produces identical results to the other methods of
     specifiying initial conditions.

     \textbf{Note:} This bug did not apply to initial conditions
     specified on \texttt{.IC} lines, or initial conditions applied
     when an operating point was calculated (i.e.,
     without \texttt{NOOP} or \texttt{UIC} on the \texttt{.TRAN}
     line). Those other use cases have always been correct.\\ \hline

     \textbf{1038-SON}: \Xyce{} will not run a netlist with a .LIB statement
     that includes a subcircuit &
     \Xyce{} rigorously supports the HSPICE .LIB syntax.  In correcting 
     this behavior for \Xyce{} version 6.9, a small logic error was introduced
     that could cause issues with parsing library definitions when subcircuit 
     definitions were inside.  This has been corrected. \\ \hline

     \textbf{1033-SON}: Issues with use of reserved words TEMP and VT
     in expressions in device instance parameters and .GLOBAL\_PARAM
     statements & If the reserved words \texttt{TEMP} or \texttt{VT}
     were used in expressions for device instance parameters or
     in \texttt{.GLOBAL\_PARAM} statements then their values would
     always be 0.  This is fixed now.  \\ \hline

     \textbf{1030-SON}: The N() operator does not work in expressions &
     The syntax \texttt{.PRINT TRAN {N(A)}} where A was a valid
     node name in the netlist did not work.  It would return a 
     parsing error about ``Function or variable N() is not defined''.  
     This is fixed now.  \\ \hline

     \textbf{1026-SON}: Fix issues with ordering of .HB (or .NOISE) 
     and .OP statements in a netlist &  If the \texttt{.OP} statement 
     preceded the \texttt{.HB} statement (or \texttt{.NOISE} statement) 
     in the netlist then no \texttt{.OP} output would be made.  Instead, 
     an error message about the analysis and print types being 
     inconsistent would be produced, if the netlist also had a 
     \texttt{.PRINT HB} (or \texttt{.PRINT NOISE}) line in it. This is 
     fixed now.   \\ \hline

     \textbf{1021-SON}: Segfaults from improperly formatted .SENS
     lines and improper handling of vector parameters on .OPTIONS
     lines &  Extra or missing commas on \texttt{.SENS} or 
     \texttt{.OPTIONS} lines could cause \Xyce{}  to segfault during 
     netlist parsing.  Also, a missing or misformatted \texttt{PARAM} 
     variable on a \texttt{.SENS} line could also cause a segfault 
     during netlist parsing.  This is fixed now. \\ \hline

     \textbf{1020-SON}: F and H sources incorrectly handle when gain
     is expression & \Xyce{} always converts F and H sources
     (current-controlled current source and current-controled voltage
     source, respectively) into equivalent B sources when the netlist
     is parsed.  It was recently discovered that the code would
     produce an invalid expression for the B source if the user had
     specified the gain (or transresistance) of the device as a
     brace-delimited expression, and \Xyce{} would exit with a syntax
     error that inadequately explained the problem.  The conversion of
     F and H sources into B sources has been fixed and now \Xyce{}
     will perform correctly even if the user specifies the gain as an
     expression.\\ \hline

     \textbf{1012-SON}: Incorrect values for failed EQN Measures when 
     used with .STEP & The values for a failed EQN measure in the 
     \texttt{<netlistname>.mtx} or \texttt{<netlistname>.msx} files 
     were incorrectly reported as 0 rather than as the ``default 
     value'' (\texttt{DEFAULT\_VAL}) if \texttt{.STEP} was used in the 
     netlist.  This is fixed now.  \\ \hline

     \textbf{1011-SON}: Fix behaviors of .PRINT HB\_FD and .PRINT HB\_TD 
     lines &  In previous \Xyce{} versions, no harmonic balance output 
     would be produced if the netlist only had a \texttt{.PRINT HB\_FD} 
     line or a \texttt{.PRINT HB\_TD} line, but not both.  In addition, 
     it was not possible to request different print formats on the 
     \texttt{.PRINT HB\_FD} and \texttt{.PRINT HB\_TD} lines.  Both of 
     these issues are fixed now. \\ \hline

     \textbf{1000-SON}: Fix the measurement window information in the 
     stdout for a failed DC measure &  The descriptive output (to 
     stdout) for a failed DC measure was incomplete.  For that case, 
     it now contains the start and end values of the first variable 
     in the DC Sweep vector, since those values likely did not overlap 
     with the FROM-TO window for that DC measure. For a netlist with a 
     .OP statement but no .DC statement, no measurement window information 
     will be printed to stdout now since there is no DC sweep vector in 
     that case.\\ \hline

     \textbf{958-SON}: Support voltage operators (e.g., V()) for .PRINT 
     NOISE & This line (\texttt {.PRINT NOISE  INOISE ONOISE V(4)}) would 
     print out ``all zeroes'' for \texttt{V(4)} in the output file, rather 
     than the complex values from the AC solution.  This is fixed now.  
     \\ \hline

     \textbf{966-SON}: Improve compatibility of -r command line option 
     with multifile output &  In previous \Xyce{} versions, no output
     was produced for \texttt{.PRINT SENS} or \texttt{.PRINT HOMOTOPY}
     lines when the -r command line option was used to get RAW file 
     output.  This was changed so that the output from any 
     \texttt{.PRINT SENS} or \texttt{.PRINT HOMOTOPY} lines is now 
     generated, as specified by those two print lines, when -r is used.
     The new behavior is documented in the ``Output File Redirection'' section 
     of the \Xyce{} Users' Guide.
     \\ \hline

     \textbf{941-SON}: Extra columns in .hb\_ic.prn and .startup.prn 
     files when separate .PRINT HB\_FD and .PRINT HB\_TD lines were 
     used in a netlist &  For previous versions of \Xyce{}, the columns 
     in those two output files would be the combination of the columns 
     requested by the \texttt{.PRINT HB\_FD} and \texttt{.PRINT HB\_TD} 
     lines.  This is fixed now in the sense that a \texttt{.PRINT HB\_FD} 
     line now only produces the \texttt{<netlistname>.HB.FD.prn} file and 
     a \texttt{.PRINT HB\_TD} line now only produces the 
     \texttt{<netlistname>.HB.TD.prn} file.  If the 
     \texttt{<netlistname>.hb\_ic.prn} and \texttt{<netlistname>.startup.prn} 
     files are needed then either a \texttt{.PRINT HB} line or explicit 
     \texttt{.PRINT HB\_IC} and \texttt{.PRINT HB\_STARTUP} lines should 
     be used. \\ \hline

     \textbf{911-SON}: Improve compatibility of multi-file output with 
     the -o command line option & If \Xyce{} was invoked with the -o command line
     option then the output could be ``interleaved'' in one file.  For example, 
     if the netlist \texttt{example.cir} had these two \texttt{.PRINT} lines: 
     {\tt
     \begin{verbatim}
     .PRINT TRAN V(1)
     .PRINT HOMOTOPY V(1)
     \end{verbatim}
     }
     and was invoked with \texttt{Xyce -o output.text example.cir} then the
     output that would normally appear in separate \texttt{example.cir.prn}
     and \texttt{example.cir.HOMOTOPY.prn} files would be jumbled together
     in the single file \texttt{output.txt}.  

     This was fixed by simplifying the behavior of the -o command line
     option.  See the new ``Output File Redirection'' section of the
     \Xyce{} Users' Guide for more details. \\ \hline

     \textbf{718-SON}: Missing error message for invalid nodes in expressions 
     on .PRINT lines & If an invalid node is specified on a \Xyce{} 
     \texttt{.PRINT TRAN} line then \Xyce{} returns a fatal error
     during netlist parsing (e.g., \texttt{.PRINT TRAN V(BOGONODE)}
     will produce an error message of \texttt{undefined symbol in .PRINT
     command: node BOGONODE}, if \texttt{BOGONODE} does not exist in the
     netlist).  However, if the invalid node was inside a \Xyce{} expression
     (e.g., \texttt{.PRINT TRAN \{V(BOGONODE)\}}) then \Xyce{} did not
     produce an error message during netlist parsing and the output value
     for \texttt{\{V(BOGONODE)\}} would be zero for all time-steps.  This
     is fixed now.  For the expression case, \Xyce{} netlist parsing will 
     now emit an error message like \texttt{Function or variable V(BOGONODE) 
     is not defined}.  \\ \hline

     \textbf{707-SON}: Fix behavior for invalid nodes on .FOUR lines and in 
     .MEASURE statements & There were issues with \texttt{.FOUR} lines and
     \texttt{.MEASURE} statements that accidentally used node names that
     were not in the netlist.  In that case, the \texttt{.cir.four} output
     file would contain a mix of all zero's and NaN's, and \Xyce{} would 
     not produce a warning or error message about the invalid node name.
     Similarly, the measure statement would run without a warning message
     about the invalid node name.  The measure result would then be zero,
     rather than FAILED.  \Xyce{} parsing will now issue an error message
     instead.  An example is ``Function or variable V(A) is not defined'' 
     when V(A) refers to an invalid node name on \texttt{.FOUR} or 
     \texttt{.MEASURE} lines. \\ \hline
     
     \textbf{274-SON}: String valued parameters not working as expected &
     String values are interpreted as a filename and then parsed into a 
     table expression.  This limits the ability of \Xyce{} to take pure
     string values and assign them to parameters.  For this release
     \Xyce{} will issue a warming when it encounters the syntax 
     \texttt{PARAMETER="filename"} and recommend using the new syntax
     \texttt{PARAMETER=tablefile("filename")}.  In a future \Xyce{} release
     the old syntax, \texttt{PARAMETER="filename"}, will only attempt to assign
     the string to the parameter and not try to automatically read it in
     as a data file.  Thus, the old syntax for specifying a filename from 
     which to read data will not work.  So, it is recommended that users use this
     release to transition to the new \texttt{PARAMETER=tablefile("filename")}
     syntax.  Additionally, this release introduces 
     \texttt{PARAMETER=string("string value")} to specify parameters that 
     are just strings.  In a future release, the \texttt{string()} keyword will
     be deprecated and the syntax for specifying string valued parameters 
     will become \texttt{PARAMETER="string value"}.\\ \hline 


     \textbf{1063-SON}: The device initial states are not set correctly when
     the tahb option is turned off in HB analysis  & This bug could have
     affected the convergence of HB for the circuits with certain devices, such as BJTs.
     This is fixed now. \\ \hline

     \textbf{1060-SON}: The startupperiods option and .IC do not work together for 
     HB analysis  & The \texttt{startupperiods} option and \texttt{.IC} can help find a good initial
     guess for HB analysis. When they were used together, the initial condition found  
     by startup transient analysis was overwritten by the IC specified in the \texttt{.IC.}
     This is fixed now. \\ \hline

     \textbf{1962-SRN}: Voltages from interface nodes for subcircuits may not 
     work correctly in expressions on \texttt{.PRINT}, \texttt{.MEASURE}, 
     \texttt{.FOUR}, \texttt{.SENS}, \texttt{.RESULT}, \texttt{.IC} and 
     \texttt{.NODESET} lines & An expression that used a voltage from an 
     interface node to a subcircuit on a \texttt{.PRINT} line would only work 
     if that voltage node was also used outside of the expression on the 
     \texttt{.PRINT} line.  A simple example was as follows.  The expression 
     \texttt{\{V(X1:a)*I(X1:R1)\}} would print out as 0, unless 
     \texttt{V(X1:a)} was also on the \texttt{.PRINT} line.  Similar issues
     also existed for \texttt{.MEASURE}, \texttt{.FOUR}, \texttt{.SENS}, 
     \texttt{.RESULT}, \texttt{.IC} and \texttt{.NODESET} lines.  This bug
     has been fixed for all of those line types now.
     \\ \hline

     \textbf{1070-SON}: Parameters specified with curly braces confuse \Xyce{} sensitivity analysis  &
     If the parameters specified for sensitivity analysis were surrounded by curly braces (as one
     might do if they were global parameters, by convention), sensitivity analysis couldn't find
     the parameter properly, but didn't exit with an error.  Instead it computed an incorrect answer.
     This is fixed now. \\ \hline

     \textbf{1073-SON}: Sensitivities which rely on finite differences are badly broken in transient &
     There was an error in how numerical derivatives were computed that caused them to be incorrect 
     for transient direct sensitivities.  This did not affect steady-state (DC) sensitivities, and it 
     also did not affect sensitivities that relied entirely on analytical derivatives.
     This is fixed now. \\ \hline

     \textbf{1077-SON}: Transient sensitivity analysis seg faults if used with .STEP or .SAMPLING &
     There was a mistake in how the nonlinear solver was reset between step iterations.
     This is fixed now. \\ \hline

\end{longtable}
}
