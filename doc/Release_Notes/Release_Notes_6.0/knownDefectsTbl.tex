% Sandia National Laboratories is a multimission laboratory managed and
% operated by National Technology & Engineering Solutions of Sandia, LLC, a
% wholly owned subsidiary of Honeywell International Inc., for the U.S.
% Department of Energy’s National Nuclear Security Administration under
% contract DE-NA0003525.

% Copyright 2002-2019 National Technology & Engineering Solutions of Sandia,
% LLC (NTESS).


%%
%% Known defects and workarounds.
%%
\small

\begin{longtable}[h,t,b,p] {|
>{\setlength{\hsize}{0.40\hsize}}Y|
>{\setlength{\hsize}{0.60\hsize}}Y|} 

  \caption{Known Defects and Workarounds.} \\ \hline

\rowcolor{XyceDarkBlue}
\color{white}\bf Defect &
\color{white}\bf Description
\\ \hline
\endhead

{\bf 855-SRN}: AC and HB analysis do not work in parallel & In this
version of \Xyce{}, analysis types that require use of block linear
algebra classes in Trilinos/Epetra do not work in parallel.  The analysis
types that use block linear algebra classes are AC and HB.

{\bf\it Workaround:} There is no workaround other than to run all such
problems in serial. \\ \hline
   
{\bf 37-SON}: Connectivity checking is broken for devices with more
than 10 leads & The diagnostic code used by the \Xyce{} setup that
checks circuit topology for basic errors such as a node having no DC
path to ground or a node being connected to only one device has a bug
in it that causes the code to emit a cryptic error message, after
which the code will exit.  This error has so far only been seen when a
user has attempted to connect a large number of inductors together
using multiple mutual inductor lines.  The maximum number of
non-ground leads that can be used without confusing this piece of code
is 10.  If you have such a large mutual inductor and the code exits
with an error message, this bug is the source of the problem.

The error message now includes a recommendation to use the workaround below.

{\bf\it Workaround:} Disable connectivity checking by adding the line

\begin{alltt}
.OPTIONS TOPOLOGY CHECK_CONNECTIVITY=0
\end{alltt}

to your netlist.  This will disable the check for the basic errors such
as floating nodes and improperly connected devices, but will allow the
netlist to run with a highly-connected mutual inductor. \\ \hline

\texttt{.DC} sweep output. & \texttt{.DC} sweep calculation does not
automatically output the sweep variable.  Only variables explicitly listed on the \texttt{.PRINT} line are output.

{\bf\it Workaround:} List output sweep variable(s) explicitly in the \texttt{.PRINT} statement.  \\ \hline

{\bf 772-SRN}: Infinite-slope transitions in B-sources causes "time step
too small" errors & The nonlinear dependent source (``B-source'') allows
the user to specify expressions that could have infinite-slope
transitions, such as

\begin{alltt}
Bcrtl OUTA 0 V=\{ IF( (V(IN) > 3.5), 5, 0 ) \} 
\end{alltt}

This can lead to ``timestep too small'' errors when \Xyce{} reaches
the transition point.  Infinite-slope transitions in expressions
dependent only on the \texttt{time} variable are a special case,
because \Xyce{} can detect that they are going to happen in the future
and set a ``breakpoint'' to capture them.  Infinite-slope transitions
depending on other solution variables cannot be predicted in advance,
and cause the time integrator to scale back the timestep repeatedly in
an attempt to capture the feature until the timestep is too small to
continue.

{\bf\it Workaround:} Do not use step-function or other infinite-slope
transitions dependent on variables other than \texttt{time}.  Smooth
the transition so that it is more easily integrated through. \\ \hline


\end{longtable}

