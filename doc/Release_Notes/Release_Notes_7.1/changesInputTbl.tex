% Sandia National Laboratories is a multimission laboratory managed and
% operated by National Technology & Engineering Solutions of Sandia, LLC, a
% wholly owned subsidiary of Honeywell International Inc., for the U.S.
% Department of Energy’s National Nuclear Security Administration under
% contract DE-NA0003525.

% Copyright 2002-2025 National Technology & Engineering Solutions of Sandia,
% LLC (NTESS).

% Sandia National Laboratories is a multimission laboratory managed and
% operated by National Technology & Engineering Solutions of Sandia, LLC, a
% wholly owned subsidiary of Honeywell International Inc., for the U.S.
% Department of Energy’s National Nuclear Security Administration under
% contract DE-NA0003525.

% Copyright 2002-2025 National Technology & Engineering Solutions of Sandia,
% LLC (NTESS).


%%
%% Changes to Xyce input since the last release.
%%
{
\small

\begin{longtable}[h] {>{\raggedright\small}m{2in}|>{\raggedright\let\\\tabularnewline\small}m{3.5in}}
  \caption{Changes to netlist specification since the last release.\label{newUsage}} \\ \hline
  \rowcolor{XyceDarkBlue}
  \color{white}\textbf{Change} &
  \color{white}\textbf{Detail} \\ \hline \endfirsthead
  \caption[]{Changes to netlist specification since the last release.\label{newUsage}} \\ \hline
  \rowcolor{XyceDarkBlue}
  \color{white}\textbf{Change} &
  \color{white}\textbf{Detail} \\ \hline \endhead

Mixed Signal Interface behavior when a ``backwards or null time step'' is requested.  &
Previously, for the Mixed Signal Interface, if the \texttt{requestedUntilTime}
in a \texttt{simulateUntil()} call was less than the current simulation time
then the \Xyce{} simulation would run to completion.  If the \texttt{requestedUntilTime}
was equal to the current simulation time then the simulation would get stuck in a
``time-step too small'' error loop.  The interface change is that both of these
cases are now simulation errors that will cause the \Xyce{} simulation to abort.
\\ \hline

Removed support for RISE, CROSS and FALL qualifiers for DUTY measure & This change
makes all of the duty-cycle calculation measures (\texttt{DUTY}, \texttt{FREQ},
\texttt{OFF\_TIME} and \texttt{ON\_TIME)} have similar behavior.  The
\texttt{FROM}, \texttt{TO} and \texttt{TD} qualifiers are still supported
for all four measure types.
\\ \hline

Handling of measures with duplicate names & In previous versions of \Xyce{},
measures with duplicate names would both be evaluated correctly.  However,
this could cause ambiguity if that name was also used as part of an
expression such as in an \texttt{EQN} measure.  An example is this
netlist fragment:
\begin{verbatim}
.MEASURE TRAN M1 MIN V(1)
.MEASURE TRAN M1 MAX V(1)
.MEASURE TRAN EQNM1 EQN M1
\end{verbatim}
To remove this ambiguity, \Xyce{} will now use the last \texttt{M1}
measure definition, which is the \texttt{MAX} measure in this example,
and issue a warning message about (and discard) any previous \texttt{M1}
definitions.  This change applies to all measure modes (\texttt{AC},
\texttt{DC} and \texttt{TRAN}).
\\ \hline

\end{longtable}
}
