% Sandia National Laboratories is a multimission laboratory managed and
% operated by National Technology & Engineering Solutions of Sandia, LLC, a
% wholly owned subsidiary of Honeywell International Inc., for the U.S.
% Department of Energy’s National Nuclear Security Administration under
% contract DE-NA0003525.

% Copyright 2002-2023 National Technology & Engineering Solutions of Sandia,
% LLC (NTESS).

% Sandia National Laboratories is a multimission laboratory managed and
% operated by National Technology & Engineering Solutions of Sandia, LLC, a
% wholly owned subsidiary of Honeywell International Inc., for the U.S.
% Department of Energy’s National Nuclear Security Administration under
% contract DE-NA0003525.

% Copyright 2002-2023 National Technology & Engineering Solutions of Sandia,
% LLC (NTESS).


%%
%% Fixed Defects.
%%
{
\small

\begin{longtable}[h] {>{\raggedright\small}m{2in}|>{\raggedright\let\\\tabularnewline\small}m{3.5in}}
     \caption{Fixed Defects.  Note that we have multiple issue
     tracking systems for Sandia users.  SON, which bugzilla on the
     open network, and SRN, which is bugzilla on the restricted
     network.  We are also transitioning from bugzilla to gitlab issue
     tracking.  Further, some issues are reported by open source users
     on GitHub and these issues may be tracked using multiple issue
     numbers.} \\ \hline
     \rowcolor{XyceDarkBlue} \color{white}\textbf{Defect} & \color{white}\textbf{Description} \\ \hline
     \endfirsthead
     \caption[]{Fixed Defects.  Note that we have two multiple issue tracking systems for Sandia Users.
     SON and SRN refer to our legacy open- and restricted-network Bugzilla system, and Gitlab refers to issues in our gitlab repositories.  } \\ \hline
     \rowcolor{XyceDarkBlue} \color{white}\textbf{Defect} & \color{white}\textbf{Description} \\ \hline
     \endhead
\textbf{Gitlab-ex issue 8}: Xyce ``Override'' rawfile output unconditionally replaces final ``\_'' in a variable name with ``\#'': &
A long-standing bug in Xyce's ``override'' rawfile output (that
produced when the command line option ``-r'' is used) caused all
variables with underscores in their names to have the last underscore
(``\_'') replaced with an octothorpe (``\#'').  This was originally done
to allow Xyce's rawfiles to be readable by IC-CAP, which expects the
branch current associated with a voltage source to have a ``\#branch''
suffix, whereas Xyce's internal name for these currents have a
``\_branch'' suffix.  This bug made rawfile output names of other
variables incorrect, so the code has been refined to make only
appropriate changes to variable names rather than blindly replacing
one character. \\ \hline

\textbf{Gitlab-ex issue 191}: Exclamation point character is not allowed in parameter names, in the new expression library &
This was supported in previous versions of \Xyce{} but was
inadvertantly broken with the new expression library for \Xyce{}
7.2. This mainly affected printing of ``Y'' device parameters on
the \texttt{.PRINT} line, but only inside of \texttt{.PRINT} line
expressions.  This was due to a minor parsing mistake and has been
addressed.  \\ \hline


\textbf{Gitlab-ex issue 157/Gitlab-ex issue 167/SON BUG 1190}: Parameters that are defined using \texttt{.param} must be constant and non-constant parameters must be defined using \texttt{.global\_param} &
\Xyce{} was originally designed with two keywords for setting
user-defined parameters in the nelist; \texttt{.param}
and \texttt{.global\_param}.  The keyword \texttt{.param} is also used
in many other simulators, but the behavior in \Xyce{} differed in
that \texttt{.param} parameters were unconditionally treated as
constants.  This was because they were handled as string substitutions
during parsing.  \Xyce{} parameters that needed to change during a
calculation had to be set using the other
keyword, \texttt{.global\_param}.

This design caused incompatibilities with other simulators, and proved
confusing to users.  However, it became possible to address this issue
with the advent of a new expression library in \Xyce{} 7.2 .

Starting in this release, all \texttt{.param} parameters in \Xyce{}
are allowed to change during a simulation. This means they are allowed
to depend on special variables such as \texttt{TIME} and are allowed
to depend on \texttt{.global\_param} parameters.
Also, \texttt{.param} parameters that are defined in the top level
circuit netlist are functionally identical to
\texttt{.global\_param} parameters, and can be swept using
analysis statements such as \texttt{.DC}, \texttt{.STEP}
and \texttt{.SAMPLING}.
  \\ \hline

\textbf{Gitlab-ex issue 150/Github issue 19}: Error Manager does not play well with repeatedly declared simulator API objects &
If a user code links to the Xyce libraries and allocates objects of
one of the Xyce simulator classes (either the base class
Xyce::Circuit::Simulator class or any of its derived classes) more
than once, the error manager's use of static data to keep count of
errors caused any failing simulation to cause all subsequent
simulations to fail.  Now, each simulator object clears the error
manager's error counts upon construction.

This fix does not eliminate problems related to the use of static data
in the error handling package.  Specifically, if multiple simulator
objects were run simultaneously in different threads then any fatal
error in any object's simulation would cause all to fail.  At this
time only sequential allocation and operation of simulator objects in
user code is supported. \\ \hline

\textbf{Gitlab-ex issue 153/Github issue 21}: External Output Interface does not prepend TIME &
The documentation of the external output interface in an application
note claimed that if a user created an external output object that did
not include output of ``TIME'' for a transient run or ``FREQ'' for a
frequency-domain run, that the outputter would automatically prepend
that variable to the list of outputs actually delivered.  That was not
happening due to a simple coding error.  Now the interface performs
that operation as intended. \\ \hline

\textbf{Gitlab-ex issue 143}: Fix transmission line breakpoint/discontinuity detection &
When the transmission line was modified in \Xyce{} 7.2 to stop forcing
a maximum timestep equal to the transmission line delay, bugs in the
handling of discontinuity detection and dynamically inserted
breakpoints were exposed.  As a result, transmission lines with very
short time delays could cause Xyce to detect a discontinuity at one
end of the line, but fail to set a breakpoint to catch the
discontinuity at the other end before the time integration had already
advanced too far.  The result of this combination of bugs could lead
to strange interpolation artifacts near the times when breakpoints
should have occurred.  The bugs in the time integration package's
timestep selection and in the nonlinear solver have been addressed,
and circuits involving transmission lines with very short delays will
no longer miss breakpoints in this manner. \\ \hline

\textbf{Gitlab-ex issue 82}:  Make AC sensitivities work in parallel &
The initial implementation of AC parameter sensitivities did not work
in parallel, but this now works correctly.
\\ \hline

\textbf{Gitlab-ex issue 115}:  Fix AC sensitivities to be correct for nonlinear
device model parameters & The original implementation of AC
sensitivities had a math mistake that caused the calculation to be
incorrect when applied to parameters from nonlinear devices.  The math
error has been fixed and now nonlinear AC sensitivies work correctly.
\\ \hline

\textbf{Gitlab-ex issue 119}:  using \texttt{.sens} with \texttt{.four} can result in an indexing mistake &
It was observed that when using .sens with \texttt{.four}, it was
possible under some circumstances that the order of outputs got
mixed up.  In other words, some outputs in the \texttt{*.four} file
would be given the wrong label.  This was due to an indexing mistake
and has been fixed.
\\ \hline

\textbf{Gitlab-ex issue 123}:  POLY in the new expression library has problems
at higer orders ($>$2) & The new expression library had mistakes in
the POLY operator for higher-order polynomials, which have been
corrected.
\\ \hline

\textbf{Gitlab-ex issue 99}:  Fix build flags associated with Trilinos &
Xyce is able to leverage some features of Trilinos in a beta mode. Some of
these were not enabled in the CMake build system, or were not functioning
correctly. This issue was focused on making those work properly. Specifically,
it was targeted at the inclusion of Amesos2, Stokhos, ShyLU, the Amesos2 Basker
(a templated linear solver), and the ShyLU Basker (a multi-threaded linear
solver). It also made Belos a required Trilinos package.
\\ \hline

\textbf{Gitlab-ex issue 108}:  Fix various problems in the limit operator in
the new expression library & The ``limit'' function in the new
expression library had mistakes in both the derivative calculation and
also in the breakpointing function.  Both issues have been fixed.
\\ \hline

\textbf{Gitlab-ex issue 113}:  Make flex/bison a required capability &
Prior to the rewrite of the expression library, flex and bison were optional;
however, now they are required. The autotools and CMake build systems have now
been modified to reflect that. The ability to turn off the reaction parser
capability with a flag has been retained, however (with the default being
``on'').
\\ \hline

\textbf{Gitlab-ex issue 128}: Fix Xyce/ADMS handling of some while and for loops &
Xyce/ADMS was emitting improper code for ``for'' or ``while'' loops
that contained what ADMS flagged as ``nonlinear'' conditions.  This
turned out to be due to an error in the implicit rules of ADMS that
has been present for at least 10 years, in which this unusual type of
condition resulted in an inappropriate double call to an early
dependency-tracking template.  The result was that contents of the
code block controlled by the loop construct was in backwards order,
and all function precomputation was happening twice (due to the
presence of duplicate entries in the ``function'' data structure).

Xyce/ADMS uses a modified version of ADMS's implicit rules file
instead of the one that ships with ADMS, and this error has now been
fixed in Xyce's modified version.

No standard Verilog-A models make use of ``for'' or ``while'' loops
that would have tripped this bug, but such a usage was found in a
Verilog-A model inside a proprietary vendor PDK. \\ \hline

\textbf{Gitlab-ex issue 103}: Fix glitches in transmission line at breakpoints &
The lossless transmission line is an ideal behavioral device that
operates by keeping a record of the histories of the two ends of the
line, and using interpolation on that history to look up prior
behavior at a time in the past in order to determine the current
behavior.  This interpolation is generally done using three-point
quadratic Lagrange interpolation.  It has been determined that Xyce
has historically been producing small ``glitches'' at the corners of
pulsed signals because quadratic interpolation is not really
appropriate with piecewise linear signals at the corners where the
slope changes signficantly.  Xyce now detects when such
discontinuities in derivative occur and performs linear interpolation
instead.  This prevents the small overshoots that have been observed
in these cases in all prior versions of Xyce. \\ \hline

\textbf{Gitlab-ex issue 137}: Address robustness issues in parallel restarting (BUG\_456) &
Parallel regression testing illustrated failures in restarting.  The
parallel hang can be tracked down to the restoration of restart data
and bad logic for distributing the RestartNode objects. \\ \hline

\textbf{Gitlab-ex issue 149}: The BUG\_507\_SON test is failing on 4-processors in parallel &
This weekly regression test failure on 4-processors was due to recent
changes in the parallel assignment of shared solution nodes.  A
balanced assignment strategy was implemented to address the
issue. \\ \hline

\textbf{Gitlab-ex Issue 101, SON BUG 274}:  Remove deprecated parsing of quoted filename to table &
Double quoted strings were once parsed as filenames into a lookup
table.  That syntax has been deprecated for some time and has now been
removed. Use the \texttt{tablefile(``filename'')} syntax to read a
file into a table.  A double quoted string is now only interpreted as
a string.  Additionally, the syntax \texttt{string(``'')} has also
been removed as it was used to signify a string.
\\ \hline

\textbf{Gitlab-ex Issue 27}:  Clear out C++11 conditionals&
To enable improved performance, Xyce now requires a C++11 capable
compiler.  Therefore, all the code conditionals related to C++11
features have been removed from the code.
\\ \hline

\textbf{1085-SON}: Expression library mishandled \texttt{.FUNC}
definitions of functions that began with ``I'' and were two characters
long &
\Xyce{}'s expression library previously assumed that all terms of the form
``\texttt{Ix(<arguments>)}'' are lead current expressions, where ``x'' is
either a lead designator such as ``D'', ``G'' , or ''S'' for a MOSFET
or ``C'', ``B'', ``E'' for a BJT, or a digit indicating the pin number
of the device associated with the lead.  This assumption made it
impossible for users to define a function with a two-character name
starts with ``I''.
\\ \hline

\textbf{1203-SON}: Issues with temperature-dependent device
instance-parameters that were expression-based & The values for
temperature-dependent device instance parameters could be incorrect
if their expression used a global parameter. This could occur if the
default temperature of 27C was used.  It could also occur if the
temperature was set with a \texttt{.OPTIONS DEVICE} line or via
\texttt{.STEP}.  A simple example is:
\begin{verbatim}
.GLOBAL_PARAM GTEMP=37
R1 1 0 1  TC=0.1 temp={GTEMP}
\end{verbatim}
If the temperature was set with a \texttt{.OPTIONS DEVICE} line then
the use of a ``normal'' parameter in their expression could also give
an incorrect answer.  An example is:
\begin{verbatim}
.PARAM ptemp=10
R2 2 0 1  TC=0.1 temp={temp+ptemp}
\end{verbatim}
\\ \hline

\textbf{1239-SON}: Address failures when parallel regression testing uses up to 4 processors &
Parallel regression testing illustrated failures when used on more
than 2 processors . These failures were caused by test and code issues
and have been addressed for testing up to 4 processors.
\\ \hline

\textbf{1241-SON}: Expression library parsing bottleneck on large expressions &
It was previously determined that the expression library in Xyce could
be the source of a severe parsing bottleneck when expressions are
large and complex.  Expressions of this sort show up most often when
parsing large PDKs with complex use of the \texttt{.FUNC} feature, and
when using the ``tablefile'' feature to import a large file of
time/voltage pairs for use in a \texttt{B} source.  These issues were
addressed by the new expression library.
\\ \hline

\textbf{1246-SON}: Supernoding seg faults / fails to remove all nodes in parallel &
Parallel regression testing illustrated a failure to perform
supernoding when run on more than 2 processors.  This failure was
caused by a code issue and has been addressed.
\\ \hline

\textbf{1263-SON}: Oscillator HB seg faults in linear solver when run in parallel using 4 processors &
Parallel regression testing illustrated a failure in oscillator HB
algorithm due to the linear solver and builder, when run on 4 or more
processors.  This failure was caused by a code issue and has been
addressed.
\\ \hline

\textbf{1311-SON}: PowerGridTransformer tests sometimes hang in parallel on more than 2 processors &
Parallel regression testing illustrated failures of circuits that have
PowerGridTransformer models when run on more than 2 processors.  These
failures were caused by a code issue and has been addressed.
\\ \hline

\textbf{Gitlab-ex Issue 130}: The FFTW interface does not work properly for even number of samples  &
The results from .HB and .FFT analysis were wrong for the N/2 harmonic
when the number of samples N is even. The real part of the result was
incorrectly copied to the imaginary part for the N/2 harmonic.  This
led to the wrong results. This is fixed now.
\\ \hline


\textbf{Gitlab-ex Issue 159}: The max time step is ignored for the first step out of a break point &
The max time step can be specified by a user or come from some special
devices. Some of them are not applied to the first step out of a break
point. This is fixed now.
\\ \hline

\textbf{Gitlab-ex Issue 122}: The dynamically inserted break points from delay device were not handled correctly &
The devices that use history information can dynamacally insert break
points during the transient simulation. This was not handled correctly
and led to accuracy issues, discontinuous behavior, wrong results and
convergence issues.  This is fixed now.
\\ \hline


\end{longtable}
}
