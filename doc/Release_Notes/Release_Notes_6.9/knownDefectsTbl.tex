% Sandia National Laboratories is a multimission laboratory managed and
% operated by National Technology & Engineering Solutions of Sandia, LLC, a
% wholly owned subsidiary of Honeywell International Inc., for the U.S.
% Department of Energy’s National Nuclear Security Administration under
% contract DE-NA0003525.

% Copyright 2002-2019 National Technology & Engineering Solutions of Sandia,
% LLC (NTESS).



%%
%% Known defects and workarounds.
%%
%% This section should highlight significant defects that were not fixed in
%% the release

{
\small

\begin{longtable}[h] {>{\raggedright\small}m{2in}|>{\raggedright\let\\\tabularnewline\small}m{3.5in}}
  \caption{Known Defects and Workarounds.} \\ \hline
  \rowcolor{XyceDarkBlue} \color{white}\bf Defect & \color{white}\bf Description
  \\ \hline \endfirsthead
  \caption[]{Known Defects and Workarounds.} \\ \hline
  \rowcolor{XyceDarkBlue} \color{white}\bf Defect & \color{white}\bf Description
  \\ \hline \endhead
% EXAMPLE:
%\textbf{bug number-SRN}: bug title & Description of KNOWN BUG THAT HAS NOT BEEN
  %FIXED.

%\textbf{\textit{Workaround}}: Describe how to work around this bug.
%\\ \hline
%
%
  \textbf{1009-SON}:  Transient adjoint sensitivities don't work with \texttt{.STEP}
  & Transient adjoint sensitivities require backward integrations to be performed after the primary transient forward integration.  Doing this properly requires information to be stored during the forward solve, and for certain bookkeeping to be performed.  Currently, these extra operations to support transient adjoints are not properly set up for \texttt{.STEP} analysis.

\textbf{\textit{Workaround}}: None
\\ \hline

  \textbf{1006-SON}:   SDT (expression library time integration) derivatives are not supported, so SDT can't be used for sensitivity analysis objective functions &
  SDT is a function supported by the \Xyce{} expression library to compute numerical time integration.  When this function is used, the expression library does not produce correct derivatives.  This impacts Jacobian matrix entries, when SDT is used with a Bsrc, and it also impacts sensitivity analysis, when SDT is used in an objective function.  For the former case, this can result in a lack of robustness for circuits that contain SDT-Bsrc devices.  For the latter case, the objective function will simply be incorrect.

\textbf{\textit{Workaround}}: None
\\ \hline

\textbf{991-SON}: Non-physical BH Loops in non-linear mutual inductor &
Nonlinear mutual inductors that have high coupling coefficients (i.e. 
model parameter \texttt{ALPHA} over 1.0e-4) and low loss characteristics 
(i.e. zero \texttt{GAP}) can produce B-H loops with nonphysical hysteresis.

\textbf{\textit{Workaround}}: Lower \texttt{ALPHA} values or larger 
\texttt{GAP} values can ameliorate this issue, but the root cause is 
still under investigation. 
\\ \hline

\textbf{989-SON}: I(*) will not print branch currents that are part of a Y
device & Bug 715-SON (I(*) for subcircuit devices does not work
properly on .PRINT lines) was fixed for the \Xyce{} 6.9 release.  The
caveat is that I(*) still does not work for branch currents that are
part of a Y device.

\textbf{\textit{Workaround}}: Explicitly request those Y-device branch
currents on the \texttt{.PRINT} line.
\\ \hline

\textbf{971-SON}: Use of default device parameter syntax on a .PRINT line causes
\Xyce{} to print 0 for that parameter & This line (\texttt{.PRINT TRAN R1}) will
cause \Xyce{} to print 0 for the resistance value of R1.

\textbf{\textit{Workaround}}: Use \texttt{.PRINT TRAN R1:R} instead.
\\ \hline

\textbf{970-SON}: Some devices do not work in frequency-domain analysis &
Devices that may be expected to work in AC or HB analysis do not at
this time.  This includes, but is not limited to, the lossy
transmission line (LTRA) and lossless transmission line (TRA).

\textbf{\textit{Workaround}}: The LTRA and TRA models will need to be replaced
with lumped transmission line models (YTRANSLINE).
\\ \hline

\textbf{967-SON}: Zoltan segmentation fault with OpenMPI 2.1.x and 3.0.0 on
some systems & It has been observed that when \Xyce{} and Trilinos are
built with OpenMPI 2.1.x or 3.0.0 a small number of test cases in the
regression suite crash with a segmentation fault inside the Zoltan
library.  To date, these versions of OpenMPI versions have only been
tested on FreeBSD, and we do not know if this bug is present on other
systems.  \\ \hline

\textbf{966-SON}: Improve Compatibility of -r command line option with multifile
output & If the -r command line option is used to get RAW file output
then any \texttt{.PRINT SENS} or \texttt{.PRINT HOMOTOPY} lines in the
netlist will be silently ignored.

\textbf{\textit{Workaround}}: Re-run the netlist without the -r command line
option.
\\ \hline

\textbf{964-SON}: Compatibility of .PRINT TRANADJOINT with .STEP & The use of
\texttt{.PRINT TRANADJOINT} is not compatible with \texttt{.STEP}.  The
resultant \Xyce{} output will not be correct.

\textbf{\textit{Workaround}}: There is none.
\\ \hline

\textbf{941-SON}: Extra columns in .hb\_ic.prn and .startup.prn files when
separate .PRINT HB\_TD and .PRINT HB\_FD lines are used in a netlist &
The columns in those two output files will be the combination of the
columns requested by the
\texttt{.PRINT HB\_TD} and \texttt{.PRINT HB\_FD} lines.

\textbf{\textit{Workaround}}: Use explicit \texttt{.PRINT HB\_IC} and
\texttt{.PRINT HB\_STARTUP} lines instead, to get the correct versions of
those two files.
\\ \hline

\textbf{939-SON}: Invalid fields (XBEGIN, XEND and SUBTITLE) in \Xyce{}-generated
HOMOTOPY.csd files & The fields in the \#H header block of the
.HOMOTOPY.csd files are currently hard-coded to 0 and 1, respectively.
The \texttt{SUBTITLE} field is incorrect for \texttt{.STEP} data.  It
is missing the values for the stepped parameters.

\textbf{\textit{Workaround}}: There is no workaround for the \texttt{XBEGIN}
and \texttt{XEND} issue.  However, it should not affect the
``viewability'' of those files in the PSpice A/D viewer.  The
workaround for the \texttt{SUBTITLE} issue is to put the stepped
parameters on the \texttt{.PRINT HOMOTOPY} line.
\\ \hline

\textbf{932-SON}: Analysis lines do not support expressions for their
operating parameters & The \Xyce{} parser and analysis handlers do not
yet support the use of expressions on netlist analysis lines such
as \texttt{.TRAN}.  The parameters of these analysis lines (such as
stop time for \texttt{.TRAN} or fundamental frequency
for \texttt{.HB}) may only be expressed as literal numbers.

\textbf{\textit{Workaround}}: There is no workaround internal to \Xyce{}.
Use of an external netlist preprocessor would be required. \\ \hline

\textbf{928-SON}: The .hb\_ic.prn file can be incorrect when .STEP is used
with .HB & \Xyce{} should only output the initial condition (ic) data
for the accepted tolerance in the \texttt{<netlist-name>.hb\_ic.prn}
file.  However, it currently outputs all of the intermediate ic data
while harmonic balance tries to find a good tolerance
if \texttt{.STEP} is used with \texttt{.HB}.

\textbf{\textit{Workaround}}: There is no workaround. \\ \hline

\textbf{911-SON}:  Improve compatibility of multi-file output with the -o
command line option & If \Xyce{} is invoked with the -o command line
option then the output may be ``interleaved'' in one file rather than
appearing correctly in multiple files.  For example, if the
netlist \texttt{example.cir} has these two \texttt{.PRINT} lines: {\tt
\begin{verbatim}
.PRINT TRAN V(1)
.PRINT HOMOTOPY V(1)
\end{verbatim}
}
and is invoked with \texttt{Xyce -o output.text example.cir} then the
output that would normally appear in separate \texttt{example.cir.prn}
and \texttt{example.cir.HOMOTOPY.prn} files would be jumbled together
in the single file \texttt{output.txt}.  This may not be what the user
intended.

\textbf{\textit{Workaround}}: There are two workarounds.  First, don't use -o
if your netlist could output more than one file.  Instead, use
separate \texttt{FILE=} qualifiers on every \texttt{.PRINT} line.
Second, use -o if desired but add
\texttt{FILE=} to every \texttt{.PRINT} line other than \texttt{.PRINT TRAN}
and \texttt{.PRINT DC} lines in your netlist.\\ \hline

\textbf{883-SON} .PREPROCESS REPLACEGROUND does not work on nodes referenced
in expressions & The \texttt{.PREPROCESS REPLACEGROUND} feature does
not replace ground synonyms if they appear in B source expressions.

\textbf{\textit{Workaround}}: Do not use ground synonyms (\texttt{GND},
\texttt{GROUND}, etc.) in expressions.  Use a literal ``0'' when
referring to the ground node in expressions.\\ \hline

\textbf{855-SON}: Missing error message when a netlist uses an operator (e.g.,
IR or P) that is not supported for .AC analyses & This is related to
SON Bug 718.  \Xyce{} will output all zeroes or all NaNs, for the
requested quantity, when a netlist uses an operator (e.g.,
\texttt{IR} or \texttt{P}) that is unsupported for .AC analyses.  Instead,
\Xyce{} should report a netlist-parsing warning or error for this case.

\textbf{\textit{Workaround}}: There is none, other than noticing that an
output waveform value is unexpectedly all zeroes or all
NaNs. \\ \hline

\textbf{812-SON}: Undocumented limitations on, and bugs with, parameter and
global parameter names & Based on external customer input and
pre-release testing, there are some bugs and undocumented limitations
on parameter and global parameter names in
\Xyce{}. Parameters and global parameters should start with a letter, rather
than with a number or ``special'' character like \#.  In addition, the
use of a single character $V$ as a global parameter name can result in
either netlist parsing failures or incorrect results
from \texttt{.PRINT} lines.  \\ \hline

\textbf{807-SON}: BSIM4 convergence problems with non-zero rgatemod value &
There have been reports of convergence problems (e.g., the \Xyce{}
simulation fails part way through and says that the ``time step is too
small'') when the \texttt{rgatemod} parameter is non-zero. \\ \hline

\textbf{794-SON}: Bug in TABLE Form of \Xyce{} Controlled Sources & In some case, a \Xyce{}
netlist that contains a controlled source that uses the TABLE form will get
the correct answer at first.  However, it may then "stall" (e.g, keep
taking really small time-steps) and never complete the simulation run.

\textbf{\textit{Workaround}}: In some cases, the TABLE specification for the controlled
source can be replaced with a Piecewise Linear (PWL) source that uses
nested IF statements. \\ \hline

\textbf{783-SON}: Use of ddt in a B-Source definition may produce incorrect
results & The \texttt{DDT()} function from the \Xyce{} expression
package, which implements a time derivative, may not function
correctly in a B-Source definition.

\textbf{\textit{Workaround}}: None. \\ \hline

\textbf{727-SON}: \Xyce{} parallel builds hang randomly on OS X & During
Sandia's internal nightly testing of the OSX parallel builds, we see
that \Xyce{} ``hangs on exit'' with an estimated frequency of less
than 1-in-5000 simulation runs.  We have not seen this issue with
parallel builds for either RHEL6 or BSD.  The hang is on exit, whether
on a successful exit or on an error exit.  The hang occurs after all
of the \Xyce{} output has occurred though.  So, the user will get
their sim results, but might have trouble if the individual \Xyce{}
runs are part of a larger script.

\textbf{\textit{Workaround}}: None. \\ \hline

\textbf{718-SON}: Missing error message for invalid nodes in expressions on
.PRINT lines & If an invalid node is specified on
a \Xyce{} \texttt{.PRINT TRAN} line then \Xyce{} should return a fatal
error during netlist parsing (e.g., \texttt{.PRINT TRAN V(BOGONODE)}
will produce an error message of \texttt{undefined symbol in .PRINT
command: node BOGONODE}, if \texttt{BOGONODE} does not exist in the
netlist).  However, if the invalid node is inside a \Xyce{} expression
(e.g., \texttt{.PRINT TRAN \{V(BOGONODE)\}}) then \Xyce{} will not
produce an error message during netlist parsing and the output value
for \texttt{\{V(BOGONODE)\}} will be zero for all time-steps.

\textbf{\textit{Workaround}}: There is none, other than noticing that an output
waveform value is unexpectedly all zeroes, and correcting
the \texttt{.PRINT} statement. \\ \hline

\textbf{707-SON}: Behavior for invalid nodes on .FOUR lines and in .MEASURE
statements & There are issues with \texttt{.FOUR} lines
and \texttt{.MEASURE} statements that accidentally use node names that
are not in the netlist.  In that case, the \texttt{.cir.four} output
file will contain a mix of all zero's and NaN's, and \Xyce{} will not
produce a warning or error message about the invalid node name.
Similarly, the measure statement will run without a warning message
about the invalid node name.  The measure result will then be zero,
rather than FAILED. \\ \hline

\textbf{661-SON} Branch currents and power accessors (I(), P() and W()) do
not work properly in .RESULT Statements & There are two issues.
First, \texttt{.RESULT} statements will fail netlist parsing if the
requested branch current is omitted from the \texttt{.PRINT TRAN}
line.  As an example, this statement (\texttt{.RESULT I(R1)}) requires
either \texttt{I(R1)},
\texttt{P(R1)} or \texttt{W(R1)} to be on the \texttt{.PRINT TRAN} line.
Second, the output value, in the \texttt{.res} file, for the lead
current or power calculation will always be zero.
\\ \hline

\textbf{652-SON}: HB output is buggy & While a straightforward use of
\texttt{.print HB} works as described in the users and reference guides,
several of the documented features do not work as intended.

\texttt{.print HB\_FD} and \texttt{.print HB\_TD} are intended as a way of
specifying variable lists for frequency- and time-domain outputs,
respectively.  It has been discovered that these only produce output
if there are print specifications for {\em both\/} frequency and time
domain.  That is, if only one of \texttt{.print HB\_FD}
or \texttt{.print HB\_TD} is present in the netlist, no output will be
produced at all.

\textbf{\textit{Workaround}}: When performing harmonic balance analysis,
always specify enough print lines so that both time- and
frequency-domain variables are output.  This could be by
specifing \texttt{.print HB} alone, by specifying both \texttt{.print
HB} and \texttt{.print HB\_TD}, or by specifying both \texttt{.print
HB\_FD} and \texttt{.print HB\_TD}.
\\ \hline


\textbf{583-SON}: Switch with RON=0 leads to convergence failure. &
The switch device does not prevent a user from
specifying \texttt{RON=0} in its model, but then takes the inverse of
this value to get the ``on'' conductance.  The resulting invalid
division will either lead to a division by zero error on platforms
that throw such errors, or produce a conductance with ``Not A Number''
or ``Infinity'' as value.  This will lead to a convergence failure.

\textbf{\textit{Workaround}}: Do not specify an identically zero resistance
for the switch's ``on'' value.  A small value of resistance such as
1e-15 or smaller will generally work well as a substitute. \\ \hline


\textbf{469-SON}: Belos memory consumption on FreeBSD and excessive CPU on other
platforms & Memory or thread bloat can result when using multithreaded
dense linear algebra libraries, which are employed by Belos.  If this
situation is observed, either build
\Xyce{} with a serial dense linear algebra library or use environment variables
to control the number of spawned threads in a multithreaded library.
\\ \hline


\textbf{468-SON}: It should be legal to have two model cards with the same model
name, but different model types. & SPICE3F5 and ngspice only require
that model cards of the same type have unique model names. They accept
model cards of different types with the same name.  \Xyce{} requires
that all model card names be unique.
\\ \hline


\textbf{250-SON}: NODESET in \Xyce{} is not equivalent to NODESET in SPICE & As
currently implemented, \texttt{.NODESET} applies the initial
conditions given throughout a full nonlinear solve for the operating
point, then uses the result as an initial guess for a second nonlinear
solve with no constraints.  This is not the same as SPICE, which
merely applies the given initial conditions to a single nonlinear
solve for the first two iterations, then lets the problem converge
with no further constraints.  This can lead to
a \Xyce{} \texttt{.NODESET} failing where the same netlist in SPICE
might not, if the initial conditions are such that a full nonlinear
solve cannot converge with those constraints in place.  There is no
workaround.
\\ \hline

\textbf{247-SON}: Expressions don't work on .options lines & Expressions
enclosed in braces (\{ \}) are handled specially throughout \Xyce{},
and may only be used in certain contexts such as in device model or
instance parameters or on \texttt{.PRINT} lines.
\\ \hline


\textbf{49-SON} \Xyce{} BSIM models recognize the model TNOM, but not the
instance TNOM & Some simulators allow the model parameter TNOM of BSIM
devices to be specified on the instance line, overriding the model
parameter TNOM.  \Xyce{} does not support this.
\\ \hline


\textbf{27-SON}: Fix handling of .options parameters & When specifying .options
for a particular package, what gets applied as the non-specified
default options might change.  \\ \hline

\textbf{1962-SRN}: Voltages from interface nodes for subcircuits may not work
correctly in expressions on \texttt{.PRINT} lines & An expression that
uses a voltage from an interface node to a subcircuit on
a \texttt{.PRINT} line may only work if that voltage node is also used
outside of the expression on the \texttt{.PRINT} line.  A simple
example is as follows.  The expression
\texttt{\{V(X1:a)*I(X1:R1)\}} prints out as 0, unless \texttt{V(X1:a)} is also
on the \texttt{.PRINT} line.
\\ \hline

\textbf{1923-SRN}: LC lines run out of memory, even if equivalent (larger) RLC
lines do not. & In some cases, circuits that run fine using an RLC
approximation for a transmission line, exit with an out-of-memory
error if the (supposedly smaller) LC approximation is used.
\\ \hline

\textbf{1595-SRN}: \Xyce{} won't allow access to inductors within subcircuits
for mutual inductors external to subcircuits & It is not possible to
have a mutual inductor outside of a subcircuit couple to inductors in
a subcircuit.

\textbf{\textit{Workaround}}: Put all inductors and mutual inductance lines
that couple to them together at the same level of circuit hierarchy.
\\ \hline

\end{longtable}
}
