% Sandia National Laboratories is a multimission laboratory managed and
% operated by National Technology & Engineering Solutions of Sandia, LLC, a
% wholly owned subsidiary of Honeywell International Inc., for the U.S.
% Department of Energy’s National Nuclear Security Administration under
% contract DE-NA0003525.

% Copyright 2002-2019 National Technology & Engineering Solutions of Sandia,
% LLC (NTESS).



%%
%% Fixed Defects.
%%
{
\small

\begin{longtable}[h] {>{\raggedright\small}m{2in}|>{\raggedright\let\\\tabularnewline\small}m{3.5in}}
     \caption{Fixed Defects.  Note that we have two different Bugzilla systems for Sandia users.
     SON, which is on the open network, and SRN, which is on the restricted network. } \\ \hline
     \rowcolor{XyceDarkBlue} \color{white}\bf Defect & \color{white}\bf Description \\ \hline
     \endfirsthead
     \caption[]{Fixed Defects.  Note that we have two different Bugzilla systems for Sandia Users.
     SON, which is on the open network, and SRN, which is on the restricted network. } \\ \hline
     \rowcolor{XyceDarkBlue} \color{white}\bf Defect & \color{white}\bf Description \\ \hline
     \endhead

     \textbf{972-SON}: Some of the parameter names for the transient sources were incorrect in
     the \Xyce{} Reference Guide & The parameter names for the \texttt{SIN}, \texttt{PULSE}, 
     \texttt{EXP} and \texttt{PWL} sources were incorrect for the I device.  The parameter names 
     for the \texttt{SFFM} source were incorrect for both the V and I devices.  This has been 
     fixed now.  \\ \hline

     \textbf{957-SON}: Segfaults during device instance-line processing & \Xyce{} could segfault on
     improperly formatted device instance lines.  Examples included invalid X device lines and
     invalid composite parameter definitions on YPDE device lines.  This is fixed now.  \\ \hline

     \textbf{955-SON}: Improper handling of netlist lines that only contain an inline comment &
     If a netlist line only contained an inline comment (e.g., the line started with a semicolon)
     then \Xyce{} would all treat subsequent netlist lines as comment lines. This is fixed now.
     \\ \hline

     \textbf{948-SON}: Stepped parameter values are wrong in the ZONE data in .HOMOTOPY.dat files. & 
     The values in the ZONE data in the \Xyce{}-generated .HOMOTOPY.dat files were incorrect when 
     \texttt{.PRINT HOMOTOPY FORMAT=TECPLOT} was used with \texttt{.STEP}  They would either be 
     ``all zeroes'' or the values from the previous step, rather than the correct values for the 
     current step.  This is fixed now.  \\ \hline

     \textbf{942-SON}: Fix issues with ordering of .PRINT AC\_IC and .PRINT AC 
     lines, and of .AC and .OP statements in a netlist &  If an explicit .PRINT AC\_IC line 
     came before a .PRINT AC line in a netlist then \Xyce{} would fail to run.  In addition, 
     the output from an explicit .PRINT AC\_IC line could contain ``extra columns'' that 
     were on the .PRINT AC line but not on the .PRINT AC\_IC line.  Finally, the 
     .PRINT AC\_IC behavior could depend on the relative ordering of the .OP and 
     .AC lines.  This is fixed now.  \\ \hline

     \textbf{930-SON}: Fix and improve information sent to stdout for .NOISE & 
     The stdout for a \texttt{.NOISE} analysis was incorrect when \texttt{.STEP} was used.
     The values output for the ``Total Output Noise'' and ``Total Input Noise'' were wrong.  
     This is fixed now.  As an improvement, the stdout for \texttt{.NOISE} analysis with 
     \texttt{.STEP} now also includes the values of the stepped parameters.  
     \\ \hline

     \textbf{715-SON}: I(*) for subcircuit devices does not work
     properly on .PRINT lines & A bug in the handling of the wildcard
     expression for branch currents ( I(*) ) led to \Xyce{} emitting a
     ``Netlist Error'' about an undefined function or variable if any
     voltage sources appeared inside subcircuits in the netlist.  This
     bug was fixed, and \Xyce{} will now properly output the branch
     currents for voltage sources inside subcircuits.  Only currents
     associated with devices for which the current is a solution
     variable are printed by this wildcard.  Voltage sources and
     inductors are the primary devices of this sort.  Lead currents of
     other devices, for which the current is not a solution variable,
     are not included. \\ \hline

     \textbf{618-SRN}: Fix FORMAT=CSV output for sensitivity output & The print format
     \texttt{FORMAT=CSV} was nominally supported for \texttt{.PRINT SENS} and \texttt{.PRINT
     TRANADJOINT} in previous \Xyce{} releases.  However, the resultant output files
     had .prn extensions rather than .csv extensions.  This is fixed now.
     \\ \hline

     \textbf{2107-SRN and 2109-SRN}: Fix segfaults and inscrutable error messages when 
     parsing vector composite parameters and MPI hangs when parsing .model statements & 
     Misformated vector composite parameters could cause \Xyce{} to either segfault or emit 
     inscrutable error messages when parsing device models.  The triggering of some existing \Xyce{} 
     error/warning messages for misformatted \texttt{.MODEL} statements could cause \Xyce{} 
     to hang during parallel execution.  These issues have been fixed.
     \\ \hline

     \textbf{977-SON}: .MEASURE FOUR is incorrect after the first solve using .STEP &
     A failure to reinitialize some internal storage variables in the FOUR measure code resulted in
     incorrect values after the first solve of a .STEP analysis.  This is fixed now.
     \\ \hline


     \textbf{1001-SON}: Improve Bsrc that leads to convergence problems  &
     The users can specify expressions that have instantaneous transitions with ABM devices. This can 
     lead to convergence problems when \Xyce{} reaches the transition point. The improved ABM devices 
     now allow smooth transitions and can be enabled by specifying device options and instance parameters.
     \\ \hline

     \textbf{37-SON}: Connectivity checker is broken if netlist contains any node with 
     more than 10 connected leads & The topology of any circuit is checked to ensure that
     there is a DC path to ground for all voltage nodes.  This code artificially limited
     the number of lead groups to 10.  Furthermore, the parallel implementation of this
     topology check exhibited severe performance issues for certain circuit topology structures.
     These issues have been fixed.
     \\ \hline

     \textbf{980-SON}: Example .LIB statement that ``swallows'' the rest of the \Xyce{} netlist
     & \Xyce{} supports the HSPICE .LIB syntax, but in some instances the parser would attempt
     to treat \texttt{.LIB <name>} statements the same as \texttt{.INC <name>} statements.
     The logic that supported this was
     flawed and has been removed.  Now \Xyce{} rigorously supports the HSPICE .LIB syntax
     where there is a library definition (\texttt{.LIB <library\_name>}) and a library inclusion
     (\texttt{.LIB <library\_file> <library\_name>}).  Incorrect use of .LIB will now return informative errors.
     \\ \hline

     \textbf{702-SON}: Expressions that use vsrc-currents (in Bsrc's or in .SENS objfunc specifications) 
     can produce wrong derivatives & This problem was caused by an obscure issue in the expression library,
     and it could result in incorrect derivatives.  When this issue was invoked via a sensitivity calculation
     it resulted in incorrect answers.  When the issue was present in Bsrc devices, it potentially impacted
     convergence and robustness.  This issue has been fixed.
     \\ \hline

     \textbf{772-SON}: Order of analysis statements can cause transient sensitivity calculation to 
     incorrectly exit with error.  & \Xyce{} used to allow only a single \texttt{.PRINT} statement 
     to be present in the netlist file, and that single \texttt{.PRINT} statement had to explicitly refer
     to the analysis type that was invoked by that file.  To insure this, the parser contained some error 
     traps for this. In other words, if the user specified transient analyis, but then invoked 
     \texttt{.PRINT DC} that would result in an error.   The code for this error trap was fragile, 
     and not properly updated when \Xyce{} was updated to handle multiple \texttt{.PRINT} lines.  As a result
     sometimes a fatal error would get improperly thrown.  This issue, as it pertains to 
     transient sensitivity analysis, has been fixed.
     \\ \hline


     \textbf{774-SON}: Improperly posed objective function causes sensitivity analysis to coredump.  &  
     Error handling code was missing from the sensitivity analysis functions in the \Xyce{} source.  
     As a result, even if the user-specified objective function expression was not parsable, 
     \Xyce{} would still attempt to compute sensitivities,  causing a coredump.  Error 
     handling code has been added to \Xyce{} to address this issue.
     \\ \hline

     \textbf{963-SON}:  expression library SDT function doesn't work on .PRINT line, and has lots of other problems & 
     The expression library supports numerical time integration with the SDT function.  This function is intended to work
     in both Bsrc devices and also in expressions on the \texttt{.PRINT} line.  Unfortunately, 
     there were several bookkeeping problems in the SDT source code, which prevented this feature 
     from producing correct answers in a variety of \texttt{.PRINT} line use cases.  These issues have been fixed.
     \\ \hline

     \textbf{978-SON}: Vcvs (E-source) device can't be used with PDE devices present in circuit & PDE devices
     (TCAD devices) are often solved using methods that filter out the external circuit for certain phases of the solve.
     In order to support this filtering, non-Ohmic devices (such as Vsrcs and E-sources) are required to 
     conditionally support diagonal matrix elements in their matrix stamp. 
     Most independent sources in \Xyce{} have conditionally 
     supported diagonal matrix elements for a long time, but the E-source was neglected.  The E-source 
     has been updated now, so it will work properly in circuits containing TCAD devices.
     \\ \hline

     \textbf{1007-SON}:   Wrong nonlinear solver used for the TRANOP phase when running a \texttt{.STEP} 
     analysis of a transient & This issue affected some transient calculations performed inside of 
     a \texttt{.STEP} loop.  \Xyce{} uses a different nonlinear solver for the DCOP phase and time 
     stepping phases of transient calculations.  (When running in transient, the initial DCOP 
     calculation is sometimes referred to as TRANOP)  When performing a \texttt{.STEP} loop around a \texttt{.TRAN} analysis, 
     the DCOP solver should be re-established at the beginning of each \texttt{.STEP} iteration. It was not,
     so the transient solver was used on all DCOP calculations other than the first one.  As a 
     result, various features unique to the DCOP solver didn't work, except on the first 
     pass.  These features include the use of \texttt{.IC} and GMIN stepping, among others.
     This issue has been fixed, and the DCOP solver is properly reallocated and reinitialized 
     at the beginning of each \texttt{.STEP} iteration.
\\ \hline


\end{longtable}
}
