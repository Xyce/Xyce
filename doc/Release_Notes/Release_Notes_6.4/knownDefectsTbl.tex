% Sandia National Laboratories is a multimission laboratory managed and
% operated by National Technology & Engineering Solutions of Sandia, LLC, a
% wholly owned subsidiary of Honeywell International Inc., for the U.S.
% Department of Energy’s National Nuclear Security Administration under
% contract DE-NA0003525.

% Copyright 2002-2019 National Technology & Engineering Solutions of Sandia,
% LLC (NTESS).


%%
%% Known defects and workarounds.
%%
%% This section should highlight significant defects that were not fixed in
%% the release

{
\small

\begin{longtable}[h] {>{\raggedright\small}m{2in}|>{\raggedright\let\\\tabularnewline\small}m{3.5in}}
  \caption{Known Defects and Workarounds.} \\ \hline
  \rowcolor{XyceDarkBlue} \color{white}\bf Defect & \color{white}\bf Description
  \\ \hline \endfirsthead
  \caption[]{Known Defects and Workarounds.} \\ \hline
  \rowcolor{XyceDarkBlue} \color{white}\bf Defect & \color{white}\bf Description
  \\ \hline \endhead
% EXAMPLE:
%\textbf{bug number-SRN}: bug title & Description of KNOWN BUG THAT HAS NOT BEEN
  %FIXED.

%\textbf{\textit{Workaround}}: Describe how to work around this bug.
%\\ \hline
%
%  

\textbf{704-SON}: Lead currents do not work in AC analyses, crash \Xyce{} & Lead currents for devices other than voltage sources are not solution variables, and are computed by \Xyce{} as a derived quantity in a post-processing step.  This is not set up correctly to work when doing small-signal AC analysis.  Attempting to print such a lead current or any of its computed quantities (magnitude, phase, real or imaginary parts) will cause \Xyce{} to crash.  The lead currents do work as expected in time-domain analysis such as transient and DC.  Lead currents work properly in HB frequency-domain analysis.

\textbf{\textit{Workaround}}: Do not attempt to print currents through a device other than a voltage source in any AC simulation.  If you need the current through a device under AC analysis, place a zero-volt voltage source in series with that device, and print the current through the voltage source.  
\\ \hline

\textbf{703-SON and 698-SON} Lead currents and the power accessors do
not work consistently in \texttt{.FOUR} and \texttt{.MEASURE} statements &
\texttt{.FOUR} and \texttt{.MEASURE} statements will fail netlist parsing if 
the requested branch current is omitted from the \texttt{.PRINT TRAN} line. 
As an example, this statement (\texttt{.MEASURE TRAN MAXI MAX I(R1)}) requires
either \texttt{I(R1)}, \texttt{P(R1)} or \texttt{W(R1)} to be on the 
\texttt{.PRINT TRAN} line.
\\ \hline 

\textbf{702-SON}  Expressions that use VSRC currents (in B sources or in \texttt{.SENS} objfunc 
specifications) can produce wrong derivatives &  As an example, if \texttt{VA} is a voltage source,
and \texttt{B} is a node name then \texttt{objfunc=\{V(B)*V(B)\}} will get the 
right answer but \texttt{objfunc=\{I(VA)*I(VA)\}} will get the wrong answer. 
\\ \hline

\textbf{657-SON} Make \texttt{.IC} and \texttt{.NODESET} usable inside subcircuits & 
\texttt{.IC} and \texttt{.NODESET} statements within subcircuits are
silently ignored by \Xyce{}. \textbf{\textit{Workaround}}: The \texttt{.IC}
and \texttt{.NODESET} sections of the \Xyce{} Reference Guide describe how to
work around this bug by moving the statements to the top-level of the
circuit.
\\ \hline

\textbf{652-SON}: HB output is buggy & While a straightforward use of \texttt{.print HB} works as described in the users and reference guides, several of the documented features do not work as intended.

\texttt{.print HB\_FD} and \texttt{.print HB\_TD} are intended as a way of specifying variable lists for frequency- and time-domain outputs, respectively.  It has been discovered that these only produce output if there are print specifications for {\em both\/} frequency and time domain.  That is, if only one of \texttt{.print HB\_FD} or \texttt{.print HB\_TD} is present in the netlist, no output will be produced at all.   

This bug was discovered too late to be fixed in time for release 6.4.

\textbf{\textit{Workaround}}: When performing harmonic balance analysis, always specify enough print lines so that both time- and frequency-domain variables are output.  This could be by specifing \texttt{.print HB} alone, by specifying both \texttt{.print HB} and \texttt{.print HB\_TD}, or by specifying both \texttt{.print HB\_FD} and \texttt{.print HB\_TD}.
\\ \hline


\textbf{583-SON}: Switch with RON=0 leads to convergence failure. &  The switch device does not prevent a user from specifying \texttt{RON=0} in its model, but then takes the inverse of this value to get the ``on'' conductance.  The resulting invalid division will either lead to a division by zero error on platforms that throw such errors, or produce a conductance with ``Not A Number'' or ``Infinity'' as value.  This will lead to a convergence failure.

\textbf{\textit{Workaround}}: Do not specify an identically zero resistance for the switch's ``on'' value.  A small value of resistance such as 1e-15 or smaller will generally work well as a substitute. \\ \hline

\textbf{526-SON}: The RISE/FALL/CROSS features of \texttt{.MEASURE} should work for all
measures (where appropriate) & The \texttt{LAST} keyword may not work correctly for
some measure types and syntaxes.
\\ \hline 

\textbf{478-SON}: Measure min/max functions need to support secondary extrema &
The \texttt{RISE} and \texttt{FALL} features of \texttt{.MEASURE} do not work
consistently for noisy waveforms. Every sign change in the waveform’s slope
is interpreted as the start of a new rise or fall.
\\ \hline

\textbf{469-SON}: Belos memory consumption on FreeBSD and excessive CPU on other
platforms &
Memory or thread bloat can result when using multithreaded dense linear algebra
libraries, which are employed by Belos.  If this situation is observed, either build
\Xyce{} with a serial dense linear algebra library or use environment variables to control
the number of spawned threads in a multithreaded library.
\\ \hline


\textbf{468-SON}: It should be legal to have two model cards with the same model
name, but different model types. & SPICE3F5 and ngspice only require that
model cards of the same type have unique model names. They accept model cards
of different types with the same name.  \Xyce{} requires that all model card names be unique.
\\ \hline


\textbf{250-SON}: NODESET in \Xyce{} is not equivalent to NODESET in SPICE & As
currently implemented, \texttt{.NODESET} applies the initial conditions given throughout
a full nonlinear solve for the operating point, then uses the result as an
initial guess for a second nonlinear solve with no constraints.  This is not
the same as SPICE, which merely applies the given initial conditions to a
single nonlinear solve for the first two iterations, then lets the problem
converge with no further constraints.  This can lead to \Xyce{}'s \texttt{.NODESET} failing
where the same netlist in SPICE might not, if the initial conditions are such
that a full nonlinear solve cannot converge with those constraints in place.
There is no workaround.
\\ \hline

\textbf{247-SON}: Expressions don't work on .options lines & Expressions enclosed
in braces (\{ \}) are handled specially throughout \Xyce{}, and may only be used
in certain contexts such as in device model or instance parameters or on
\texttt{.PRINT} lines.
\\ \hline


\textbf{49-SON} \Xyce{} BSIM models recognize the model TNOM, but not the
instance TNOM & 
\\ \hline

\textbf{37-SON}: Connectivity checking is broken for devices with more than 10
leads & The diagnostic code used by the \Xyce{} setup that checks circuit
topology for basic errors such as a node having no DC path to ground or a node
being connected to only one device has a bug in it that causes the code to emit
a cryptic error message, after which the code will exit.  This error has so far
only been seen when a user has attempted to connect a large number of inductors
together using multiple mutual inductor lines.  The maximum number of
non-ground leads that can be used without confusing this piece of code is 10.
If your circuit has that type of large, highly-connected mutual inductor and
the code exits with an error message, this bug may be the source of the problem.

The error message now includes a recommendation to use the workaround below.

\textbf{\textit{Workaround}}: Disable connectivity checking by adding the line

\begin{alltt} .OPTIONS TOPOLOGY CHECK_CONNECTIVITY=0 \end{alltt}

to your netlist.  This will disable the check for the basic errors such as
floating nodes and improperly connected devices, but will allow the netlist to
run with a highly-connected mutual inductor.
\\ \hline


\textbf{27-SON}: Fix handling of .options parameters & When specifying .options
  for a particular package, what gets applied as the non-specified default
  options might change.  \\ \hline

\textbf{1962-SRN}: Voltages from interface nodes for subcircuits may not work
correctly in expressions on \texttt{.PRINT} lines & An expression that uses a 
voltage from an interface node to a subcircuit on a \texttt{.PRINT} line 
may only work if that voltage node is also used outside of the expression on 
the \texttt{.PRINT} line.  A simple example is as follows.  The expression 
\texttt{\{V(X1:a)*I(X1:R1)\}} prints out as 0, unless \texttt{V(X1:a)} is also on 
the \texttt{.PRINT} line.
\\ \hline

\textbf{1923-SRN}: LC lines run out of memory, even if equivalent (larger) RLC
lines do not. &  In some cases, circuits that run fine using an RLC approximation for a
transmission line, exit with an out-of-memory error if the (supposedly smaller) LC
approximation is used.
\\ \hline

\textbf{1903-SRN}: \Xyce{} fails to collect several inductors into a linear mutual inductor &
In some rare cases with complex include file usage, the mutual inductor syntax with multiple couplings can fail to work. \Xyce{} will return an error message that it can not find 
\texttt{L\_L1}:
{\tt
\begin{verbatim}
L_L1         node1 node1 inductance1
L_L2         node3 node4  inductance2
L_L3         node5  node6  inductance3
L_L4         node7 node8  inductance4
K_K1         L_L1 L_L2 L_L3 L_L4   .999 
\end{verbatim}
}

\\ \hline

\textbf{1595-SRN}: \Xyce{} won't allow access to inductors within subcircuits for
mutual inductors external to subcircuits & It is not possible to have a mutual
inductor outside of a subcircuit couple to inductors in a subcircuit.

\textbf{\textit{Workaround}}: Put all inductors and mutual inductance lines that couple to
them together at the same level of circuit hierarchy.
\\ \hline


\end{longtable}
}
