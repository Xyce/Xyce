% Sandia National Laboratories is a multimission laboratory managed and
% operated by National Technology & Engineering Solutions of Sandia, LLC, a
% wholly owned subsidiary of Honeywell International Inc., for the U.S.
% Department of Energy's National Nuclear Security Administration under
% contract DE-NA0003525.

% Copyright 2002-2024 National Technology & Engineering Solutions of Sandia,
% LLC (NTESS).

%%
%% Fixed Defects.
%%
{
\small

\begin{longtable}[h] {>{\raggedright\small}m{2in}|>{\raggedright\let\\\tabularnewline\small}m{3.5in}}
    \caption{Fixed Defects.  The Xyce team has multiple issue
     trackers, and the table below indicates fixed issues by
     indentifying both the tracker and the issue number.  Further,
     some issues are reported by open source users on GitHub and these
     issues may be tracked using multiple issue numbers.} \\ \hline
     \rowcolor{XyceDarkBlue} \color{white}\textbf{Defect} & \color{white}\textbf{Description} \\ \hline
     \endfirsthead
     \caption[]{Fixed Defects.  Note that we have two multiple issue tracking systems for Sandia Users.
     SON and SRN refer to our legacy open- and restricted-network Bugzilla system, and Gitlab refers to issues in our gitlab repositories.  } \\ \hline
     \rowcolor{XyceDarkBlue} \color{white}\textbf{Defect} & \color{white}\textbf{Description} \\ \hline
     \endhead

  \textbf{Xyce Project Backlog/841}: Valgrind reports memory leaks with Intel MKL
  &  Versions of Xyce built with the oneAPI IntelMKL library can report memory leaks when
  run under valgrind.  This was due to memory allocated by the Intel MKL and has been 
  fixed. 
  \\\hline

\textbf{Xyce Project Backlog/858}: Include Python, Matlab, Simulink Examples
  &  Examples of calling Xyce from Python, Matlab and Simulink are now included in the 
  installed version of Xyce within the \texttt{share} directory.
  \\\hline

\textbf{Xyce Project Backlog/861}: Diode model led to Inf/NaN
  &  When the saturation current was set to zero, it would cause the diode model to have Inf/NaN errors.  
  This is a use case that needs to work, and it has been fixed.
  \\\hline

  \textbf{Xyce Project Backlog/884}: Fix doping interpolations to for 1D PDE (TCAD) device to be more robust
  & The interpolation of doping profiles was not working correctly, when the parameter "nmaxchop" was used.  
  This resulted in unphysical noisy solutions in regions of high doping.  This has been fixed.
  \\\hline
  
\textbf{Xyce Project Backlog/xxx}: Description
  &  Details
  \\\hline


\end{longtable}
}
