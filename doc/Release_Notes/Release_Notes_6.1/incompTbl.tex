% Sandia National Laboratories is a multimission laboratory managed and
% operated by National Technology & Engineering Solutions of Sandia, LLC, a
% wholly owned subsidiary of Honeywell International Inc., for the U.S.
% Department of Energy’s National Nuclear Security Administration under
% contract DE-NA0003525.

% Copyright 2002-2019 National Technology & Engineering Solutions of Sandia,
% LLC (NTESS).


%%
%% Incompatibilities with other circuit simulators.
%%
\small

\begin{longtable}[h] {>{\raggedright\small}m{2in}|>{\raggedright\let\\\tabularnewline\small}m{3.5in}}
  \caption{Incompatibilities with other circuit simulators.} \\ \hline
  \rowcolor{XyceDarkBlue}
  \color{white}\bf Issue &
  \color{white}\bf Comment
  \\ \hline
  \endfirsthead
  \caption[]{Incompatibilities with other circuit simulators.} \\ \hline
  \rowcolor{XyceDarkBlue}
  \color{white}\bf Issue &
  \color{white}\bf Comment
  \\ \hline
  \endhead

    \texttt{.DC} sweep output. & The \texttt{.DC} sweep calculation does not
    automatically output the sweep variable.  Only variables explicitly listed on
    the \texttt{.PRINT} line are output.\\ \hline

    Pulsed source rise time of zero. & A requested pulsed source rise/fall
    time of zero really is zero in \Xyce{}.  In other simulators, requesting a
    zero rise/fall time causes them to use the printing interval found on
    the \texttt{.TRAN} line.\\ \hline

    Mutual Inductor Model. & Not the same as PSpice.  This is a Jiles-Atherton non-linear
    model developed at Sandia. However, it is compatible with the PSpice parameter set.\\ \hline

    \texttt{.PRINT} line shorthand. & Output variables have to be specified
    as V(node) or I(source).  Specifying the node alone will not work.  \\ \hline

    MOSFET levels. & In \Xyce{} the MOSFET levels are not the same. A BSIM3 is MOSEFET level 9.  Other simulators have
    different levels for the BSIM3. \\ \hline

    BSIM SOI v3.2 level. & In \Xyce{} the BSIM SOI (v3.2) is MOSFET level 10.  
    Other simulators have different levels for the BSIM SOI. \\ \hline

    BSIM4 level. & In \Xyce{} the BSIM4 is MOSFET levels 14 and 54.  
    Other simulators have different levels for the BSIM4. \\ \hline

    The `\texttt{\%}' symbol is not a valid in a device or node name. &
    The `\texttt{\%}' symbol has a special meaning in \Xyce{}, and therefore cannot
    be used in other cases. \\ \hline

    Interactive mode. & \Xyce{} does not have an interactive mode. \\ \hline

    Syntax for \texttt{.STEP} is different. & The manner of specifying
    a model parameter to be swept is slightly different than in some
    other simulators.  See the \Xyce{} Users' and Reference Guides for
    details.  \\ \hline

    Switch is not the same as SPICE. &  The \Xyce{} switches are not
	compatible with the simple switch implementation in SPICE3F5.  The
	switch in \Xyce{} smoothly transitions between the ON and OFF
	resistances over a small range between the ON and OFF values of the
	control signal (voltage, current, or control expression).  See the
	\Xyce{} Reference Guide for the precise equations that are used to compute the
	switch resistance from the control signal values.  The SPICE3F5 switch
	has a single switching threshold voltage or current, and RON is used
	above threshold while ROFF is used below threshold.  \Xyce{}'s switch is
	considerably less likely to cause transient simulation failures. Results
	similar to SPICE3F5 can be obtained by setting VON and VOFF to the same
	threshold value, but this is not a recommended practice.  \\ \hline

    U digital device syntax differs from PSpice U Device. &  \Xyce{} 6.1 
       contains a new U device with the goal of improved compatibility between
       the instance lines of \Xyce{} and PSpice digital devices.  Known
       incompatibilities are that PSpice uses separate model cards for
       IO and timing characteristics, while \Xyce{} combines the IO and timing
       characteristics in one model card.  The PSpice and \Xyce{} model cards
       also have different parameters. \\ \hline

    U device is different from SPICE3F5 U device. &  The \Xyce{} U devices are
       digital behavioral models, which are intended to be similar to the
       PSpice U digital devices.  The SPICE3F5 U device is a uniform RC
       transmission line model. \\ \hline 

\end{longtable}

