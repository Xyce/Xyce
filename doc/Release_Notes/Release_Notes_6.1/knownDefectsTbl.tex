% Sandia National Laboratories is a multimission laboratory managed and
% operated by National Technology & Engineering Solutions of Sandia, LLC, a
% wholly owned subsidiary of Honeywell International Inc., for the U.S.
% Department of Energy’s National Nuclear Security Administration under
% contract DE-NA0003525.

% Copyright 2002-2019 National Technology & Engineering Solutions of Sandia,
% LLC (NTESS).


%%
%% Known defects and workarounds.
%%
%% This section should highlight significant defects that were not fixed in
%% the release

\small

\begin{longtable}[h] {>{\raggedright\small}m{2in}|>{\raggedright\let\\\tabularnewline\small}m{3.5in}}
  \caption{Known Defects and Workarounds.} \\ \hline
  \rowcolor{XyceDarkBlue} \color{white}\bf Defect & \color{white}\bf Description
  \\ \hline \endfirsthead
  \caption[]{Known Defects and Workarounds.} \\ \hline
  \rowcolor{XyceDarkBlue} \color{white}\bf Defect & \color{white}\bf Description
  \\ \hline \endhead
% EXAMPLE:
%\textbf{bug number-SRN}: bug title & Description of KNOWN BUG THAT HAS NOT BEEN
  %FIXED.

%\textbf{\textit{Workaround}}: Describe how to work around this bug.
%\\ \hline
%
%

   
\textbf{27-SON}: Fix handling of .options parameters & When specifying .options
  for a particular package, what gets applied as the non-specified default
  options might change.  \\ \hline


\textbf{37-SON}: Connectivity checking is broken for devices with more than 10
leads & The diagnostic code used by the \Xyce{} setup that checks circuit
topology for basic errors such as a node having no DC path to ground or a node
being connected to only one device has a bug in it that causes the code to emit
a cryptic error message, after which the code will exit.  This error has so far
only been seen when a user has attempted to connect a large number of inductors
together using multiple mutual inductor lines.  The maximum number of
non-ground leads that can be used without confusing this piece of code is 10.
If your circuit has that type of large, highly-connected mutual inductor and
the code exits with an error message, this bug may be the source of the problem.

The error message now includes a recommendation to use the workaround below.

\textbf{\textit{Workaround}}: Disable connectivity checking by adding the line

\begin{alltt} .OPTIONS TOPOLOGY CHECK_CONNECTIVITY=0 \end{alltt}

to your netlist.  This will disable the check for the basic errors such as
floating nodes and improperly connected devices, but will allow the netlist to
run with a highly-connected mutual inductor.
\\ \hline


\textbf{195-SON}: Restart files are not produced if initial\_interval is not
specified & The users' guide states that if \texttt{.OPTIONS RESTART} is
specified without an \texttt{INITIAL\_INTERVAL} that the restart file will be
saved at every time step.  This is not working, and if no initial interval is
specified, no restart is saved at all.  

\textbf{\textit{Workaround}}: Always specify a suitable initial interval when requesting
restart.
\\ \hline


\textbf{244-SON}: The defaults in Xyce bsimsoi do not match defaults in spice3
bsimsoi & Some of the default parameters used in the BSIM3 SOI (level 10)
MOSFET do not match SPICE3F5's defaults, leading to observed differences in
behavior if a model card does not specify these parameters.  

\textbf{\textit{Workaround}}:  The parameters at issue appear all to be parameters that
have \textit{computed} rather than hard-coded defaults.  These include
\texttt{CGDO}, \texttt{CGSO}, \texttt{K1}, \texttt{K2}, \texttt{XJ}.  It is
best to use extracted values for these parameters rather than relying on the
defaults until this bug is fixed.
\\ \hline


\textbf{247-SON}: Expressions don't work on .options lines & Expressions enclosed
in braces (\{ \}) are handled specially throughout Xyce, and may only be used
in certain contexts such as in device model or instance parameters or on
\texttt{.PRINT} lines.
\\ \hline


\textbf{250-SON}: NODESET in xyce is not equivalent to NODESET in SPICE & As
currently implemented, .NODESET applies the initial conditions given throughout
a full nonlinear solve for the operating point, then uses the result as an
initial guess for a second nonlinear solve with no constraints.  This is not
the same as SPICE, which merely applies the given initial conditions to a
single nonlinear solve for the first two iterations, then lets the problem
converge with no further constraints.  This can lead to Xyce's .NODESET failing
where the same netlist in SPICE might not, if the initial conditions are such
that a full nonlinear solve cannot converge with those constraints in place.
There is no workaround.
\\ \hline


\textbf{365-SON}: Missing Default Instance Length Parameter in Resistor Model
Parameters & The resistor allows a semiconductor resistor formulation given a
sheet resistance, length, and width, but no reasonable default is given for the
length.  A default width parameter is provided through the model card, and is
used if no width is specified on the instance line.

\textbf{\textit{Workaround}}: Always specify a length on the instance line for any
resistor for which the semiconductor resistor syntax is used (see the \Xyce{} Reference Guide for usage).
\\ \hline


\textbf{384-SON}: Errors compiling with Bison 3.0 & Bison 3.0's C++ parser has
changed significantly and thus Bison version 3.0 cannot be used to build Xyce's
reaction parser.  

\textbf{\textit{Workaround}}: Use only Bison versions between 2.3 and 2.7 if reaction
parsing is desired.  Disable building of the reaction parser by configuring with
\texttt{--disable-reaction\_parser} if unable to provide a suitable version of
Bison.
\\ \hline


\textbf{427-SON}: Certain subcircuit nodes cannot be accessed in expressions &
Xyce allows nodes within a subcircuit to be referenced in expressions and on
.print lines using the syntax ``$<$subcircuit instance
name$>$:$<$nodename$>$''.  On .print lines this works even if the node named is
one that is on the subcircuit declaration line (i.e., dummy argument nodes,
which are just aliases for names in the calling circuit context).  This does
not work inside expressions.

\textbf{\textit{Workaround}}: Use the name of the node at the higher level of the
circuit instead of the dummy argument names.
\\ \hline


\textbf{429-SON}: Support AC print-line accessors for magnitude, phase, etc. of
branch currents & SPICE is capable of outputting the magnitude, phase, real and
imaginary parts, and magnitude in dB for currents.  Xyce currently supports
these accessors only for voltage nodes.
\\ \hline


\textbf{438-SON}: .step output for HB in Standard (.prn) format does not reset
Index for each step or otherwise mark new step data &  It is difficult to separate the results, in an .HB.FD.prn file, for each step because the Index column
is monotonic instead of resetting to zero at the beginning of each step.  In transient or DC analysis runs, .step (wrapping the other analysis) causes the
Index column to reset to zero for every step.
\\ \hline


\textbf{439-SON}: Interpolation does not work well with TRAP and GEAR &
The problem occurs in certain situations when the output is interpolated, e.g.,
in a circuit with 
\begin{verbatim}
.tran 50us 15ms
.options output initial_interval=150us
.print tran i(V2)
\end{verbatim}
as part of the output directions could show the problem. The problem does
go away when using BDF. 
\textbf{\textit{Workaround}}: Use BDF, or do not specify an
\texttt{initial\_interval}. 
\\ \hline


\textbf{442-SON}: lead currents are not correctly output for AC and HB analysis &

The lead currents were not calculated in HB and AC analyses. The output will be
all zeros for the lead currents in HB and AC analyses.

\\ \hline


\textbf{466-SON}: .RESULT output doesn't recognize .param parameters &  As
an example, consider: 
{\tt
\begin{verbatim}
.param par1=1.0 
.param par2=2.0 
.result {par1/par2} 
\end{verbatim}
}
The code will run for a while, and then when the first .RESULT output is
processed (at the end of the first .STEP iteration) it will exit with an error,
claiming that par1 and par2 are not recognized variables.

This .print line does work though:
{\tt
\begin{verbatim}
.print tran {par1/par2}
\end{verbatim}
}
\\ \hline


\textbf{468-SON}: It should be legal to have two model cards with the same model
name, but different model types. & SPICE3F5 and ngspice only require that
model cards of the same type have unique model names. They accept model cards
of different types with the same name.  Xyce requires that all model card names be unique.
\\ \hline


\textbf{469-SON}: Belos memory consumption on FreeBSD and excessive CPU on other
platforms &
\\ \hline


\textbf{477-SON}: Internal variables are not accessible inside expressions &
An internal variable, accessed via n(…), is not properly resolved in an
expression. As an example, this fails parsing. 
{\tt
\begin{verbatim}
.print tran {n(y%neuron%neuron1_V1)} 
\end{verbatim}
}
\textbf{\textit{Workaround}}: If the braces are removed then the netlist
runs without problems.
\\ \hline

\textbf{483-SON}: ASCII rawfile output does not work with .dc without the
-r flag &  As an example, if the netlist:
{\tt
\begin{verbatim}
VAB 1 0 5
.dc VAB 0 5 1
.print dc format=raw v(1) 
.end
\end{verbatim}
}
is run using ``Xyce -a $<$filename$>$'', the resulting file is in
 the binary format, not ASCII. 

\textbf{\textit{Workaround}}: Using both the ``-r $<$rawfile$>$'' and
``-a'' flags will result in $<$rawfile$>$ being in the ASCII format.
\\ \hline

\textbf{484-SON}: putting resistor parameter on .PRINT line, with homotopy simulation causes segfault &
For example,\newline
\texttt{
.DC RC:R 2K 2K 2K\newline
.PRINT DC format=tecplot RC:R V(4)\newline
+ I(VMON1) I(VMON2) V(1) V(2)\newline
}
causes a segfault.
\\ \hline

\textbf{1595-SRN}: Xyce won't allow access to inductors within subcircuits for
mutual inductors external to subcircuits & It is not possible to have a mutual
inductor outside of a subcircuit couple to inductors in a subcircuit.

\textbf{\textit{Workaround}}: Put all inductors and mutual inductance lines that couple to
them together at the same level of circuit hierarchy.
\\ \hline


\textbf{1922-SRN}: Xyce nonlinear core model defaults seriously underestimate
hysteresis effects for small signals & The magnetic core model has some
parameter defaults that effectively turn off magnetic effects when
small-amplitude driving signals are applied.  This can show up when attempting
to generate B-H loops for magnetic cores using standard test harnesses.  A
robust solution to this issue is being sought.

\textbf{\textit{Workaround}}: Set the model parameter CONSTDELVSCALING to false.  This
will activate dynamic voltage scaling withing the mutual inductor to better capture
the magentic effects for small signals.  If that failes, then one can also revert to 
the default behavior (i.e. CONSTDELVSCALING=true) and adjust the model paramter DELVSCALING
manually to work with a given circuit.  Make DELVSCALING as large as possile while 
still getting circuit convergence (e.g. 1e2 to 1e6 may work).
\\ \hline


\end{longtable}

