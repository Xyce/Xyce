% Sandia National Laboratories is a multimission laboratory managed and
% operated by National Technology & Engineering Solutions of Sandia, LLC, a
% wholly owned subsidiary of Honeywell International Inc., for the U.S.
% Department of Energy’s National Nuclear Security Administration under
% contract DE-NA0003525.

% Copyright 2002-2019 National Technology & Engineering Solutions of Sandia,
% LLC (NTESS).


%%
%% Fixed Defects.
%%
\small

\begin{longtable}[h] {>{\raggedright\small}m{2in}|>{\raggedright\let\\\tabularnewline\small}m{3.5in}}
  \caption{Fixed Defects.  Note that we have two different Bugzilla systems for Sandia users.
  SON, which is on the open network, and SRN, which is on the restricted network. } \\ \hline
  \rowcolor{XyceDarkBlue} \color{white}\bf Defect & \color{white}\bf Description \\ \hline \endfirsthead
  \caption[]{Fixed Defects.  Note that we have two different Bugzilla systems for Sandia Users.
  SON, which is on the open network, and SRN, which is on the restricted network. } \\ \hline
  \rowcolor{XyceDarkBlue} \color{white}\bf Defect & \color{white}\bf Description \\ \hline \endhead

%\textbf{bug xxxx }: Describe issue and fix. \\ \hline 

% joseki (-SON)bugs:
%% \textbf{BUG NUMBER-SON}: Bug title & Bug and fix description \\ \hline
\textbf{474-SON}: Errors in HB time domain output and during HB solve &  The
time domain output is incorrectly shifted when a nonzero startupperiods is
specified. During HB solve, the related vector and matrix evaluations and
voltage limiting handling have errors.

\textbf{\textit{Solution}}: These errors were fixed.
\\ \hline  

\textbf{475-SON}: Voltage limiting information is not assembled consistently
for high orders in the Trap and Gear time integrators & The voltage limiting
information obtained from device package has F and Q parts. They need to be
assembled in the similar way to how the Jacobian is assembled. For the Trap and
Gear time integrators, it is consistent for the first order. Higher order needs
to be fixed.

\textbf{\textit{Solution}}: This was fixed.
\\ \hline

\textbf{387-SON}: Failure to find $<$file$>$ in .lib $<$file$>$ causes Xyce to
emit confusing error message & If a \texttt{.LIB} statement is given a file
argument, but the file cannot be found, an error in input processing causes
Xyce to exit with an error stating that no analysis statement was specified
instead of a more meaningful ``cannot find file'' error.   

\textbf{\textit{Solution}}: The code now emits a ``Could not find include file
\ldots'' when no filename is present.
\\ \hline  


\textbf{401-SON}: Frequency-domain-specific voltage drop specifiers produce
incorrect output & The VR, VI, VM, VP and VDB output specifiers for printing
real and imaginary parts, magnitude, phase, or magnitude in DB of voltages in
frequency domain do not work if given two nodes, but Xyce will not emit any
warning or error messages.

\textbf{\textit{Solution}}: This was fixed in the parser.
\\ \hline


%This is at VERIFIED FIXED AS OF 3/12
\textbf{412-SON}: .measure ... when ... cross= does not seem to work as advertised
& The code to process the .measure ... when... cross=... construct is not
working correctly. 

\textbf{\textit{Solution}}: The code was only checking for zero crossings. This
was corrected.
\\ \hline


\textbf{413-SON}: Parser ignores instance parameters if user incorrectly places
them before model name & If instance parameters (e.g. length or width of a
MOSFET device) are placed before the model name on an instance line, Xyce
simply ignores them without emitting a syntax error message.

\textbf{\textit{Solution}}: An error message is now output when the above
conditions occur.
\\ \hline


\textbf{440-SON}: Measure functions should support lead currents and expressions. &
The measure function code was only parsing out voltage nodes and current values for
voltage sources.  Better solution variable parsing was needed in the measure support code.

\textbf{\textit{Solution}}: Parsing was improved in the measure code to support lead 
currents and expressions in measure statements.  
\\ \hline

\textbf{460-SON}: Support equaiton evaluation measure. & This was a feature request 
to support the general evaluation of an equation within the same context as 
the measure functions.

\textbf{\textit{Solution}}:  The measure function "EQN" has been added to support 
the evaluation of expressions during the simulation.
\\ \hline

\textbf{461-SON}: A \emph{Version} string should not be part of the RAW file format header. & 

\textbf{\textit{Solution}}:  \emph{Version} has been removed from Xyce's default RAW header. 
The \emph{Version} string can be re-enabled by setting the following option in the netlist:
\texttt{.options output outputversioninrawfile=true}
\\ \hline

\textbf{481-SON}: Measure functions should have a settable default. &  The default value
for a measure function was zero, but if the measure function was unable to calculate a value
it was unclear to the user if this was the case. 

\textbf{\textit{Solution}}:  The measure functions now have a default value of $-1$.  This is settable 
in the measure function definition with the paramter \texttt{default\_val} as in \texttt{default\_val=-100}.
\\ \hline


% charleston (-SRN) bugs:
%%\textbf{BUG NUMBER-SRN}: Bug title & Bug and fix description \\ \hline

\textbf{772-SRN}: Infinite-slope transitions in B-sources causes ``time step too
small'' errors & The nonlinear dependent source (``B-source'') allows the user
to specify expressions that could have infinite-slope transitions, such as

\begin{alltt} Bcrtl OUTA 0 V=\{ IF( (V(IN) > 3.5), 5, 0 ) \} \end{alltt}

This can lead to ``timestep too small'' errors when \Xyce{} reaches the
transition point.  Infinite-slope transitions in expressions dependent only on
the \texttt{time} variable are a special case, because \Xyce{} can detect that
they are going to happen in the future and set a ``breakpoint'' to capture
them.  

\textbf{\textit{Solution}}: Infinite-slope transitions depending on other
solution variables cannot be predicted in advance, and cause the time
integrator to scale back the timestep repeatedly in an attempt to capture the
feature until the timestep is too small to continue.  Since this kind of
situation is non-physical, and any fix would likely be not robust, this bug was
closed as a WONTFIX.
\\ \hline


\textbf{855-SRN}: AC and HB analysis do not work in parallel & Analysis types that
require use of block linear algebra classes in Trilinos/Epetra do not work in
parallel.  The analysis types that use block linear algebra classes are AC and
HB.

\textbf{\textit{Solution}}: This was fixed with recent enhancements to the
solvers.
\\ \hline

\textbf{1918-SRN}: The FREQ measure function should support the From/To time qualifiers. & 

\textbf{\textit{Solution}}: The FREQ measure function now supports the From/To qualifiers to
limit the time window where the frequency is measured.
\\ \hline


\end{longtable}

