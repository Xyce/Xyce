% Sandia National Laboratories is a multimission laboratory managed and
% operated by National Technology & Engineering Solutions of Sandia, LLC, a
% wholly owned subsidiary of Honeywell International Inc., for the U.S.
% Department of Energy’s National Nuclear Security Administration under
% contract DE-NA0003525.

% Copyright 2002-2019 National Technology & Engineering Solutions of Sandia,
% LLC (NTESS).


%%-------------------------------------------------------------------------
%% Purpose        : Release notes for Xyce version 6.2.
%% Special Notes  : Graphic files (pdf format) work with pdflatex.  
%% Creation Date  : {03/24/2006}
%%-------------------------------------------------------------------------

\documentclass[11pt,report,strict]{SANDreport}
\usepackage[hyperindex=true, colorlinks=false]{hyperref}
\usepackage{makeidx, ltxtable, multirow}
\usepackage{Xyce}
\pdfcatalog {/PageMode /UseNone}

{\renewcommand{\arraystretch}{1.2}

% ---------------------------------------------------------------------------- %
%
% Set the title, author, and date
%
    \title{\XyceTitle{} Parallel Electronic Simulator Release Notes\\Release 6.2}
    \XyceVersion{6.2}

    % There is a "Printed" date on the title page of a SAND report, so
    % the generic \date should generally be empty.
    \date{}

% ---------------------------------------------------------------------------- %
% Set some things we need for SAND reports. These are mandatory
%
%Submitted to R&A 18 Sep 2014 Tracking number 155786,  approved 24 Sep

\SANDnum{SAND2014-18070}
\SANDprintDate{September 2014}
\author{Eric R. Keiter, 
            Ting Mei, 
            Thomas V. Russo, \\
            Richard L. Schiek, 
            Peter E. Sholander,
            Heidi K. Thornquist,
            Jason C. Verley\\
            Electrical Models and Simulation \\[1ex]
\\
            Sandia National Laboratories\\
            P.O. Box 5800, M/S 1177\\
            Albuquerque, NM 87185-1177 \\
\\
\and 
David G. Baur\\
  Raytheon\\
  1300 Eubank Blvd\\
  Albuquerque, NM 87123 \\
}
\SANDauthor{Eric R. Keiter, Ting Mei, Thomas V. Russo, Richard L. Schiek, Peter E. Sholander, Heidi K. Thornquist, Jason C. Verley, David G. Baur}
% ---------------------------------------------------------------------------- %
% Start the document

\begin{document}
\maketitle
%\draft
\begin{abstract}
  The highlights of the \XyceTM\ 6.2 release are documented.
\end{abstract}
\cleardoublepage
\tableofcontents


% ---------------------------------------------------------------------- %
% This is where the body of the report begins; usually with an Introduction
%
\SANDmain		% Start the main part of the report
\section{Scope/Product Definition}

The \Xyce{} Parallel Electronic Simulator has been written to support, in a
rigorous manner, the simulation needs of the Sandia National Laboratories
electrical designers.  Specific requirements include, among others, the
ability to solve extremely large circuit problems by supporting large-scale
parallel computing platforms, improved numerical performance and
object-oriented code design and implementation.

The \Xyce{} release notes describe:
\begin{XyceItemize}
  \item Hardware and software requirements
  \item New features and enhancements
  \item Any defects fixed since the last release
  \item Current known defects and defect workarounds
  \item Incompatibilities With Other Circuit Simulators
\end{XyceItemize}

For up-to-date information not available at the time these notes 
were produced, please visit the \Xyce{} web page at 
{\color{XyceDeepRed}\url{http://xyce.sandia.gov/}}.


\section{Hardware/Software}
This section gives basic information on supported platforms and hardware
and software requirements for running \Xyce{} 6.2.

\subsection{Supported Platforms (Certified Support)}
\Xyce\ 6.2 currently supports any of the following operating system platforms
(all versions imply the earliest supported---\Xyce{} generally works on later
versions as well).  These platforms are supported in the sense that they have
been subject to certification testing for the \Xyce{} version 6.2 release.  

\begin{XyceItemize}
\item Red Hat Enterprise Linux${}^{\mbox{\textregistered}}$ 5, x86 (serial only) and x86-64 (serial and parallel)
\item Red Hat Enterprise Linux${}^{\mbox{\textregistered}}$ 6, x86-64 (serial and parallel)
\item Microsoft Windows 7${}^{\mbox{\textregistered}}$, x86 (serial)
\item Apple${}^{\mbox{\textregistered}}$ OS X, x86-64 (serial and parallel)
\item TLCC (serial and parallel)
\item Red Sky (serial and parallel)
\end{XyceItemize}

\subsection{Build Supported Platforms (not Certified)}
The platforms listed in this section are ``not supported'' in the
sense that they are not subject to nightly regression testing, and
they also were not subject to certification testing for the \Xyce{}
version 6.2 release.  Despite this lack of testing rigor, \Xyce{} has
been known to run under these configurations.

\begin{XyceItemize}
\item FreeBSD 9.x on Intel x86 and x86-64 architectures (serial and parallel)
\item Distributions of Linux other than Red Hat Enterprise Linux
\item Microsoft Windows under Cygwin and MinGW.
\end{XyceItemize}

Please contact the Xyce development team for platform and configuration availability.

\subsection{Hardware Requirements}

The \Xyce{} team has not determined a minimum memory or processor
speed requirement.  Any modern computer should have enough memory and
processor power to run moderately sized circuits in serial with
\Xyce{}.  Naturally, memory requirements grow with problem size.  

Running \Xyce{} in parallel will require a system with at least two
processors. For problems of the size where parallel operation is
beneficial (thousands of devices per processor), one can expect to
need several gigabytes of memory per processor. 

For the very largest problems that \Xyce{} can run (millions of
devices), one will require a cluster system with a sufficient number
of nodes to distribute the problem efficiently, and sufficient memory
per node to support the distributed problem.  

\subsection{Software Requirements}
Several libraries are required to run \Xyce{} or build \Xyce{} from source on a
platform.  Serial versions of the \Xyce{} binary have no run-time software
requirements, as they ship with all the shared libraries they need.  However,
parallel versions require Open MPI
({\color{XyceDeepRed}\url{http://www.open-mpi.org/}}) (version 1.4 or higher)
at run time.  TLCC and Red Sky users can load the \textbf{xyce} module to properly
set the environment. 

Several libraries (all freely available from Sandia National Laboratories or
other sites) are always required when building \Xyce{} from source.  These are:
\begin{XyceItemize}
\item Trilinos Solver Library version 11.10.2
  ({\color{XyceDeepRed}\url{http://trilinos.org}}).  Trilinos is a suite of
  packages that includes solver and partitioning libraries in parallel and
  serial, in addition to many other utilites.
\item UMFPACK version 4.1 and AMD version 1.0 (libumfpack.a, libamd.a)
  ({\color{XyceDeepRed}
  \url{http://www.cise.ufl.edu/research/sparse/umfpack/}}).  This is also
  provided by the SuiteSparse package that is available on many systems.
\item BLAS (Basic Linear Algebra Subprograms). This may be included with the
  compiler, or it may require separate package installation. One may also use
  ATLAS ({\color{XyceDeepRed} \url{http://math-atlas.sourceforge.net}}).
\item LAPACK (Linear Algebra PACKage). This may be included with the
  compiler, or it may require separate package installation. One may also use
  ATLAS ({\color{XyceDeepRed} \url{http://math-atlas.sourceforge.net}}).
\end{XyceItemize}
For parallel builds, the following libraries are additionally required:
\begin{XyceItemize}
\item Open MPI ({\color{XyceDeepRed}\url{http://www.open-mpi.org}}) library for
  message passing (version 1.4 or higher).  The version used to build Xyce must
  be the same that is used for building Trilinos.
\item ParMETIS ({\color{XyceDeepRed}
  \url{http://glaros.dtc.umn.edu/gkhome/metis/parmetis/overview}}) library for
  graph partitioning (version 3.1 or higher).  The MPI compiler used to compile
  ParMETIS must be the same that is used for Trilinos and Xyce.
\end{XyceItemize}

\section{\Xyce{} Release 6.2 Documentation}
The following \Xyce{} documentation is available at the \Xyce{} website in pdf
form.
\begin{XyceItemize}
\item \Xyce{} Release Notes, Version 6.2 (this document)
\item \Xyce{} Users' Guide, Version 6.2
\item \Xyce{} Reference Guide, Version 6.2
\item \Xyce{} Mathematical Formulation
\item Application Node: Using Open Source Schematic Capture Tools with \Xyce{}
\end{XyceItemize}
Also included at the \Xyce{} website as web pages are the following.
\begin{XyceItemize}
\item Building Guide (instructions for building Xyce from the source code)
\item Running the Xyce Regression Test Suite
\item Frequently Asked Questions
\end{XyceItemize}

\section{New Features and Enhancements}

Highlights for \Xyce{} Release 6.2 include:
\begin{XyceItemize}
  \item Transient direct sensitivities.
  \item Support lead current calculation and output for Harmonic Balance (HB) analysis.
  \item Improved interpolation for both the Trapezoid and Gear
    time integrators.
  \item Improved results output (.PRINT) capabilities.
  \item Lumped tranmission line model.
  \item Support for a new digital latch (DLTCH) model.
  \item Multiple .PRINT's result in multiple output files.
  \item GMIN stepping is attempted automatically if the initial attempt to find
    a DC operating point fails.
  \item ``\%'' no longer a reserved character in names.
  \item ``!'' is a reserved character in names to separate device name from ancillary information.
\end{XyceItemize}
For details of each of these new features, see the \Xyce{} Users' Guide, the
\Xyce{} Installation Guide, and the \Xyce{} Reference Guide.  A more complete
listing of new features and improvements is given in the following sections.

\subsection{Device Support}
Table~\ref{deviceListTable} contains a complete list of devices for \Xyce{}
Release 6.2.

% device list 
% Sandia National Laboratories is a multimission laboratory managed and
% operated by National Technology & Engineering Solutions of Sandia, LLC, a
% wholly owned subsidiary of Honeywell International Inc., for the U.S.
% Department of Energy’s National Nuclear Security Administration under
% contract DE-NA0003525.

% Copyright 2002-2019 National Technology & Engineering Solutions of Sandia,
% LLC (NTESS).


%%
%% Analog Device Description Table.
%%
\small

\begin{longtable}[h,t,b,p] {|
>{\setlength{\hsize}{0.40\hsize}}Y|
>{\setlength{\hsize}{0.60\hsize}}Y|} 

\caption{Devices Supported by Xyce\label{deviceListTable}} \\ \hline

\rowcolor{XyceDarkBlue}
\color{white}\bf Device &
\color{white}\bf Comments
\\ \hline
\endhead
    Capacitor & Age-aware, semiconductor \\ \hline

    Inductor & Nonlinear mutual inductor (see below) \\ \hline

    Nonlinear Mutual Inductor & Sandia Core model (not fully PSpice
    compatible) Stability improvements\\ \hline

    Resistor (Level 1) & Semiconductor \\ \hline

    Resistor (Level 2) & Thermal Resistor \\ \hline

    Diode (Level 1) &  \\ \hline

    Diode (Level 2) & Addition of PSPICE enhancements \\ \hline

    Independent Voltage Source (VSRC) & \\ \hline

    Independent Current Source (ISRC) & \\ \hline

    Voltage Controlled Voltage Source (VCVS) & \\ \hline
    Voltage Controlled Current Source (VCCS) & \\ \hline
    Current Controlled Voltage Source (CCVS) & \\ \hline
    Current Controlled Current Source (CCCS) & \\ \hline

    Nonlinear Dependent Source  (B Source) & \\ \hline

    Bipolar Junction Transistor (BJT) (Level 1)&  \\ \hline

    Bipolar Junction Transistor (BJT) (Level 10)& 
Vertical Bipolar Intercompany (VBIC) model, version 1.2 \\ \hline

    Bipolar Junction Transistor (BJT) (Level 23)& 
FBH (Ferdinand-Braun-Institut f\"ur H\"ochstfrequenztechnik) HBT model, version 2.1 \color{red}{\bf New!} \color{black}\\ \hline

    Junction Field Effect Transistor (JFET) (Level 1)  &  
SPICE-compatible JFET model\\ \hline
    Junction Field Effect Transistor (JFET) (Level 2) &  
Shockley JFET model\\ \hline

    MESFET &  \\ \hline

    MOSFET (Level 1) &  \\ \hline
    MOSFET (Level 2) &  Spice level 2 MOSFET \\ \hline
    MOSFET (Level 3) &  \\ \hline
    MOSFET (Level 6) &  Spice level 6 MOSFET \\ \hline
    MOSFET (Level 9) &  BSIM3 model \\ \hline
    MOSFET (Level 10) & BSIM SOI model \\ \hline
    MOSFET (Level 14) & BSIM4 model \\ \hline
    MOSFET (Level 18) &  VDMOS general model \\ \hline
    MOSFET (Level 103 ) & PSP model \\ \hline
    MOSFET (Level 301)& EKV model \\ \hline

    Transmission Line &  Lossless \\ \hline
    Transmission Line &  Lossy \color{red}{\bf New!} \color{black}\\ \hline

    Controlled Switch (S,W) (VSWITCH/ISWITCH) & Voltage or current controlled\\ \hline

    Generic Switch (SW) & Controlled by an expression\\ \hline

    PDE Devices (Level 1) & one-dimensional\\ \hline

    PDE Devices (Level 2) & two-dimensional\\ \hline

    Digital (Level 1)  & Behavioral Digital \\ \hline

    EXT (Level 1)  & External device, used for code coupling and power-node parasitics simulations \\ \hline

    OP AMP (Level 1) & Ideal operational amplifier \\ \hline 

    ACC & Accelerated mass device, used for simulation of electromechanical and magnetically-driven machines \\ \hline

    ROM (Level 1) & Reduced-order model device for linear (RLC) circuits\\ \hline

\end{longtable}



%\subsection{Robustness Improvements}
%\begin{XyceItemize}
%\item Placeholder
%\end{XyceItemize}

% new device types:
\subsection{New Devices}
\begin{XyceItemize}
\item Digital latch model (DLTCH).
\item Lumped transmission line model.
\end{XyceItemize}

\subsection{Enhanced Solver Stability, Performance and Features}
\begin{XyceItemize}
\item Transient direct sensitivity analysis.  This is an enhancement to {\texttt{.SENS}},
which previously only applied to steady-state (.DC) analysis.
\item New nonlinear solution strategy similar to SPICE, so that when the standard Newton method fails to get a DC Operating Point (DCOP), GMIN stepping is automatically attempted.  This new strategy is now default behavior in \Xyce{}.
\end{XyceItemize}

%\subsection{Interface Improvements}
%\begin{XyceItemize}
%\item   Placeholder
%\end{XyceItemize}

\newpage
\section{Defects of Release 6.1 Fixed in this Release}
% Sandia National Laboratories is a multimission laboratory managed and
% operated by National Technology & Engineering Solutions of Sandia, LLC, a
% wholly owned subsidiary of Honeywell International Inc., for the U.S.
% Department of Energy's National Nuclear Security Administration under
% contract DE-NA0003525.

% Copyright 2002-2025 National Technology & Engineering Solutions of Sandia,
% LLC (NTESS).

% Sandia National Laboratories is a multimission laboratory managed and
% operated by National Technology & Engineering Solutions of Sandia, LLC, a
% wholly owned subsidiary of Honeywell International Inc., for the U.S.
% Department of Energy's National Nuclear Security Administration under
% contract DE-NA0003525.

% Copyright 2002-2025 National Technology & Engineering Solutions of Sandia,
% LLC (NTESS).


%%
%% Fixed Defects.
%%
{
\small

\begin{longtable}[h] {>{\raggedright\small}m{2in}|>{\raggedright\let\\\tabularnewline\small}m{3.5in}}
     \caption{Fixed Defects.  The Xyce team has multiple issue
     trackers, and the table below indicates fixed issues by
     indentifying both the tracker and the issue number.  Further,
     some issues are reported by open source users on GitHub and these
     issues may be tracked using multiple issue numbers.} \\ \hline
     \rowcolor{XyceDarkBlue} \color{white}\textbf{Defect} & \color{white}\textbf{Description} \\ \hline
     \endfirsthead
     \caption[]{Fixed Defects.  Note that we have two multiple issue tracking systems for Sandia Users.
     SON and SRN refer to our legacy open- and restricted-network Bugzilla system, and Gitlab refers to issues in our gitlab repositories.  } \\ \hline
     \rowcolor{XyceDarkBlue} \color{white}\textbf{Defect} & \color{white}\textbf{Description} \\ \hline
     \endhead

  \textbf{Xyce Project Backlog/xxx}: Desciption
  &  Details
  \\\hline

\textbf{Xyce Backlog Bugs/64}: LTRA device does not properly initialize history vector when UIC/NOOP is used &
  The lossy transmission line device (LTRA) was not bein intialized correctly,
  for transient simulations that skipped the DCOP calculation.  As a result,
  transient simulations using this device would crash if the netlist contained
  a \texttt{UIC} or \texttt{NOOP} keyword on the \texttt{.TRAN} line.  This has
  been fixed.  \\ \hline

  \textbf{Xyce Project Backlog/571}: Modify \Xyce{} parser so that it can
  handle multiple \texttt{.options} statements for the same component (for
  example, support \texttt{.options timeint} statements) &
  Most SPICE-style simulators use the \texttt{.option} (non-plural) command,
  and allow the netlist to have multiple \texttt{.option} comamnds.   \Xyce{}
  historically was different in two respects.  One difference is that \Xyce{}
  uses a plural \texttt{.options} command, and another difference is a required
  keyword to identify the part of the code to which the block of options is
  applied.  Also, finally, \Xyce{} was designed to only have a single
  \texttt{.options} statement of each type.  So, for example, at most one
  \texttt{.options device} statement, at most one \texttt{.options timeint}
  statement, etc.  Unforutunately, recent versions of \Xyce{} silently accepted
  multiple \texttt{.options} statements, and behaved differently depending on
  the keyword.  For example, if the netlist contained multiple \texttt{.options
  timeint} statements, \Xyce{} would only use one of them and silently ignore
  the others.  In contrast, if the netlist contained multiple \texttt{.options
  device} statements, it would use them all, but would ignore duplicate
  parameters.  This was confusing for users of other simulators, and has been
  fixed.  \Xyce{} now treats multiple statements as a combined single
  statement, and issues warnings for duplicate parameters.
  \\\hline
 
 \textbf{Xyce Project Backlog/609}: 
  Global parameters that are applied to subcircuit instance parameters don't work in parallel & 
  If a subcircuit instance parameter (on the \texttt{X} line) depended on a
  \texttt{.param} or \texttt{.global\_param}, and that parameter was a
  variable, this dependency worked in serial, but not parallel.  In this case,
  ``variable'' means parameters that are allowed to change as part of a
  \texttt{.STEP} sweep, or as part of a sampling method.  This was related to
  the order of operations in the parser.  This has been fixed.  \\ \hline

 \textbf{Xyce Project Backlog/631}: 
  Expression-based local variation doesn't always work with subcircuit arguments & 
  When expression-based local variation was applied to a subcircuit argument, it 
  was handled incorrectly.  This meant that subcircuit arguments that should have 
  had a random distribution were fixed to the mean value during sampling.  
  This has been fixed.  \\ \hline

  \textbf{Xyce Project Backlog/637}: Support shared library build of \Xyce{} under Windows.
  &  Modify the cmake build system to support building a shared library version
  of xyce, \texttt{xyce-share.dll}, so that Xyce functionality can be accessed
  as a library from other applications.  \\\hline


\end{longtable}
}


\newpage

\section{Known Defects and Workarounds}
% Sandia National Laboratories is a multimission laboratory managed and
% operated by National Technology & Engineering Solutions of Sandia, LLC, a
% wholly owned subsidiary of Honeywell International Inc., for the U.S.
% Department of Energy’s National Nuclear Security Administration under
% contract DE-NA0003525.

% Copyright 2002-2019 National Technology & Engineering Solutions of Sandia,
% LLC (NTESS).


%%
%% Known defects and workarounds.
%%
%% This section should highlight significant defects that were not fixed in
%% the release

\small

\begin{longtable}[h] {>{\raggedright\small}m{2in}|>{\raggedright\let\\\tabularnewline\small}m{3.5in}}
  \caption{Known Defects and Workarounds.} \\ \hline
  \rowcolor{XyceDarkBlue} \color{white}\bf Defect & \color{white}\bf Description
  \\ \hline \endfirsthead
  \caption[]{Known Defects and Workarounds.} \\ \hline
  \rowcolor{XyceDarkBlue} \color{white}\bf Defect & \color{white}\bf Description
  \\ \hline \endhead
% EXAMPLE:
%\textbf{bug number-SRN}: bug title & Description of KNOWN BUG THAT HAS NOT BEEN
  %FIXED.

%\textbf{\textit{Workaround}}: Describe how to work around this bug.
%\\ \hline
%
%  
\textbf{27-SON}: Fix handling of .options parameters & When specifying .options
  for a particular package, what gets applied as the non-specified default
  options might change.  \\ \hline


\textbf{37-SON}: Connectivity checking is broken for devices with more than 10
leads & The diagnostic code used by the \Xyce{} setup that checks circuit
topology for basic errors such as a node having no DC path to ground or a node
being connected to only one device has a bug in it that causes the code to emit
a cryptic error message, after which the code will exit.  This error has so far
only been seen when a user has attempted to connect a large number of inductors
together using multiple mutual inductor lines.  The maximum number of
non-ground leads that can be used without confusing this piece of code is 10.
If your circuit has that type of large, highly-connected mutual inductor and
the code exits with an error message, this bug may be the source of the problem.

The error message now includes a recommendation to use the workaround below.

\textbf{\textit{Workaround}}: Disable connectivity checking by adding the line

\begin{alltt} .OPTIONS TOPOLOGY CHECK_CONNECTIVITY=0 \end{alltt}

to your netlist.  This will disable the check for the basic errors such as
floating nodes and improperly connected devices, but will allow the netlist to
run with a highly-connected mutual inductor.
\\ \hline


\textbf{49-SON} \Xyce{} BSIM models recognize the model TNOM, but not the
instance TNOM & 
\\ \hline



\textbf{247-SON}: Expressions don't work on .options lines & Expressions enclosed
in braces (\{ \}) are handled specially throughout \Xyce{}, and may only be used
in certain contexts such as in device model or instance parameters or on
\texttt{.PRINT} lines.
\\ \hline


\textbf{250-SON}: NODESET in \Xyce{} is not equivalent to NODESET in SPICE & As
currently implemented, .NODESET applies the initial conditions given throughout
a full nonlinear solve for the operating point. It then uses that result as an
initial guess for a second nonlinear solve with no constraints.  This is not
the same as SPICE, which merely applies the given initial conditions to a
single nonlinear solve for the first two iterations, then lets the problem
converge with no further constraints.  This can lead to Xyce's .NODESET failing
where the same netlist in SPICE might not, if the initial conditions are such
that a full nonlinear solve cannot converge with those constraints in place.
There is no workaround.
\\ \hline


\textbf{468-SON}: It should be legal to have two model cards with the same model
name, but different model types. & SPICE3F5 and ngspice only require that
model cards of the same type have unique model names. They accept model cards
of different types with the same name.  \Xyce{} requires that all model card names be unique.
\\ \hline


\textbf{469-SON}: Belos memory consumption on FreeBSD and excessive CPU on other
platforms &
Memory or thread bloat can result when using multithreaded dense linear algebra
libraries, which are employed by Belos.  If this situation is observed, either build
\Xyce{} with a serial dense linear algebra library or use environment variables to control
the number of spawned threads in a multithreaded library.
\\ \hline


\textbf{583-SON}: Switch with RON=0 leads to convergence failure. &  The switch device does not prevent a user from specifying \texttt{RON=0} in its model, but then takes the inverse of this value to get the ``on'' conductance.  The resulting invalid division will either lead to a division by zero error on platforms that throw such errors, or produce a conductance with ``Not A Number'' or ``Infinity'' as value.  This will lead to a convergence failure.

\textbf{\textit{Workaround}}: Do not specify an identically zero resistance for the switch's ``on'' value.  A small value of resistance such as 1e-15 or smaller will generally work well as a substitute. \\ \hline

\textbf{620-SON}: TO, FROM and TD Qualifiers Function Incorrectly in the FREQ Measure &
The use of these qualifiers in a {\tt FREQ} measure statement may give an incorrect
measurement window.  This is noted in the Reference Guide. \\ \hline

\textbf{621-SON}: FIND-WHEN Measure in \Xyce{} Only Correctly Supports the WHEN Syntax. 
FIND-WHEN is incorrect. & An example is given in the HSpice Compatibility subsection of
the {\tt .MEASURE} section of the Reference Guide. \\ \hline

\textbf{1595-SRN}: \Xyce{} won't allow access to inductors within subcircuits for
mutual inductors external to subcircuits & It is not possible to have a mutual
inductor outside of a subcircuit couple to inductors in a subcircuit.

\textbf{\textit{Workaround}}: Put all inductors and mutual inductance lines that couple to
them together at the same level of circuit hierarchy.
\\ \hline


\textbf{1903-SRN}: \Xyce{} fails to collect several inductors into a linear mutual inductor &
In some rare cases with complex include file usage, the mutual inductor syntax with multiple couplings can fail to work. \Xyce{} will return an error message that it can not find 
\texttt{L\_L1} in this example:
{\tt
\begin{verbatim}
L_L1         node1 node1 inductance1
L_L2         node3 node4  inductance2
L_L3         node5  node6  inductance3
L_L4         node7 node8  inductance4
K_K1         L_L1 L_L2 L_L3 L_L4   .999 
\end{verbatim}
}

\\ \hline


\textbf{1923-SRN}: LC lines run out of memory, even if equivalent (larger) RLC
lines do not. &  In some cases, circuits that run fine using an RLC approximation for a
transmission line, exit with an out-of-memory error if the (supposedly smaller) LC
approximation is used.
\\ \hline

\end{longtable}


\newpage

\section{Resolved Incompatibilities With Other Circuit Simulators}
Table ~\ref{incomp} lists incompatibilities between Xyce{} and other circuit
simulators that were resolved in this release.  Please consult the Xyce{} Reference
Guide for a list of known incompatibilities, and how to work around them.
% Sandia National Laboratories is a multimission laboratory managed and
% operated by National Technology & Engineering Solutions of Sandia, LLC, a
% wholly owned subsidiary of Honeywell International Inc., for the U.S.
% Department of Energy’s National Nuclear Security Administration under
% contract DE-NA0003525.

% Copyright 2002-2019 National Technology & Engineering Solutions of Sandia,
% LLC (NTESS).


%%
%% Incompatibilities with other circuit simulators.
%%
\small

\begin{longtable}[h,t,b,p] {|
>{\setlength{\hsize}{0.40\hsize}}Y|
>{\setlength{\hsize}{0.60\hsize}}Y|} 

\caption{Incompatibilities with other circuit simulators.} \\ \hline

\rowcolor{XyceDarkBlue}
\color{white}\bf Issue &
\color{white}\bf Comment
\\ \hline
\endhead

    Pulsed source rise time of zero. & A requested pulsed source rise/fall
    time of zero really is zero in Xyce.  In other simulators, requesting a
    zero rise/fall time causes them to use the printing interval found on
    the \texttt{.TRAN} line.\\ \hline

    Mutual Inductor Model. & Not the same as PSpice.  This is a Jiles-Atherton non-linear
    model developed at Sandia. It is compatible with Cadence PSpice parameter set.\\ \hline

    \texttt{.PRINT} line shorthand. & Output variables have to be specified
    as V(node) or I(source).  Specifying the node alone will not work.  \\ \hline

    BSIM3 level. & In \Xyce{} the BSIM3 is MOSEFET level 9.  Other simulators have
    different levels for the BSIM3. \\ \hline

    BSIM SOI v3.2 level. & In \Xyce{} the BSIM SOI (v3.2) is MOSFET level 10.  
    Other simulators have different levels for the BSIM SOI. \\ \hline

    BSIM4 level. & In \Xyce{} the BSIM4 is MOSFET levels 14 and 54.  
    Other simulators have different levels for the BSIM4. \\ \hline

    Interactive mode. & \Xyce{} does not have an interactive mode. \\ \hline

    Syntax for \texttt{.STEP} is different. & The manner of specifying
    a model parameter to be swept is slightly different than in some
    other simulators.  See the Users' and Reference Guides for
    details.  \\ \hline

    Switch is not the same as SPICE. &  The \Xyce{} switches are not
	compatible with the simple switch implementation in SPICE3F5.  The
	switch in \Xyce{} smoothly transitions between the ON and OFF
	resistances over a small range between the ON and OFF values of the
	control signal (voltage, current, or control expression).  See the
	Reference Guide for the precise equations that are used to compute the
	switch resistance from the control signal values.  The SPICE3F5 switch
	has a single switching threshold voltage or current, and RON is used
	above threshold while ROFF is used below threshold.  \Xyce{}'s switch is
	considerably less likely to cause transient simulation failures. Results
	similar to SPICE3F5 can be obtained by setting VON and VOFF to the same
	threshold value, but this is not a recommended practice.  \\ \hline

\end{longtable}



\newpage
\section{Important Changes to \Xyce{} Usage Since the Release 6.1.}
Table ~\ref{newUsage} lists some usage changes for \Xyce{}.
% Sandia National Laboratories is a multimission laboratory managed and
% operated by National Technology & Engineering Solutions of Sandia, LLC, a
% wholly owned subsidiary of Honeywell International Inc., for the U.S.
% Department of Energy’s National Nuclear Security Administration under
% contract DE-NA0003525.

% Copyright 2002-2021 National Technology & Engineering Solutions of Sandia,
% LLC (NTESS).

% Sandia National Laboratories is a multimission laboratory managed and
% operated by National Technology & Engineering Solutions of Sandia, LLC, a
% wholly owned subsidiary of Honeywell International Inc., for the U.S.
% Department of Energy’s National Nuclear Security Administration under
% contract DE-NA0003525.

% Copyright 2002-2021 National Technology & Engineering Solutions of Sandia,
% LLC (NTESS).


%%
%% Changes to Xyce input since the last release.
%%
{
\small

\begin{longtable}[h] {>{\raggedright\small}m{2in}|>{\raggedright\let\\\tabularnewline\small}m{3.5in}}
  \caption{Changes to netlist specification since the last release.\label{newUsage}} \\ \hline
  \rowcolor{XyceDarkBlue}
  \color{white}\textbf{Change} &
  \color{white}\textbf{Detail} \\ \hline \endfirsthead
  \caption[]{Changes to netlist specification since the last release.\label{newUsage}} \\ \hline
  \rowcolor{XyceDarkBlue}
  \color{white}\textbf{Change} &
  \color{white}\textbf{Detail} \\ \hline \endhead

Continued support for FRAC\_MAX qualifier on TRIG-TARG measure lines &
For backwards compatibility with previous \Xyce{} versions for internal
users, \texttt{.OPTIONS MEASURE USE\_LTTM=<value>} has been added.  This
option defaults to 0, which uses the new version of the TRIG-TARG measure;
while setting it to 1 will use the old version of the TRIG-TARG measure
for all TRIG-TARG measures in the netlist.  If the FRAC\_MAX qualifier is
used on a TRIG-TARG line then \Xyce{} will automatically default to
\texttt{USE\_LTTM=1} for that particular measure line.  \\ \hline

\textbf{Gitlab-ex issue 338}: Added the function \texttt{Simulator:: getTimeVoltagePairsSz} & The function
\texttt{Simulator::getTimeVoltagePairsSz} supplies the maximun 
number of time, voltage or state values for an ADC in a subsequent call to 
\texttt{Simulator::getTimeVoltagePairs} or \texttt{Simulator::getTimeStatePairs}.
Also, calling \texttt{Simulator::getTimeStatePairs} no longer clears the voltage
history of an ADC, so a calling program that needs both state and voltage history
can first call \texttt{Simulator::getTimeStatePairs} and then call 
\texttt{Simulator::getTimeVoltagePairs} to get both.  Calling 
\texttt{Simulator::getTimeVoltagePairs} still clears the ADC history.
See the Application note, Mixed Signal Simulation with \Xyce{} 7.5 for further 
details. \\ \hline

\textbf{Gitlab-ex issue 348}:  Added the function \texttt{xyce\_getTimeVoltagePairs\-ADCLimitData()}  &
  A new function has been added to the \texttt{XyceCInterface} called 
  \texttt{xyce\_getTimeVoltagePairsADCLimitData()} which limits the data copied 
  to the caller allocated space to whatever maximum 
  allocation length is provided.  This avoids potential memory overwriting that
  could occur with the general access function \texttt{xyce\_getTimeVoltagePairsADC()}.
  See the Application note, Mixed Signal Simulation with \Xyce{} 7.5 for further details.  \\ \hline

\end{longtable}
}


\clearpage
% acknowledgements, trademarks, contact information.
% Sandia National Laboratories is a multimission laboratory managed and
% operated by National Technology & Engineering Solutions of Sandia, LLC, a
% wholly owned subsidiary of Honeywell International Inc., for the U.S.
% Department of Energy’s National Nuclear Security Administration under
% contract DE-NA0003525.

% Copyright 2002-2019 National Technology & Engineering Solutions of Sandia,
% LLC (NTESS).


%% Acknowledgements, Trademarks, and contact information for the Xyce
%% project.
\HeadingC{Acknowledgements}
The authors would like to acknowledge the entire Sandia National
Laboratories electrical modeling community, for their support 
on this project.  This includes, but is not limited to, Bill Ballard,
Dave Shirley, Carolyn Bogdan, Chuck Hembree, 
Biliana Paskeleva, Ken Kambour, Brian Owens, and Nathan Nowlin.

\HeadingC{Trademarks}
The information herein is subject to change without notice.\\[0.5em]
Copyright \copyright{} 2002-2014 Sandia Corporation.  All rights
reserved.\\
\XyceTM{} Electronic Simulator and \XyceTM{} trademarks of Sandia
Corporation.\\
Portions of the \XyceTM{} code are:  \\
Copyright \copyright{} 2002, The Regents of the University of California. \\
Produced at the Lawrence Livermore National Laboratory. \\
Written by Alan Hindmarsh, Allan Taylor, Radu Serban. \\
UCRL-CODE-2002-59 \\
All rights reserved. \\[0.5em]
ModelSim is a registered trademark of Mentor Graphics, Inc.\\[0.5em]
Orcad, Orcad Capture, PSpice and Probe are registered trademarks of Cadence Design Systems, Inc.\\[0.5em]
HSpice is a registered trademarks of Synopsys, Inc.\\[0.5em]
Microsoft, Windows and Windows 7 are registered trademark of Microsoft
Corporation.\\[0.5em]
\Xyce{}'s expression library is based on that inside Spice 3F5 developed by
the EECS Department at the University of California. \\[0.5em]
The EKV3 MOSFET model was developed by the EKV Team of the Electronics Laboratory-TUC of the Technical University of Crete. \\[0.5em]
All other trademarks are property of their respective owners.
\HeadingC{Contacts} \label{Contacts}
Bug Reports \hfill \texttt{\color{XyceDeepRed}http://charleston.sandia.gov/bugzilla} \\
Email \hfill \texttt{\color{XyceDeepRed}xyce-support@sandia.gov} \\
World Wide Web \hfill \texttt{\color{XyceDeepRed}http://xyce.sandia.gov/}

\vspace*{\fill}
\includegraphics[height=0.5in]{../Common_Guide_Files/xyce_flat_white}
\hfill
\includegraphics[height=0.5in]{../Common_Guide_Files/snllineblubrd}


\begin{SANDdistribution}[NM]% or [CA]
    % \SANDdistCRADA	% If this report is about CRADA work
    % \SANDdistPatent	% If this report has a Patent Caution or Patent Interest
    % \SANDdistLDRD	% If this report is about LDRD work

    % External Address Format: {num copies}{Address}
%    \SANDdistExternal{}{}
    \bigskip

    % The following MUST BE between the external and internal distributions!
    % \SANDdistClassified % If this report is classified

    % Internal Address Format: {num copies}{Mail stop}{Name}{Org}
%    \SANDdistInternal{}{}{}{}

    % Mail Channel Address Format: {num copies}{Mail Channel}{Name}{Org}
%    \SANDdistInternalM{}{}{}{}
\end{SANDdistribution}


\end{document}
