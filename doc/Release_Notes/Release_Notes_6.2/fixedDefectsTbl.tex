% Sandia National Laboratories is a multimission laboratory managed and
% operated by National Technology & Engineering Solutions of Sandia, LLC, a
% wholly owned subsidiary of Honeywell International Inc., for the U.S.
% Department of Energy’s National Nuclear Security Administration under
% contract DE-NA0003525.

% Copyright 2002-2019 National Technology & Engineering Solutions of Sandia,
% LLC (NTESS).



%%
%% Fixed Defects.
%%
\small

\begin{longtable}[h] {>{\raggedright\small}m{2in}|>{\raggedright\let\\\tabularnewline\small}m{3.5in}}
  \caption{Fixed Defects.  Note that we have two different Bugzilla systems for Sandia users.
  SON, which is on the open network, and SRN, which is on the restricted network. } \\ \hline
  \rowcolor{XyceDarkBlue} \color{white}\bf Defect & \color{white}\bf Description \\ \hline \endfirsthead
  \caption[]{Fixed Defects.  Note that we have two different Bugzilla systems for Sandia Users.
  SON, which is on the open network, and SRN, which is on the restricted network. } \\ \hline
  \rowcolor{XyceDarkBlue} \color{white}\bf Defect & \color{white}\bf Description \\ \hline \endhead

%\textbf{bug xxxx }: Describe issue & Describe fix. \\ \hline 

% joseki (-SON)bugs:
%% \textbf{BUG NUMBER-SON}: Bug title & Bug and fix description \\ \hline
\textbf{351-SON}: .MEASURE console output by every processor in parallel. &
When .MEASURE is used in parallel, every processor outputs to the console.  
However only the root processor's output might appear in the 
"***Measure functions***" part of the output file.  The rest of the .MEASURE output
could be scattered around and mixed with the execution statistics. 

\textbf{\textit{Solution}}: Only the root processor outputs the .MEASURE results.
\\ \hline 

\textbf{368-SON}: Fix and document the REPEAT feature of the PWL transient
specification  &  The repeat (\texttt{R}) and time delay (\texttt{TD})
features of the Piecewise Linear voltage or current source, PWL, 
were undocumented and did not work correctly.

\textbf{\textit{Solution}}: This is fixed but partially incompatible with
HSpice syntax since \texttt{R} (without the \texttt{=0}), to mean repeat the
waveform starting at time=0, is legal in HSpice.  \Xyce{} requires \texttt{R=0},
in that case. The \Xyce{} syntax for PWL sources is not compatible with
PSpice's REPEAT syntax.
\\ \hline 

\textbf{380-SON}: Measure statements need to support expressions &
The variable that .measure used in a calculation had to be a solution
or state/store variable.  Expressions, such as \texttt{max(v(node1)-v(node2))}
were not allowed.  

\textbf{\textit{Solution}}: This is fixed.
\\ \hline

\textbf{384-SON}: Errors compiling \Xyce{} with Bison 3.0.  &  Bison 3.0
deprecated access to identifiers that \Xyce{} used for error reporting.

\textbf{\textit{Solution}}: This is fixed and \Xyce{} should compile with
either Bison 2.7 or Bison 3.0.x.
\\ \hline

\textbf{427-SON}: Certain subcircuit nodes cannot be accessed in expressions &
\Xyce{} allows nodes within a subcircuit to be referenced in expressions and on
.print lines using the syntax ``$<$subcircuit instance
name$>$:$<$nodename$>$''.  This works on .print lines even if the node named is
one that is on the subcircuit declaration line (i.e., dummy argument nodes,
which are just aliases for names in the calling circuit context).  This does
not work inside expressions.

\textbf{\textit{Solution}}: These values are now accessible within expressions.
\\ \hline

\textbf{438-SON}: .step output for HB in Standard (.prn) format does not reset
Index for each step or otherwise mark new step data &  It is difficult to separate the results, 
in an .HB.FD.prn file, for each step because the Index column is monotonic instead of resetting
to zero at the beginning of each step, as in transient or DC analysis runs. 

\textbf{\textit{Solution}}: The index is now reset at the start of each .STEP.
\\ \hline


\textbf{439-SON}: Interpolation does not work well with TRAP and GEAR &
The problem occurs in certain situations when the output is interpolated. For
Trap, the problem can occur when the 2nd order Trap is used. For Gear, the
problem occurs when the lead currents are interpolated.

\textbf{\textit{Solution}}: This is fixed.
\\ \hline

\textbf{441-SON}: \texttt{.AC} output for step-tecplot is wrong. &  When
using STD output, \Xyce{} would create both \texttt{netlist.cir.FD.prn} and 
\texttt{netlist.cir.prn} files.  When using tecplot output, \Xyce{} would
only create one file, \texttt{netlist.cir.dat}.

\textbf{\textit{Solution}}: This is fixed.
\\ \hline

\textbf{442-SON}: lead currents are not correctly output for HB analysis &

The lead currents were not calculated in HB analysis. The output would be
all zeros for the lead currents in HB analysis.

\textbf{\textit{Solution}}: This is fixed.
\\ \hline

\textbf{483-SON}: ASCII rawfile output does not work with .dc without the
-r flag. &  Running \texttt{Xyce -a <filename>} resulted in a file in
binary format, rather than ASCII. However, using both the \texttt{-r <rawfile>}
and \texttt{-a} flags resulted in the \texttt{<rawfile>} being in ASCII format.

\textbf{\textit{Solution}}: This is fixed.
\\ \hline

\textbf{484-SON}: Putting resistor parameter on \texttt{.PRINT} line, with
homotopy simulation causes segfault &  Including \texttt{RC:R} on the 
\texttt{.PRINT} line caused \Xyce{} to seg-fault.

\textbf{\textit{Solution}}: This is fixed.
\\ \hline

\textbf{487-SON}: \texttt{.OP} with PDE devices gives a seg fault & In some 
circuits with PDE devices, \texttt{.OP} would cause \Xyce{} to crash.

\textbf{\textit{Solution}}: This is fixed.
\\ \hline

\textbf{488-SON}: HB-specific output should be excluded from the *.hb\_ic.prn
file & If the following HB print line \texttt{.print HB VDB(1)} was used
then the output files associated with the initial condition (which are
the transient output files) would print \texttt{-inf} for VDB.  This caused problems
with Tecplot. 

\textbf{\textit{Solution}}: This can be resolved by by adding an 
\texttt{HB\_IC} output specifier.  Please see the \Xyce{} Reference Guide,
Version 6.2 for more details.
\\ \hline

\textbf{493-SON}: Ill-formed Source Definitions Cause a Core Dump & A source
definition without any parameters caused a core dump.  An example instance
line was: \texttt{V1 1 0 sin}

\textbf{\textit{Solution}}: This is fixed.
\\ \hline

\textbf{494-SON}: Remove metacharacters (\%) in the specification of 
devices and variables & \Xyce{} used \% as a white space filler for various
internal variables.  So, the use of \% in device names (which is legal in
other SPICE-like circuit simulators) would have unpredictable results in
\Xyce{}.

\textbf{\textit{Solution}}: \% is now a legal character in a \Xyce{} device
names.
\\ \hline

\textbf{496-SON}: Calling RAND() from within a user defined function causes a 
segmentation fault & An external user reported that using \texttt{RAND()} in
a user defined function caused \Xyce{} to crash with a segmentation fault. 

\textbf{\textit{Solution}}: This is fixed.
\\ \hline

\textbf{497-SON}: FILE keyword in voltage/current source PWL function doesn't
work in lower case. &  The Piecewise Linear function for a voltage or current
source, PWL, can read its data from a file with the syntax: \texttt{ImySource
 1 0 PWL FILE "mydata.dat"}.  If the FILE keyword was in lower case text as in
"file" then \Xyce{} emitted an error.

\textbf{\textit{Solution}}: This is fixed in the parser.
\\ \hline 

\textbf{507-SON}: Parser bottleneck for .PRINT lines with many arguments &
An external user wanted to output the lead current through every resistor in
a very large (more than 1M device) circuit. This caused \Xyce{} to get
stuck during the .PRINT line processing.  

\textbf{\textit{Solution}}: This is fixed in the parser. 
\\ \hline

\textbf{511-SON}: Remove Unused .OUTPUT feature. &  .output was obsolete code
that was not documented in the \Xyce{} User Guide or Reference Guide.  It
was the predecessor of the currently documented mechanism for specifying
output intervals, .options output. 

\textbf{\textit{Solution}}: This was feature was removed from the parser.  
Consult the \Xyce{} Reference Guide, Version 6.2 for the syntax for .options
output instead.
\\ \hline 

\textbf{518-SON}: Non-Linear Mutual Inductor should cleanly handle
magnetic moment equation & The non-linear mutual inductor adds an equation
for the magnetic moment of the device to the system of equations.  If the user
specifies model parameters where the gap is zero and the domain flexing
parameter is small, (e.g., \texttt{C $\ll$ 0.1}), then the magnetic moment
equation becomes very similar to the sum of the current derivatives equation
(R(t) in \Xyce{}'s formulation).  This can cause the simulation to be unstable.

\textbf{\textit{Solution}}:  The non-linear mutual inductor drops the magnetic
moment equation from the system of equations it adds when the domain flexing
parameter (\texttt{C}) is below a user settable limit (\texttt{CLIM}). The
user can recover \Xyce{}'s previous behavior by setting \texttt{CLIM} to zero.
The defaults are \texttt{CLIM=0.005} and \texttt{C=0.2}.
\\ \hline

\textbf{519-SON}: Raw file outputs zero for resistor lead current & An open
source customer reported that \Xyce{} was improperly outputting lead currents in the rawfile
format.  When run with "Xyce -a netlist", the ascii rawfile always showed I(R1)= zero.
Replacing the .print line with a default format (removing FORMAT=RAW and
FILE=foobar.raw) produced a standard columnar output file with the correct lead
current in it.

\textbf{\textit{Solution}}: This is fixed.
\\ \hline

\textbf{1788-SRN}: Make level 57 synonym for level 10 mosfet & This is a
compatibility request to better accomodate SmartSpice and HSpice users. 

\textbf{\textit{Solution}}:  \Xyce{} provide partial compatibility that is
adequate for internal Sandia users.  In most SPICE-like codes, 
\texttt{level=57} is the BSIM4 SOI.  In \Xyce{}, \texttt{level=57} is the
BSIM3 SOI. 
\\ \hline

\textbf{1922-SRN}: \Xyce{} nonlinear core model defaults seriously underestimate
hysteresis effects for small signals & The magnetic core model has some
parameter defaults that effectively turn off magnetic effects when
small-amplitude driving signals are applied.  

\textbf{\textit{Solution}}: The performance of the \Xyce{} non-linear 
mutual inductor model has been improved for small-signal inputs.  Please
consult the Inductor section of the \Xyce{} Reference Guide for more
details on the usage of the \texttt{DEVSCALING} and 
\texttt{CONSTANTDEVSCALING} parameters.
\\ \hline

\end{longtable}

