% Sandia National Laboratories is a multimission laboratory managed and
% operated by National Technology & Engineering Solutions of Sandia, LLC, a
% wholly owned subsidiary of Honeywell International Inc., for the U.S.
% Department of Energy's National Nuclear Security Administration under
% contract DE-NA0003525.

% Copyright 2002-2025 National Technology & Engineering Solutions of Sandia,
% LLC (NTESS).

%%
%% Fixed Defects.
%%
{
\small

\begin{longtable}[h] {>{\raggedright\small}m{2in}|>{\raggedright\let\\\tabularnewline\small}m{3.5in}}
    \caption{Fixed Defects.  The Xyce team has multiple issue
     trackers, and the table below indicates fixed issues by
     indentifying both the tracker and the issue number.  Further,
     some issues are reported by open source users on GitHub and these
     issues may be tracked using multiple issue numbers.} \\ \hline
     \rowcolor{XyceDarkBlue} \color{white}\textbf{Defect} & \color{white}\textbf{Description} \\ \hline
     \endfirsthead
     \caption[]{Fixed Defects.  Note that we have two multiple issue tracking systems for Sandia Users.
     SON and SRN refer to our legacy open- and restricted-network Bugzilla system, and Gitlab refers to issues in our gitlab repositories.  } \\ \hline
     \rowcolor{XyceDarkBlue} \color{white}\textbf{Defect} & \color{white}\textbf{Description} \\ \hline
     \endhead

\textbf{Xyce Project Backlog/613}: 
  Noise  comma-separated values (CSV) output format should put quotes around variable/column names
  & Comma-separated values (CSV) output for noise had a bug, in that the comma character was used as a separator between variables, but for noise analysis variable names often include commas.  This has been fixed in two ways.  One is to add quotes around variable names in the CSV noise output file header.  The other is to expand the \texttt{DELIMITER} parameter on the \texttt{PRINT} line to allow other options, including \texttt{SEMICOLON} and \texttt{COLON}.  Also, this parameter previously applied only to \texttt{STD} format files.  It now also applies to \texttt{CSV} format files.  \\ \hline

\textbf{Xyce Project Backlog/694}: 
Error in MOSFET numerical sensitivities, when using \texttt{.option scale}
  & When using sensitivity analysis, one aspect of the calculation is device-level derivatives.  For some device models, analytic derivatives are available.  When they are not available, numerical device-level derivatives must be used.  This is the case with several older MOSFET models, such as the BSIM4.  The code for computing these device-level derivatives did not handle length scalars (set by the netlist command \texttt{.option scale} properly, so this produced at best nonsense results and at worst caused the code to exit with a fatal error.  This has been fixed. \\ \hline

\textbf{Xyce Project Backlog/729}: 
Expression-based objective functions for .SENS conflate I(VA) with V(VA) (nodes and devices w/ same names)
& Sensitivity analysis had a bug related to the naming of circuit variables in the objective function.
If a circuit had a voltage source and a voltage node of the same name, the expression capability could not differentiate between the two, and could choose the wrong one.  This has been fixed. \\ \hline

\textbf{Xyce Project Backlog/740}: Adjoint, output, and user-provided breakpoints present a scaling issue for
extremely long lists & A user can request output using \texttt{.options output}
in the form of a list of time points.  If this list was very long (10k+ values),
then processing the list would result in a substantial slowdown in the setup time of the 
simulation.  This has been fixed. \\ \hline

\textbf{Xyce Project Backlog/767}: Location of device model repositories 
should be optional & When building Xyce from source with CMake, the 
location of optional device model packages can now be specifed as 
configuration parameters. The new parameter options and their 
associated default values are given below as \texttt{parameter name} 
\texttt{(default value)}. \newline
\texttt{Xyce\_ADMS\_MODELS\_DIR} \texttt{(\$\{PROJECT\_SOURCE\_DIR\}/src/DeviceModelPKG/ADMS)} \newline
\texttt{Xyce\_NEURON\_MODELS\_DIR} \texttt{(\$\{PROJECT\_SOURCE\_DIR\}/src/DeviceModelPKG/NeuronModels)} \newline
\texttt{Xyce\_NONFREE\_MODELS\_DIR} \texttt{(\$\{PROJECT\_SOURCE\_DIR\}/src/DeviceModelPKG/Xyce\_NonFree)} \newline
\texttt{Xyce\_RAD\_MODELS\_DIR} \texttt{(\$\{PROJECT\_SOURCE\_DIR\}/src/DeviceModelPKG/SandiaModels)} \newline
\\ \hline

\textbf{Xyce Project Backlog/752}: Enable Hysteresis in Switch Devices.
&  Hysteresis in the on and off behavior of voltage, current and general 
expression based switches has been added to the switch device.  See the 
reference guide for further details. 
\\\hline


\end{longtable}
}
