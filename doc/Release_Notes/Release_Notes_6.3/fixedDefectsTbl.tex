% Sandia National Laboratories is a multimission laboratory managed and
% operated by National Technology & Engineering Solutions of Sandia, LLC, a
% wholly owned subsidiary of Honeywell International Inc., for the U.S.
% Department of Energy’s National Nuclear Security Administration under
% contract DE-NA0003525.

% Copyright 2002-2019 National Technology & Engineering Solutions of Sandia,
% LLC (NTESS).


%%
%% Fixed Defects.
%%
\small

\begin{longtable}[h] {>{\raggedright\small}m{2in}|>{\raggedright\let\\\tabularnewline\small}m{3.5in}}
  \caption{Fixed Defects.  Note that we have two different Bugzilla systems for Sandia users.
  SON, which is on the open network, and SRN, which is on the restricted network. } \\ \hline
  \rowcolor{XyceDarkBlue} \color{white}\bf Defect & \color{white}\bf Description \\ \hline \endfirsthead
  \caption[]{Fixed Defects.  Note that we have two different Bugzilla systems for Sandia Users.
  SON, which is on the open network, and SRN, which is on the restricted network. } \\ \hline
  \rowcolor{XyceDarkBlue} \color{white}\bf Defect & \color{white}\bf Description \\ \hline \endhead

%\textbf{bug xxxx }: Describe issue & Describe fix. \\ \hline 

% joseki (-SON)bugs:
%% \textbf{BUG NUMBER-SON}: Bug title & Bug and fix description \\ \hline
\textbf{244-SON}: A handful of parameter defaults in the BSIM SOI v3.2 device were incorrect & The \texttt{CGDO}, \texttt{CGSO}, \texttt{K1} and \texttt{K2} parameters had incorrect defaults.  The parameter defaults have been updated to match the correct values.
\\ \hline 

\textbf{277-SON}: Create infrastructure for doing true SPICE-like ``source stepping'' in \Xyce{} &   When doing a DC operating point solve in SPICE, it first attempts standard Newton's method to solve the problem.  If that fails, it attempts GMIN stepping, and if that fails it attempts source stepping (scaling all DC sources in the circuit iteratively from 0 to their nominal values).  Until release 6.3, \Xyce{} had not implemented source stepping in its own solution strategy.  \Xyce{} now attempts source stepping on DC solves if Newton and GMIN stepping both fail.
\\ \hline 

\textbf{359-SON}: scaleParam function doesn't work properly &   \Xyce{} 6.2 supported a ``VSRCSCALE'' parameter that could have been used to simulate SPICE's ``source stepping'' technique, but attempting to use this parameter in continuation didn't apply it properly.  As of release 6.3, the VSRCSCALE continuation parameter is now properly applied to all DC sources as expected.
\\ \hline 

\textbf{466-SON}: .RESULT output doesn't recognize .param parameters &  As an
example, consider: 
{\tt
\begin{verbatim}
.param par1=1.0 
.param par2=2.0 
.result {par1/par2} 
\end{verbatim}
}
In prior releases, the code would run for a while and then when the first .RESULT output was
processed (at the end of the first .STEP iteration) it would exit with an error,
claiming that par1 and par2 are not recognized variables.  This has been fixed in Release 6.3.
\\ \hline

\textbf{477-SON}: Internal variables are not accessible inside expressions &
An internal variable, accessed via n(…), was not properly resolved in an
expression. As an example, this failed parsing in prior releases: 
{\tt
\begin{verbatim}
.print tran {n(yneuron!neuron1_V1)} 
\end{verbatim}
}
\\ \hline

\textbf{534-SON}: .option outputs can print duplicate time points & When .options outputs were used in the netlist, the output could have had duplicate time points depending on the frequency used in the simulation and the output interval specified.  Duplication of output  time points no longer occurs in \Xyce{} 6.3.
\\ \hline

\textbf{544-SON}: Initial condition statement results in failed
simulation & In versions 6.1 and older of \Xyce{}, any .IC statements within a
subcuit were ignored in favor of the value provided from the device
line.  This behavior has been restored in \Xyce{} 6.3. \\ \hline

\textbf{546-SON}: TNOM handled incorrectly by capacitor & The capacitor in \Xyce{} allows specification of the temperature-dependence of the capacitance via the \texttt{TNOM}, \texttt{TC1}, and \texttt{TC2} model parameters.  In versions of \Xyce{} prior to 6.3, the TNOM value was ignored and always replaced with the device package default value (27 degrees C, or the value specified in a \texttt{.OPTIONS DEVICE} line).  As of \Xyce{} 6.3, the capacitor no longer ignores a user-specified \texttt{TNOM} value.
\\ \hline 

\textbf{550-SON}: The DERIVATIVE Measure in \Xyce{} is actually a Slope Measurement Between Two Endpoints &  
This was an incorrect definition of a derivative.  As of \Xyce{} 6.3, 
\Xyce{} now supports calculating the derivative of a waveform at either a 
user-specified time (\texttt {AT=<time>}) or when a user-specified condition occurs 
(\texttt {WHEN v(1)=<value>}). 
\\ \hline
 
\textbf{560-SON}: Too many '.ENDS' does not inform used of location of
problem & \Xyce{} now informs the user of the file and line number where the
extra .ENDS occurs. \\ \hline

\textbf{580-SON}: Supernoding capability broken in \Xyce{} 6.2. &  Supernoding is the ability to remove devices attached to the same node or resistors with zero resistances. This capability was disabled in \Xyce{} 6.2.  It has been fixed for \Xyce{} 6.3, and can be used by specifying \texttt{.options topology supernode=true} in any netlist.\\ \hline

\textbf{605-SON}: MPI hang on exit on some platforms & On a handful of ``build supported'' platforms using OpenMPI 1.8.x, Xyce's method of exit on parsing error sometimes tripped an obscure race condition in MPI, leaving mpirun hanging even though all \Xyce{} processes had exited.  The method of exit in this event was changed to rely only on well-defined behaviors of MPI and this race condition is no longer tripped.
\\ \hline 

\textbf{609-SON}: No-op code in MOSFET6 &   The MOSFET6 code had an extraneous semicolon that rendered part of the \texttt{Von} expression as a no-op.  This semicolon was present in the original SPICE3F5 implementation and was copied over into \Xyce{} when MOS6 was implemented.

\textbf{\textit{Solution}}: The extraneous semicolon was removed and the expression is  no longer a no-op.  This causes the ``sigma'' model parameter (which defaults to zero) to impact the \texttt{Von} expression as it was intended to do (and as it does in other simulators such as ngspice, where this error was corrected).
\\ \hline

\textbf{1957-SRN and 575-SON}: \Xyce{} not handling multiply-coupled mutual inductor correctly &
\texttt{K} lines with more than two inductors were not properly creating the couplings between all inductors as stated in the reference guide.
{\tt
\begin{verbatim}
L1         node1 node1 inductance1
L2         node3 node4  inductance2
L3         node5  node6  inductance3
L4         node7 node8  inductance4
K_K1         L1 L2 L3 L4   .999 
\end{verbatim}
}

should be equivalent to
{\tt
\begin{verbatim}
L1         node1 node1 inductance1
L2         node3 node4  inductance2
L3         node5  node6  inductance3
L4         node7 node8  inductance4
K_K12        L1 L2 .999
K_K13        L1 L3 .999 
K_K14        L1 L4 .999 
K_K23        L2 L3 .999 
K_K24        L2 L4 .999 
K_K34        L3 L4 .999 
\end{verbatim}
}

In all versions of \Xyce{} prior to 6.3, only the snippet with
multiple \texttt{K} lines produced the correct results.  The other
version improperly coupled only coupled L1 to L2 and L3 to L4.
\\ \hline

\textbf{1945-SRN}: Ending subcircuit with .END, not .ENDS, results in an application hang &  An error message now indicates the location of the .END statement. \\ \hline

\end{longtable}

