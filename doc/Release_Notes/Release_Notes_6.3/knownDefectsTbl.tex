% Sandia National Laboratories is a multimission laboratory managed and
% operated by National Technology & Engineering Solutions of Sandia, LLC, a
% wholly owned subsidiary of Honeywell International Inc., for the U.S.
% Department of Energy’s National Nuclear Security Administration under
% contract DE-NA0003525.

% Copyright 2002-2019 National Technology & Engineering Solutions of Sandia,
% LLC (NTESS).


%%
%% Known defects and workarounds.
%%
%% This section should highlight significant defects that were not fixed in
%% the release

\small

\begin{longtable}[h] {>{\raggedright\small}m{2in}|>{\raggedright\let\\\tabularnewline\small}m{3.5in}}
  \caption{Known Defects and Workarounds.} \\ \hline
  \rowcolor{XyceDarkBlue} \color{white}\bf Defect & \color{white}\bf Description
  \\ \hline \endfirsthead
  \caption[]{Known Defects and Workarounds.} \\ \hline
  \rowcolor{XyceDarkBlue} \color{white}\bf Defect & \color{white}\bf Description
  \\ \hline \endhead
% EXAMPLE:
%\textbf{bug number-SRN}: bug title & Description of KNOWN BUG THAT HAS NOT BEEN
  %FIXED.

%\textbf{\textit{Workaround}}: Describe how to work around this bug.
%\\ \hline
%
%  
\textbf{27-SON}: Fix handling of .options parameters & When specifying .options
  for a particular package, what gets applied as the non-specified default
  options might change.  \\ \hline


\textbf{37-SON}: Connectivity checking is broken for devices with more than 10
leads & The diagnostic code used by the \Xyce{} setup that checks circuit
topology for basic errors such as a node having no DC path to ground or a node
being connected to only one device has a bug in it that causes the code to emit
a cryptic error message, after which the code will exit.  This error has so far
only been seen when a user has attempted to connect a large number of inductors
together using multiple mutual inductor lines.  The maximum number of
non-ground leads that can be used without confusing this piece of code is 10.
If your circuit has that type of large, highly-connected mutual inductor and
the code exits with an error message, this bug may be the source of the problem.

The error message now includes a recommendation to use the workaround below.

\textbf{\textit{Workaround}}: Disable connectivity checking by adding the line

\begin{alltt} .OPTIONS TOPOLOGY CHECK_CONNECTIVITY=0 \end{alltt}

to your netlist.  This will disable the check for the basic errors such as
floating nodes and improperly connected devices, but will allow the netlist to
run with a highly-connected mutual inductor.
\\ \hline


\textbf{49-SON} \Xyce{} BSIM models recognize the model TNOM, but not the
instance TNOM & 
\\ \hline



\textbf{247-SON}: Expressions don't work on .options lines & Expressions enclosed
in braces (\{ \}) are handled specially throughout \Xyce{}, and may only be used
in certain contexts such as in device model or instance parameters or on
\texttt{.PRINT} lines.
\\ \hline


\textbf{250-SON}: NODESET in \Xyce{} is not equivalent to NODESET in SPICE & As
currently implemented, .NODESET applies the initial conditions given throughout
a full nonlinear solve for the operating point. It then uses that result as an
initial guess for a second nonlinear solve with no constraints.  This is not
the same as SPICE, which merely applies the given initial conditions to a
single nonlinear solve for the first two iterations, then lets the problem
converge with no further constraints.  This can lead to Xyce's .NODESET failing
where the same netlist in SPICE might not, if the initial conditions are such
that a full nonlinear solve cannot converge with those constraints in place.
There is no workaround.
\\ \hline


\textbf{468-SON}: It should be legal to have two model cards with the same model
name, but different model types. & SPICE3F5 and ngspice only require that
model cards of the same type have unique model names. They accept model cards
of different types with the same name.  \Xyce{} requires that all model card names be unique.
\\ \hline


\textbf{469-SON}: Belos memory consumption on FreeBSD and excessive CPU on other
platforms &
Memory or thread bloat can result when using multithreaded dense linear algebra
libraries, which are employed by Belos.  If this situation is observed, either build
\Xyce{} with a serial dense linear algebra library or use environment variables to control
the number of spawned threads in a multithreaded library.
\\ \hline


\textbf{583-SON}: Switch with RON=0 leads to convergence failure. &  The switch device does not prevent a user from specifying \texttt{RON=0} in its model, but then takes the inverse of this value to get the ``on'' conductance.  The resulting invalid division will either lead to a division by zero error on platforms that throw such errors, or produce a conductance with ``Not A Number'' or ``Infinity'' as value.  This will lead to a convergence failure.

\textbf{\textit{Workaround}}: Do not specify an identically zero resistance for the switch's ``on'' value.  A small value of resistance such as 1e-15 or smaller will generally work well as a substitute. \\ \hline

\textbf{620-SON}: TO, FROM and TD Qualifiers Function Incorrectly in the FREQ Measure &
The use of these qualifiers in a {\tt FREQ} measure statement may give an incorrect
measurement window.  This is noted in the Reference Guide. \\ \hline

\textbf{621-SON}: FIND-WHEN Measure in \Xyce{} Only Correctly Supports the WHEN Syntax. 
FIND-WHEN is incorrect. & An example is given in the HSpice Compatibility subsection of
the {\tt .MEASURE} section of the Reference Guide. \\ \hline

\textbf{1595-SRN}: \Xyce{} won't allow access to inductors within subcircuits for
mutual inductors external to subcircuits & It is not possible to have a mutual
inductor outside of a subcircuit couple to inductors in a subcircuit.

\textbf{\textit{Workaround}}: Put all inductors and mutual inductance lines that couple to
them together at the same level of circuit hierarchy.
\\ \hline


\textbf{1903-SRN}: \Xyce{} fails to collect several inductors into a linear mutual inductor &
In some rare cases with complex include file usage, the mutual inductor syntax with multiple couplings can fail to work. \Xyce{} will return an error message that it can not find 
\texttt{L\_L1} in this example:
{\tt
\begin{verbatim}
L_L1         node1 node1 inductance1
L_L2         node3 node4  inductance2
L_L3         node5  node6  inductance3
L_L4         node7 node8  inductance4
K_K1         L_L1 L_L2 L_L3 L_L4   .999 
\end{verbatim}
}

\\ \hline


\textbf{1923-SRN}: LC lines run out of memory, even if equivalent (larger) RLC
lines do not. &  In some cases, circuits that run fine using an RLC approximation for a
transmission line, exit with an out-of-memory error if the (supposedly smaller) LC
approximation is used.
\\ \hline

\end{longtable}

